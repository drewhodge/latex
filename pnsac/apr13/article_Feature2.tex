% Template PNSAC newsletter - Article
% Language: Latex
%

% Head

\title{RAPCAN visits CASM}
\subtitle{Retired Airline Pilots of Canada Visit the Canada Aviation and
  Space Museum}
\author{Bill Tate}

\maketitle

At a monthly luncheon meeting of Retired Airline Pilots of Canada,
Captain Jim Strang, retired A-340 Captain, advised me the annual
general meeting for RAPCAN would be held in Ottawa in September
2012. Jim asked me if Project North Star Association of Canada could
assist the Ottawa Chapter of RAPCAN in making sure all would have a
good time.

Our guests would want to visit the North Star, and they would also
like to see the DC-9 and the F-86 in the storage hangar. Many of them
started their early jet experience flying the Sabres and had their
command promotion on the DC-9. These visits were quickly approved,
along with the booking of the Bush Theatre with the assistance of
Stephen Quick, Director General of the CASM.

RAPCAN had no idea how many people would be coming to Ottawa. Based on
previous annual general meeting attendance, from a low of 30 to a high
of 300, a figure of 125 was suggested for planning purposes. Next
question, how many people would want to come to the CASM and see the
North Star versus the Vintage Wings of Canada Air Show?

As we got closer to the date, the North Star visit was chosen by 71
visitors and 75 attended the Air Show. Our own Mike Irvin made sure
that the DC-9 was shown in the best possible way, and PNSAC volunteers
assisted through the day. Among our guests there were several former
North Star pilots for Trans Canada Airlines who added interesting
insights into that aircraft.

Ron Lemieux, former Director of Protocol at Rideau Hall, was
instrumental in ensuring a private tour of the Governor General's
Residence . All members of that tour were impressed with the history
and beauty, and with the kindness of staff who assisted those with
mobility issues.

The day started with Richard Lodge welcoming the group in the Bush
Theatre and an additional welcome by Stephen Quick. Next, a Power
Point presentation highlighted the work to date, followed by the
drawing of a 50/50 draw.

Having some insight of the pilot psyche and knowing there would be a
two hour beer call at the RCAF Mess before dinner I made an
off-the-cuff suggestion. That suggestion was - rather than accept the
50/50 prize and "have fifty of your closest friends" hit you up for
free beer at the Mess, the winner might make a cash donation back to
PNSAC. It was most appreciated that the winner did so, which in turn
made Paul Labranche, our ``guardian of the vault'' very happy as well.

After the tours of the North Star, DC-9, engine shop and restoration
area were finished, eight former F-86 pilots were allowed an up-close
and personal view of their former aircraft. It was amazing to watch
fifty years evaporating on their faces as they went up the step ladder
to see their old airplane. Much to the relief of Jim Strang, the tours
were finished in time to allow for a visitation of the CASM exhibits
and still make beer call!

In conversation with various members of RAPCAN they all seemed
consistent that this was an incredibly detailed restoration that has
far exceeded anything else they have seen. A well deserved ``well done''
for those who work on the North Star as well as the ambassadors of
PNSAC who gave their time to show what we have done on the restoration
of this iconic aircraft.


\begin{footnotesize}
    \raggedleft PNSAC\\
\end{footnotesize}

% End of text.

%%% Local Variables: 
%%% mode: latex
%%% TeX-master: main_document.tex
%%% End: 

