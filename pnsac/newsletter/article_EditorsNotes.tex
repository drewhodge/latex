% Template PNSAC newsletter - Article % Language: Latex
%

% Head



\title{Editor's Notes} 
\author{Bruce Grant}

\maketitle

\end{multicols}

While our project has been on hold, your editor still had to find some
content for the Chronicle.

%This issue of the Chronicle carries the last of four chapters in
%Richard Lodge's story of his time at Rolls-Royce. Taken together, they
%make an interesting ethnology of an extinct industrial culture. The
%whole series is accessible through our website,
%{\color{blue}\texttt{\url{http://www.projectnorthstar.ca}}} (note that
%the location of the newsletter archive will change when the new
%website goes live) starting with the April 2013 issue of the
%Chronicle.

\vspace{12pt}

Jim Riddoch came to the rescue with a story about a
successful/disastrous flight test.  We like these kinds of
stories\textemdash too small and personal to turn up in the history books, but
they do pull you into real human experiences. There will be more of these. 

%We also have another hair-raising flying story from Tim Timmins. A
%series of previous Timmins stories can be found through the same
%website, starting with the December 2010 issue
\vspace{12pt}

Your editor became a writer too, and the writer submitted a tragic
story about a French aviatice. The story was accepted by the editor in
a blatant conflict of interest.

%Motorheads among our readers should look into Ted Devey's extensive
%writings on the Rolls-Royce Merlin, starting in the December 2006
%issue.

\vspace{12pt}

And we bring you a poem by David Lambeth, an ode to the North Star
that was first published in the Chronicle of November 2005. Volunteer
Tim Timmins had found it in a hidey-hole in the aeroplane in the early
days of our restoration project.  Enjoy!

%All together, the archived issues are a valuable web resource for
%readers with an interest in aviation history or the history of Project
%North Star.



\begin{footnotesize} \raggedleft PNSAC\\
\end{footnotesize}

\begin{multicols}{2}

% End of text.

%%% Local Variables: %%% mode: latex %%% TeX-master: main_document.tex
%%% End:

