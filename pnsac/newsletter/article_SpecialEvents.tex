

% Template PNSAC newsletter
% Language: Latex


\title{Quarterly Meeting Report}
\author{Drew Hodge}

\maketitle

On Saturday 7th December 2013, Project North Star Association of
Canada (PNSAC) held a quarterly meeting in the auditorium of the
Canadian Aviation and Space Museum (CASM) in Ottawa. It was a
memorable meeting with over 200 people in attendance. Those in the
audience that day were able to hear the story of the Gimli Glider told
in person by Captain (retd.) Robert (Bob) Pearson, the pilot of Air
Canada Boeing 767-200, fin number 604, when it made an unscheduled
stop at Gimli Manitoba, thirty years ago.

The meeting started with an introduction by Bill Tate, Project North
Star's Vice President.  Bill began by welcoming the audience, which,
as well as PNSAC volunteers, included members of Vintage Wings of
Canada, Ottawa Airport Watch, the Canadian Aviation Historical
Society, the College of Professional Pilots of Canada, CASM volunteers
and staff, Air Canada and WestJet pilots, and visitors from the
general public, one of whom had travelled to Ottawa from Vancouver for
the meeting. After describing the goals of Project North Star and
asking PNSAC volunteers to stand and be recognized, Bill invited the
audience to visit the world class exhibits in the Museum after the
meeting.

Bill's acquaintance with Bob Pearson goes back thirty-five years, when
Bill was Captain Pearson's Second Officer on Boeing 727s based in
Montreal. His duties in the cockpit back then gave Bill ample
opportunities to watch Captain Pearson at work. Bill described Captain
Pearson as a "highly professional pilot" who, "wearing his command
lightly", demonstrated "highly polished interpersonal skills and a
delightful sense of humour".  He then invited Captain Pearson to tell
the story of flying unscheduled into Gimli.

Captain Pearson spoke for a little over an hour with no notes and with
reference to a few appropriate slides.  He described eloquently how,
because of errors in the conversion of metric measurements, his Boeing
767-200 ran out of fuel and lost both engines over Manitoba on the way
from Ottawa to Edmonton. The full story is well documented in the book
"Freefall: From 41,000 feet to zero - a true story, William and
Marilyn Hoffer, Simon \& Schuster, 1989", but to hear it told in
person by the pilot in command that Saturday, 23rd July, 1983, was a
rare privilege.

\begin{figure}[htbp]
   \vspace{2em}
   \centering
   %name of the graphic, without the path AND in EPS format:
   \includegraphics[scale=0.5]{Pearson004.eps}
   %caption of the figure 
   \caption*{\small \em Captain (retd.) Robert Pearson.}
   %label of the figure, which has to correspond to \ref{}:
   \label{fig:pearson004}
\end{figure}

The port engine stopped first at 41,000 feet prompting the decision to
divert to Winnipeg.  At 28,000 feet the starboard engine ran down too,
and Maurice Quintal, Captain Pearson's First Officer, made
calculations that showed they couldn't reach Winnipeg.  Quintal
advised Captain Pearson to try for the nearest runway on the former
RCAF aerodrome at Gimili, twelve miles away.  Some eight or ten
minutes after losing the first engine, Captain Pearson side-slipped
the big Boeing down to an old runway that was being used as a racing
track by members of the Winnipeg Sports Car Club. The nose wheel
collapsed on landing, but the aircraft came to a stop and no one was
seriously injured. The racers helped to put out a minor fire in the
nose.

Captain Pearson talked about the inquiry that followed the incident
and its eventual exoneration of both pilots.  The aircraft was flown
to Winnipeg just two days after the Gimli landing and saw service with
Air Canada for another twenty-five years.  When the Boeing was retired
to the Mohave desert in 2008, Captain Pearson was at the controls to
make that last landing in Arizona.

Following his talk, Captain Pearson answered questions from the
audience.  Richard Lodge, PNSAC President, thanked Captain Pearson and
presented him with a PNSAC golf shirt, a baseball cap, a coffee mug,
and finally a certificate conferring upon him an Honorary Life
Membership of the Project North Star Association.  Captain Pearson
then graciously agreed to make presentations of volunteer-hours
certificates and 10th Anniversary photos to PNSAC volunteers.  The
following volunteers received certificates for the hours shown:

\begin{itemize}
  \item Bruce Gemmill (7000 hrs)
  \item Ted Devey (4000 hrs)
  \item Bill Tate (3000 hrs)
  \item Richard Lodge (1000 hrs)
  \item Bruce Grant (1000 hrs)
  \item Robert D\'{e}sjardins (1000 hrs)
  \item Peter Trowbridge (1000 hrs)
  \item Charles Baril (1000 hrs)
\end{itemize}

\begin{figure}[htbp]
   \vspace{2em}
   \centering
   %name of the graphic, without the path AND in EPS format:
   \includegraphics[scale=0.5]{Pearson007.eps}
   %caption of the figure 
   \caption*{\small \em Richard Lodge presenting Captain Pearon with a
     certificate for Honorary Life Membership in PNSAC.}
   %label of the figure, which has to correspond to \ref{}:
   \label{fig:engine3propcowl}
\end{figure}

To conclude the meeting, Captain Pearson presented signed copies of
the book "Freefall: From 41,000 feet to zero - a true story" to three
audience members whose ticket numbers he had drawn at random.

The Association's sincere thanks go to Bill Tate for organizing this
quarterly meeting and the talk by Captain Pearson -- it was largely
Bill's efforts that made the meeting the great success that it was.


\begin{footnotesize}
    \raggedleft PNSAC\\
\end{footnotesize}

%%% Local Variables: 
%%% mode: latex
%%% TeX-master: main_document.tex
%%% End: 
