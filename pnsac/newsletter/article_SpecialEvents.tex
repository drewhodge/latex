


% Template PNSAC newsletter
% Language: Latex


\title{Doors Open Ottawa 2018}
\author{Roger Button}

\maketitle

Doors Open Ottawa gives the general public an opportunity to view facilities for free which they would
normally not have access. Evidently, it is the second largest event of its kind in North America and one of
the largest in the world with over one million visitors since its inception in 2002. In the case of the
Canadian Aviation and Space Museum its participation meant opening the doors of the reserve hangar.
On June 2nd and 3rd the Museum welcomed some 3000 visitors. The visits of the reserve hangar took
place by way of a self-guided tour supported by many of the Museum's volunteers coordinated by Cedric
St. Amour the volunteer coordinator for the Museums Corporation. Project North Star was ably
represented by Charles Baril, Claire Cameron, Garry Dupont, Ted Devey, Chris McGuffin, Neil Raynor,
Jacques Roy, Nelson Smith, Tim Timmins, and Gary Whitten who ably answered questions about the
aircraft. Rejean Demers, the Conservator Special Projects Manager was also present.

\begin{figure}[httb]
   \vspace{2em}
   \centering
   \includegraphics[scale=0.275]{"PNSAC DOO 2018-scaled".png}
  \caption*{\small \em Volunteers Roger Button, Richard Lodge, Chris McGuffin, Charles Baril and Garry Dupont on hand to explain our project at Doors Open Ottawa June 3rd.}
   \label{fig:stab-one}
\end{figure}


This event is also an opportunity to do a little fundraising for the Association. The merchandise desk was
managed by Richard Lodge, Drew Hodge, and Roger Button. Sales of memorabilia and clothing were
steady throughout the two days. A few memberships were also sold and donations received. The
merchandise table was well positioned near the exit to catch visitors as they left. Unfortunately, the
aircraft was not next to the merchandise table and it was clear that some visitors did not make the
connection and others left without seeing the aircraft. It is hoped that these issues will be addressed for
future viewings of the aircraft. Thanks to all the volunteers who devoted their time to this successful
event.



%\begin{figure}[htbp]
%   \vspace{2em}
%   \centering
%   %name of the graphic, without the path AND in EPS format:
%   \includegraphics[scale=0.9]{Pearson004.eps}
%   %caption of the figure 
%   \caption*{\small \em Captain (retd.) Robert Pearson.}
%   %label of the figure, which has to correspond to \ref{}:
%   \label{fig:pearson004}
%\end{figure}
%
%
%
%\begin{figure}[htbp]
%   \vspace{2em}
%   \centering
%   %name of the graphic, without the path AND in EPS format:
%   \includegraphics[scale=0.9]{Pearson007.eps}
%   %caption of the figure 
%   \caption*{\small \em Richard Lodge presenting Captain Pearson with a
%     certificate for Honorary Life Membership in PNSAC.}
%   %label of the figure, which has to correspond to \ref{}:
%   \label{fig:engine3propcowl}
%\end{figure}



\begin{footnotesize}
    \raggedleft PNSAC\\
\end{footnotesize}

%%% Local Variables: 
%%% mode: latex
%%% TeX-master: main_document.tex
%%% End: 
