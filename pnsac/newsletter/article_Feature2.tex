% Template PNSAC newsletter - Article % Language: Latex
%

% Head

\title{The Civil Merlin} 
\author{Dr. Jakob Whitfield}
\maketitle

\textit{The following article is an extract from a larger piece which appeared
in the July 2017 edition of Aeroplane magazine under the heading
"Rolls-Royce Merlin."  We wish to thank Allan Bowes for bringing it to
our attention. This article is republished with the kind permission of
the publishers of the magazine and the author Dr. Jakob Whitfield.}\\


The earliest Merlins to operate in a civil mode were the Merlin T24
series, developed in 1944. These were single-stage twin-speed units
similar to the Merlin 24s fitted to the RAF's Lancasters, but were
modified to improve service life, and were fitted to Transport
Command's Avro Yorks. Long-range transport operation entailed running
at relatively low cruise powers for long periods---under these
conditions the lower cylinder head temperatures caused deposits of lead
oxide from the fuel, resulting in excessive spark plug fouling. To
counter this, the Merlin T24/4 incorporated a charge heater to increase
the inlet temperature. Post-war, the Merlin 500 series essentially
comprised civil and export versions of the T24, incorporating its
modifications, though only the Merlin 501 included a charge heater.

\begin{figure}[httb]
   \vspace{2em}
   \centering
   \includegraphics [scale=0.5]{merlin-1-feb-2020.jpg}
   \caption*{\small \em Rolls-Royce Merlin Engine}
   \label{fig:wall-two}
\end{figure}

The 600 and 700-series engines were two-speed two-stage engines, based
on the military 100-series. They were fitted with the so-called
'transport heads and banks', strengthened for greater reliability. Most
marks had some form of variable intercooling to allow for charge
heating to reduce plug fouling under cruise conditions. Initially the
system was plagued by coolant leaks, but as experience showed that zero
intercooling at cruise allowed enough charge heating to reduce leading,
a simple stop valve was fitted to the system. This allowed full
intercooling for take-off and climb, which could then be turned off for
cruise.

Civil Merlins were fitted to Avro Lancastrians, Yorks and Tudors, but
their flagship role was on the Canadair DC-4M North Star, a Douglas
DC-4-derived design intended for Trans-Canada Air Lines (TCA). The
North Stars used a modified Universal Power Plant installation, as the
DC-4's nacelle bulkheads were slightly larger than the SBAC standard.
In practice it turned out the Merlins were not ideally suited for use
in civil operations: though the North Stars flew higher and faster than
DC-4s, their time between engine overhauls was around 850 hours,
roughly a third of that of commercial US radial engines. The engines
also produced a lot of cabin noise, though this was alleviated somewhat
by revised 'cross-over' exhausts that ducted the cabin-side stacks
outboard, developed first by TCA and then by Rolls-Royce themselves.

\begin{figure}[httb]
   \vspace{2em}
   \centering
   \includegraphics[scale=0.5]{nstar-tca-feb-2020.jpg}
   \caption*{\small \em TCA Canadair DC-4M North Star}
   \label{fig:wall-two}
\end{figure}

The extra running costs were mostly borne by Rolls-Royce. When TCA's
managing director expressed dissatisfaction with the Merlin's
commercial performance, Hives asked what a reasonable level of
maintenance cost would be. Being told \$4 per engine hour, he agreed to
service the fleet's engines for this amount. This early form of 'power
by the hour' was initially expensive for Rolls-Royce, but by the end of
the Merlin's life the company had learned enough to supposedly make a
small profit at this level.

What was undoubtedly true was that Rolls-Royce learned a great deal
about the harsh realities of commercial operation in a short time.
Hives is supposed to have said to TCA, "We didn't know the Merlin until
you started operating it!" In the longer term, this paid off,
TCA---later Air Canada---selecting Rolls--Royce engines for its future
fleet: Darts in Viscounts, Tynes in Vanguards, Conways in DC-8s, and
RB211s in TriStars.


\begin{footnotesize} \raggedleft PNSAC\\
\end{footnotesize}

% End of text.

%%% Local Variables: %%% mode: latex %%% TeX-master: main_document.tex
%%% End:

