% Template PNSAC newsletter - Article
% Language: Latex
%

% Head

\title{RCAF Overseas Ferry Operations}
\author{AJS (Tim) Timmins, RAD/NAV/AI}

\maketitle

Over a three year period starting in 1952, with Operation Leap Frog,
the deployment of 12 Sabre squadrons from Canada to Air Division in
Europe, No 1 Overseas Ferry Unit (1OFU), based at St Hubert, Quebec,
delivered over 800 aircraft. 426 Squadron North Stars flew support
missions, carrying servicing personnel and providing enroute
navigation assistance, airborne radio beacons known as duck
butts. Besides the Sabre squadrons, four CF100 Squadrons were
delivered to Air Division and 53 CF100 aircraft to the Belgium Air
Force. Silver Stars and Expeditors were flown to Europe for delivery
to NATO members under the Mutual Aid agreement. My personal experience
with ferry operations was with Beechflight, the delivery of 25 C45s
(Expeditors) to France and Portugal.

\begin{figure}[htbp]
   \vspace{2em}
   \centering
   %name of the graphic, without the path AND in EPS format:
   \includegraphics[scale=0.5]{rcaf_c45.eps}
   %caption of the figure 
   \caption*{\small \em RCAF C45 Expeditor a military variant of the Beech 18.Purchased in large numbers during WW II and after for pilot training, navigator training, and as a utility transport.}
   %label of the figure, which has to correspond to \ref{}:
   \label{fig:rcaf_c45}
\end{figure}

Beachflight was a Training Command operation, under the command of
Wing Commander Harry Forbell. Twenty six expeditors were assembled at
Trenton where they were fitted with a 100 gallon fuel tank in the main
cabin, plus safety and navigation gear required for the planned
route. A pilot and navigator were assigned to each aircraft. I flew
with Squadron Leader Brian, Marfleet, the designated Deputy Operation
Commander. The aircraft were to fly in formation, four sections of
five and one of six, within block airspace. 

On 20 May 1959 Operation Beachflight was launched. Sections departed
Trenton for Goose Bay at designated intervals in order to maintain
separation along the route. Big trouble, the weather within the
assigned block airspace deteriorated to the point where aircraft could
not maintain contact with each other and abandoned their
formations. Now it was everybody for himself. The 26 Expeditors were
up, down, right, left, within the block airspace and no doubt outside
it. My section dispersed and we did not see any of them until we
arrived at Goose Bay. Not a great start to the operation, but everyone
arrived safely at Goose Bay, so a success in that sense. Our flight
time was 6:45.

The flight from Goose Bay to Frobisher (Iqualuit) the next day was
uneventful. Our Flight time was 6:05.

As I recall, we had a weather delay in Frobisher and did not depart
until 24 May. The plan was for each section to proceed to Sondrestrom
AFB (Kangerlussuag-big Fjord) located on the West coat of Greenland,
refuel and carry on over the icecap to Keflavik. The Deputy Commander
would lead the 5th and last section. Our aircraft was unserviceable so
we had to take the spare, the 26th aircraft. The unserviceable
aircraft was repaired and returned to Trenton. We were late into
Sondrestrom, to find that section four was still on the ground because
the three earlier sections had experienced heavy icing over the ice
cap.  It was not known if the crews and aircraft were safe, a great
concern as the Expeditor was a notoriously poor performer in icing
conditions. We assembled in the bar to await word on the fate of our
mates. All the aircraft made it over the ice cap and arrived safely in
Keflavik, which was ample reason for us to celebrate another
success. Our flight time to Sondrestrom was 3:50.

We departed Sondrestrom on 27 May after a two day delay waiting for
improved weather over the ice cap. The flight to Keflavik was
uneventful. Our flight time was 5:10.

The first three sections waited for us in Keflavik. Every crew had a
story to tell about their experience over the ice cap. Fifteen
aircraft were loaded with ice and unable to maintain altitude. Their
only salvation was to make it past the ice cap and out over the water
where they could descend to warmer air and shed the ice. Thankfully,
all of them did.

The east coast of Greenland is very rugged with numerous high
peaks. Several crews in their descent out of the icing conditions
broke cloud between these peaks. These stories were oft repeated and
all called for another round of drinks. 

The next day, 28 May, all the sections proceeded to Prestwick. The
weather was perfect and I got a lot of stick time because my pilot was
very tired after the debriefings in Keflavik. Our flight time was
5:10.

We remained overnight in Prestwick and flew to 1 Wing, Marville,
arriving mid afternoon on 29 May.  The aircraft were prepared for
delivery to the French and Portugese air forces. The cabin fuel tanks
were removed as well as all the safety and navigation gear.
 
On 3 June we delivered 19 Expeditors to the French Air Force at
Chateaudun, located not far from Chartres. The flight crews were
picked up by C47 Dakotas and returned to Marville. Nineteen crews
boarded a 426 Squadron North Star for return to Trenton. Of course the
Commander and his Deputy remained behind to deliver six expeditors to
the Portugese Air Force in Lisbon.

We departed Marville on 4 June for Bordeaux where we planned to refuel
and then proceed along an air route over the Pyrenees Mountains to
Lisbon. The weather along the planned route was very bad with
intensive thunder storms over the mountains. Ever resourceful, our
Commander decided to file a flight plan west over the Bay of Biscay,
clear of Spain, then south to Lisbon. This would be a long overwater
flight and we had no extra fuel, safety equipment or nav gear. We took
off and were in the process of forming up in two sections of three,
when several light military aircraft flew through our formation. No
one saw them coming, it happened so fast there was no reaction. Not a
very good start to the day.

Our Commander led us west over the Bay of Biscay, maintaining VFR
under a cloud layer. The cloud  base got lower and lower, we were down
with the shipping and it was decided to climb above the cloud, VFR on
top. This worked for a while but the cloud tops kept creeping up over
10,000 feet and we had no oxygen. The Commander decided to descend and
try for better conditions. The Deputy Commander, my pilot, decided to
stay high; our six plane formation was split three high and three
low. I was now doing the navigation for the high section. With nothing
but a chart to work with, I managed to make use of the marine beacons
along the coast to guide us west then south to Lisbon. We arrived at
Lisbon before the low section and the Deputy Commander wisely elected
to hold until the Commander arrived and landed. Our flight time was
5:45.

The next day we participated in a small handing over ceremony. Mission
accomplished!



\begin{footnotesize}
    \raggedleft PNSAC\\
\end{footnotesize}

% End of text.

%%% Local Variables: 
%%% mode: latex
%%% TeX-master: main_document.tex
%%% End: 

