% Template PNSAC newsletter - Article
% Language: Latex
%

% Head

\title{Elise D\'{e}roche, Aviatrice}
\author{Bruce Grant, Fl\^{a}neur}

\maketitle

Wandering about in Paris recently I came upon a plaque, bolted to a
wall at 61 Rue de la Verrerie in the 4th. Wandering about is my
favourite activity in Paris. Let others wait in the long lines at the
tourist sites (I've done all that) or plunder the shops in Faubourg
Saint Honor\'{e} (I'll never do that); I wander about.

So, who was this Elise D\'{e}roche a.k.a. Baronne Elisa Raymonde de
Laroche, Aviatrice?  I took the picture and resolved to look her up at
home.

\begin{figure}[htbp]
   \vspace{2em}
   \centering
   %name of the graphic, without the path AND in EPS format:
   \includegraphics[scale=0.5]{aviatrice.eps}
   %caption of the figure 
   %\caption*{\small \em Garry installing the intercooler on engine 3.}
   %label of the figure, which has to correspond to \ref{}:
   \label{fig:aviatrice}
\end{figure}

Here's a brief outline of the story I encourage readers to look
for her online.  Many sites relate her story though with some
contradictions among the various accounts.

Born into the working class family of a plumber, Elise D\'{e}roche
became an artist and actress, took the stage name of Baronne Elisa
Raymonde de Laroche. She also became the "amie intime" of the artist
and aviator L\'{e}on Delagrange who introduced her to his friend Charles
Voisin, a builder of airplanes.

There is an interesting chronology that indicates a life full of
tragedy, a life that she lived with intensity and audacity: L\'{e}on
Delagrange died, January 1910 when the wing came off his Bl\'{e}riot
airplane. Elise recovered quickly from her grief and appealed to her
new ami intime Charles Voisin to teach her to fly. She won her pilot's
licence in March 1910. A crash in July 1910 put her in hospital with
18 fractures. She recovered and returned to flying.

In a car crash in September 1912, Charles Voisin was killed but
Elise survived. She maried the pilot Jacques Vial in February of
1915, her first marriage. During the war, she was not allowed to
fly but served in the military as a driver. In 1918 her only
child Andr\'{e} (with L\'{e}on Delagrange) died aged 15 of the Spanish
Flu. July 1919, she was killed in a plane crash while training to
become a test pilot. Audace? Mais oui!

La Baronne Elisa Raymonde de Laroche is buried in Cimeti\`{e}re
P\`{e}re Lachaise in Paris.

\vspace{10mm}

\begin{centering}
My candle burns at both ends;\\
It will not last the night;\\
But ah, my foes, and oh, my friends --\\
It gives a lovely light!\\
\vspace{2mm}
\begin{footnotesize}
Edna St. Vincent Millay
\end{footnotesize}

\end{centering}

\begin{footnotesize}
    \raggedleft PNSAC\\
\end{footnotesize}

% End of text.

%%% Local Variables: 
%%% mode: latex
%%% TeX-master: main_document.tex
%%% End: 

