% Template PNSAC newsletter - Article
% Language: Latex
%

% Head

\title{Project Manager's Progress Report}
\subtitle{April 2015}
\author{Bruce Gemmill}

\maketitle

Since our last Project Report, our project has gone through much
upheaval, and work has been delayed for a number of reasons.  However,
I can report now that we are back on track and making steady progress
on several parts of the North Star. This report tracks progress since
April 2014.

\section{Restoration Shop Shutdown}
\label{sec:restoration_shop}

Engine \#3 was in the final stages of assembly when we had to shut down
our project at the end of December 2013, to make way for other museum
work, in preparation for events to mark the 100th anniversary of the
beginning of the First World War in 2014.  In particular, new exhibits
in the museum required all of the available floor space in the shop,
as well as full time work by all museum conservation staff, meaning
the volunteer crew could not be supervised as usual.

However, two volunteers, myself and Peter Trobridge, were allowed to
work on the aircraft during the shutdown between January and July
2014.  The focus was on completing the assembly of the cockpit and
crew lounge.  More on this later.

\section{Engine Frame}
\label{sec:engine_frame}

Nr 3 engine frame was completed after the installation of the engine
and supercharger.  Fitting the many cowl panels proved difficult, due
to the extensive rework that many of these items went through, causing
some alignment difficulties when fitting these panels onto the engine
frame.  This work was done after work resumed in July 2014, and the
completed engine was successfully installed on the aircraft last
summer.

%\begin{figure}[htbp]
%   \vspace{2em}
%   \centering
%   %name of the graphic, without the path AND in EPS format:
%   \includegraphics[scale=0.5]{fwd_cockpit_complete.eps}
%   %caption of the figure 
%   \caption*{\small \em Forward cockpit -- complete.}
%   %label of the figure, which has to correspond to \ref{}:
%   \label{fig:fwd_cockpit_complete}
%\end{figure}

\section{Nr 4 Engine}
\label{sec:engine_4}

The last engine to undergo restoration was removed in July and work
immediately began in the engine shop to disassemble the engine frame
and move the engine to the engine stand for restoration.  All cowl
panels, pipes, hoses and accessories have been removed and stored
until space is available to restore these items.  The priority has
been to disassemble the engine, which was completed in October, and
restoration and reassembly of engine 4 is now well underway.  The
refurbished crankshaft has been installed in the engine block, and the
connecting rods and pistons cleaned and polished.  Work has now begun
on the more complex cylinder heads and valve train assemblies.

Normally, work would also be underway to restore the engine frame, but
due to a lack of space, this will be delayed until the engine is back
together and can be moved out of the engine shop.  Some cleaning and
paint stripping has been done, but further work on the frame will
likely have to wait until more space is available for full restoration
of this large item.  This work will likely delay the completion of Nr
4 engine until 2016.

\section{Cocpit, Crew Lounge, Galley,  and Forward Washroom}
\label{crewlounge}

Since our last report, the crew lounge and galley equipment was fully
restored.  A new wall was installed in the galley, as the old wall
contained asbestos and needed to be removed for safety reasons.  The
galley cupboards were badly corroded, so a new set was built, with
only the original doors remaining.  This was installed in the galley,
along with the auxiliary hydraulic reservoir, after two sections of
the floor were patched to repair corrosion damage.  New flooring was
cut and glued in the crew lounge and washroom, then the washroom
equipment was installed, including the toilet and washroom door, and
the vanity sink and mirror.

The five crew seats that had been removed early in the project were
finally installed in the cockpit, after installing the radio operator
and navigator tables, and the appropriate seat rails.  We also
installed new cockpit curtains, and completed these crew positions by
installing a donated headset and several microphones, along with a
Morse code key at the radio position.

The seat rails and seats for the crew lounge were installed, along
with the crew table and secure stowage case.  The crew bed was
installed above the crew seats, and then the curtain rod with a new
set of curtains was installed to complete the crew lounge.  The access
door to the crew area has not been fully restored, but it was
re-installed with a new lock to secure the forward portion of the
aircraft.  The cockpit, crew lounge and forward washroom are now
complete.  This is a significant milestone for the project, and has
been very popular with select museum visitors who have been allowed
access.

\section{Fuselage and Main Cabin}
\label{sec:main_cabin}

The baggage compartments are now fully restored, and the refurbished
battery elevators were installed just aft of the nose wheel.  Work has
now begun on the main cabin.  This started with removal of various
fittings along the starboard side of the cabin, then removing the wood
paneling below the windows, and then removing the wood cargo floor
panels to expose the metal sub floor.  Most of the side panels and
some floor panels are badly damaged and will need to be replaced.  The
fittings for patient litters and troop seats were also removed.
Finally, the starboard windows were removed.  Once all the fittings
were removed, work began on repairing corrosion damage along the
exposed ribs and sub floor.

The rear washroom was also dismantled, after removing the toilet and
fittings.  The floor under the toilet was badly corroded, and the old
floor panel was removed.  This required drilling out approximately
1000 rivets!  A new panel has been made and will be installed shortly.
Portions of the corroded rear bulkhead were also removed, so that new
panels can be installed.  Some items were also removed from the tail
section, to allow for thorough cleaning and spot priming bare metal
where necessary.  During this work damage was discovered on the torque
tube that controls the movement of the rudder.  This was unexpected,
and will require removal of the tail cone to allow for a new torque
tube to be fitted.

Several badly corroded sections of the starboard side of the cabin
have been repaired, and work continues with corrosion repairs and
other cleanup while the aircraft is inside.  Part of the forward cargo
door has been removed for restoration.  The paratroop door insert is
complete, and the remaining portion of the forward cargo door is being
stripped and repaired prior to painting.  It is likely this will have
to be re-installed on the aircraft before all painting and restoration
is complete, so as not to delay it’s move outside.

A major effort was launched to restore the large heater ducts removed
from the main cabin.  These were covered with formed mica insulation
sections, which were removed earlier.  Many of these sections fell
apart and could not be reused.  After many repairs to the metal
ductwork, new sections of mica were formed in a jig and glued to the
ducts.  These duct sections are in storage waiting for completion of
the main cabin so they can be installed on the ceiling.

Work has now begun on the port windows and wall sections.  We hope to
complete the windows soon, as these must be installed before the
aircraft can be moved outside.

\section{Cabin Liners and Troop Seats}
\label{sec:liners_troop_seats}

The original cabin liners were removed from the aircraft earlier in
the project.  These had been washed and stored until repair work could
be done.  All liners were unrolled, photographed and assessed for
damage.  Some liners will be repaired, while others, particularly the
cargo door liners, need to be replaced because the fabric is too badly
deteriorated.  A sewing station has been set up in the storage hangar,
and several of the new liners have been completed.  These will need to
have new fasteners installed, then painted to match the original
liners.

The North Star was originally fitted with troop seats down both sides
of the main cabin.  These could be stowed against the wall to make
room for cargo, or unrolled and set up to carry military personnel.
The original seats were removed by the RCAF, so a decision was made to
make a new set, using funds donated by Project North Star members.
Seat fabric, steel and aluminum tubing was ordered, and the troop seat
sections were fabricated, based on available drawings.  All seat
sections have been completed, and have been stored until the main
cabin is completed.

\section{Planned Restoration Work---2015}
\label{sec:plannedwork}

Over the next year, we will continue to work on Engine 4 and the main
cabin and tail section.  The aircraft will be moved outside this
summer for some interior painting.  This means the windows and doors
will need to be installed for weatherproofing and security.  During
this time we also hope to begin fabricating new wall and floor
sections.  One other work item that may be started is to clean and
restore the four engine nacelles.  Pipes and fittings need to be
removed, followed by a thorough cleaning and a fresh coat of paint.
This work may be delayed until qualified volunteers become available.

\begin{footnotesize}
  \raggedleft PNSAC\\
\end{footnotesize}

% End of text.

%%% Local Variables: 
%%% mode: latex
%%% TeX-master: main_document.tex
%%% End: 

