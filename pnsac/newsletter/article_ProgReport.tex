% Template PNSAC newsletter - Article
% Language: Latex
%

% Head

\title{Project Progress Report}
\subtitle{PNS 2016 Status Update}
\author{Bruce Gemmill, Peter Trowbridge, Garry Dupont}

\maketitle

\section{Nr 4 Engine}
\label{sec:engine_4}

Work has continued on much of the ancillary equipment for the engine. The
supercharger and intercooler have been completed and installed, as well as the
reduction gearbox. The radiators and header tank have been restored and painted, and
two of the three radiator sections have been installed. Restoration work has
begun on the air intake system, and many of the fuel and oil lines. 

\begin{figure}[htbp]
   \vspace{2em}
   \centering
   \includegraphics[scale=0.5]{"PNS 007-im2".png}
   \caption*{\small \em Volunteer Peter Trowbridge inspecting the air intake fairing from engine \#4 prior to beginning restoration work.}
   \label{fig:engine_no_4_2}
\end{figure}

\begin{figure}[htbp]
   \vspace{2em}
   \centering
   \includegraphics[scale=0.5]{"PNS 003-im3".png}
   \caption*{\small \em Volunteers Robert Desjardins, Bruce Gemmill and Michel Cote hold the refinished coolant tank from engine \#4.  Hours of polishing were required to get the tank to look like new.}
   \label{fig:coolant_tank}
\end{figure}

The starter motor was found to be very corroded, and disassembly and repair not viable. A
new unit has been ordered and will be installed once delivered. The fire
suppression lines are complete and will be installed along with the cowls when
ready. Some of the cowl panels have been restored, although it has been found
that this engine has more corrosion than the other three engines, possibly it
had been in service longer? This has meant that more work is required to
refurbish each of the many panels. We still hope to have this engine ready for
installation by next spring. 

\begin{figure}[htbp]
   \vspace{2em}
   \centering
   \includegraphics[scale=0.5]{"PNS 008-im1".png}
   \caption*{\small \em Volunteer Garry Dupont working on the re-assembly of engine \#4.}
   \label{fig:engine_no_4}
\end{figure}

\section{Fuselage and Main Cabin}
\label{sec:main_cabin}

With the aircraft moved outside in March, little work was done in the aircraft
over the winter. Once the temperatures warmed up, we continued work in the main
cabin to prepare for completion of the painting started last year. Repairs to
the rear bulkhead are now complete with the installation of several new pieces
of aluminum to replace corroded ones. A new section of flooring was installed
where the rear washroom was situated. This required replacing about 1000 rivets!

\begin{figure}[htbp]
   \vspace{2em}
   \centering
   \includegraphics[scale=0.5]{"pns 043-im4".png}
   \caption*{\small \em A single new floor panel at the rear of the main cabin required over 1000 rivets.  
   Cleco clips are used to accurately align the panel during riveting.}
   \label{fig:panel_revets}
\end{figure}

Cleaning and corrosion removal continued throughout the cabin. It seems whenever
we feel we are near completion, we find more corrosion. We had also planned to
finish priming the floor and painting the rear bulkhead before the aircraft
returned to the hanger later in the year, but this was delayed by the shutdown
of the project while a new Project Manager was hired by the museum. The wood
floors and walls still need to be repaired, or new panels made, and all will
require painting. 

\section{Cabin Liners}
\label{sec:liners}

All the large pieces of liner have either been repaired or replaced, but with
limitations on space, most remaining work on the cabin liners has been
postponed. New window surrounds have also been made, but then all liners will
require painting and stenciling, and this must wait until there is more room in
the shop and the paint booth. Liners were also made for the top of the heater
ducts.

\begin{figure}[htbp]
   \vspace{2em}
   \centering
   \includegraphics[scale=0.5]{P1020369-im5.png}
   \caption*{\small \em One of the refurbished heater duct sections with a new fabric headliner secured to the top. The sides are covered with mica insulation that was removed, cleaned and reattached after repairs to the metal duct. These will be installed in the main cabin.}
   \label{fig:heaters_liners}
\end{figure}

\section{PNS 2017 Update}
\label{sec:pnsupdate}

We were able to restart work on the aircraft in late January, following the
arrival of our new Project manager, Rej Demers. So far, we have completed the
top cowl panels and one of the radiator flaps, which required a repair patch to
the outer skin, and the electrical harnesses, which are now waiting for
painting. Work continues on the air box and the remaining cowl panels.

We have decided not to move the aircraft outside this year. With our efforts
concentrated on completing engine \#4, it is unlikely we would have the resources
to also complete the repair work in the cabin needed to allow for priming and
painting, which is the main reason for moving the aircraft outside in the summer.


\begin{footnotesize}
  \raggedleft PNSAC\\
\end{footnotesize}

% End of text.

%%% Local Variables: 
%%% mode: latex
%%% TeX-master: main_document.tex
%%% End: 

