% Template PNSAC newsletter - Article
% Language: Latex
%

% Head

\title{Project Manager's Progress Report}
\subtitle{April 2014}
\author{Bruce Gemmill}

\maketitle

Due to museum preparations for a display to mark the beginning of
the First World War and other events in 2014, the restoration
space used by Project North Star was taken over by other museum
projects, and most work on the North Star was suspended.
However, some limited work did continue.

\section{Nr 3 Engine}
\label{sec:engines_3}

Work on Engine 3 was halted for several weeks in early 2014, but
has now resumed on a reduced work schedule.  Most accessories are
now installed, and soon the cowl panels will be fitted.  It is
expected that Engine 3 will be installed late this spring or
early summer, so that work can begin on removing and restoring
Engine 4.

This downtime also offered the opportunity for the engine crew to
clean and re-organize the engine shop.

\begin{figure}[htbp]
   \vspace{2em}
   \centering
   %name of the graphic, without the path AND in EPS format:
   \includegraphics[scale=0.5]{Eng3Intercooler.eps}
   %caption of the figure 
   \caption*{\small \em Garry installing the intercooler on engine 3.}
   %label of the figure, which has to correspond to \ref{}:
   \label{fig:Eng3Intercooler}
\end{figure}

\section{Cockpit}
\label{cockpit}

While restoration work in the shop could not be continued, we
used the opportunity to install the crew seats and other restored
items to the cockpit.  The pilot seat was installed (first
officer seat was installed last year) then the railings and
safety board were installed behind the seats.  Next we installed
the radio officer and navigator tables, and the seats for these
positions.  The radio officer seat proved to be a particular
challenge, as the floor in this area is no longer flat, and some
modifications to the seat rails were necessary to allow the seat
to slide.  Finally the flight engineer’s jump seat was attached
to the railing.  With five crew seats installed, the cockpit is
very cramped.

The cockpit installations were finished off by installing the
curtains, crew oxygen hoses, and a Morse code key at the radio
position.

\section{Crew Lounge, Galley,  and Forward Washroom}
\label{crewlounge}

The floor under this section was repaired by removing a section
of badly corroded “I” beam and fabricating a new one from several
pieces of aluminum which were riveted in place to support the
floor.  Once this repair was completed, new cushion flooring was
cut and glued to the crew lounge and the washroom floor, then
various pieces of floor trim were installed.  To install the crew
seat, it was necessary to remove the lower support section, since
the fully assembled seat could not easily fit through the narrow
doorway.  Once the seat was installed, we added the crew table
and bunk bed, along with the secure storage box.  We also
installed the curtain rod and new curtains to close off the bunk
bed. The installation was finished off by re-installing the door
between the lounge and the main cabin, so that this fully
restored area can be protected from dust and debris while the
main cabin undergoes restoration over the next two years.

The completion of the forward area of the aircraft marks a
significant milestone in the restoration of the North Star.

\section{Planned Restoration Work 2014}
\label{sec:plannedwork}

Some limited work is being done to remove fittings and wall
panels from the main cabin, in preparation for some major
restoration work this summer.  Fabric work on the troop seats is
nearing completion, although the seats will not be installed
until all work on the main cabin is complete.  We hope to resume
a full work schedule by June, with the aircraft outside once
more.

\begin{footnotesize}
  \raggedleft PNSAC\\
\end{footnotesize}

% End of text.

%%% Local Variables: 
%%% mode: latex
%%% TeX-master: main_document.tex
%%% End: 

