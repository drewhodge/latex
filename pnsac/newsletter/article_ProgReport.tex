% Template PNSAC newsletter - Article
% Language: Latex
%

% Head

\title{Project Manager's Progress Report}
\subtitle{November 2015}
\author{Bruce Gemmill}

\maketitle

The aircraft spent considerable time outside this year, but since we
are short of volunteers, not as much work was done as hoped.  The
aircraft is still outside because we are waiting for space to be freed
up in the storage hangar.  We also expect the North Star to be moved
outside again in early spring, much sooner than normal, to accommodate
other uses of the storage facility.  This extended time outside has
delayed some work, and may result in additional weather damage.

\section{Nr 4 Engine}
\label{sec:engine_4}

The last engine to undergo restoration was disassembled last fall.
Since the engine shop crew have cleaned, painted and re-assembled much
of the engine on the engine assembly stand. The refurbished crankshaft
has been installed in the engine block, and the connecting rods and
pistons cleaned and polished.  Also completed are the more complex
cylinder heads and valve train assemblies, and work has begun to
restore many ancillary pieces, such as the magnetos and other
electrical components, and accessories.

The auxiliary gearbox was removed from Nacelle 4 and has been
completely rebuilt, along with the generator.  These won't be
installed on the aircraft until shortly before engine 4 is ready to be
installed.

Work has also begun on the engine frame which holds the 12 cylinder
engine and supports the surrounding cowl panels.  Once the engine
frame is ready, the engine will be transferred to it from the engine
stand.  Because of a lack of space and shortage of qualified
volunteers, we don't expect to have engine 4 completed before 2017.  A
great deal of work is also needed on the cowl panels.  We hope to
start work on these in the New Year.

\section{Fuselage and Main Cabin}
\label{sec:main_cabin}

Work has continued in the main cabin.  After removing all floor and
wall panels, as well as numerous fittings, the floor was cleaned and
all cargo tiedowns thoroughly treated to remove corrosion.  Several
required removal and repair. The starboard windows were repainted and
new Plexiglas installed, then the windows were re-installed on the
aircraft.  The port windows were then removed and repaired, but had to
be installed prior to completion so the aircraft could stay outside
this summer.  These will need to be removed again this winter and the
new glass installed before being returned to the aircraft.  Some
corrosion damage along the exposed ribs and sub floor has been
completed.  The rear bulkhead required considerable work.  This
involved removing a large corroded floor panel and three rear bulkhead
panels.  New panels were made and are being installed.

The forward bulkhead and forward half of the floor have been painted,
but further painting was stopped by the arrival of cold weather in
October.  The painting will have to be completed next summer.

The forward cargo door was cleaned, painted and new seals installed.
The door was put back on the aircraft to secure it while outside this
summer.  This winter we plan on removing the rear cargo door for
similar work.

The tail section and rear belly compartment underwent some cleaning
and repair.  This work will also continue this winter in the storage
hangar.

\section{Cabin Liners and Troop Seats}
\label{sec:liners_troop_seats}

Most original cabin liners have now been repaired or replaced with new
liners.  The liners still need to be painted and many stencils need to
be applied, as well as many new snap fasteners.  

The troop seats were built last year and placed in storage.  Some work
has been done to restore the many fitting needed to install the troop
seats, along with litter straps and tiedowns for litters and cargo.

\section{Planned Restoration Work---2016}
\label{sec:plannedwork}

Over the next year, we will continue to work on Engine 4 and the main
cabin and doors.  The aircraft will be moved outside in the late
winter or early spring to allow setup for a planned Star Trek exhibit.
Work will not begin inside the aircraft until the weather warms up, so
further delays are inevitable.  

We will be able to make new wall panels and either replace or restore
the wood floor panels.

We hope to have all main cabin painting done this summer, so that the
wall and floor panels can be installed.  

One other work item that may be started is to clean and restore the
four engine nacelles.  Pipes and fittings need to be removed, followed
by a thorough cleaning and a fresh coat of paint.  This work may be
delayed until qualified volunteers become available.

\section{March 2016 Update}
\label{sec:update}

Since our report at the November AGM, the aircraft was moved inside,
so that it could properly sealed against the weather, as it will spend
much of 2016 outside. The aircraft was moved out again in March.
During this time, very little work was done on the aircraft, except
for replacement of the Plexiglas in the port windows, and the
installation of the ARC-552 UHF radio, kindly transferred to the
museum by the Department of National Defence.

\subsection{Nr 4 Engine}
\label{sec:update-engine}

The engine frame has been refurbished, and the engine was transferred
to the frame, so that ancillary equipment can be installed.  Work has
also begun on the supercharger, as well as such items as the starter
motor and header tank.  Some work is also being done on the cowl panels.

%\begin{figure}[htbp]
%   \vspace{2em}
%   \centering
%   %name of the graphic, without the path AND in EPS format:
%   \includegraphics[scale=0.5]{fwd_cockpit_complete.eps}
%   %caption of the figure 
%   \caption*{\small \em Forward cockpit -- complete.}
%   %label of the figure, which has to correspond to \ref{}:
%   \label{fig:fwd_cockpit_complete}
%\end{figure}


\begin{footnotesize}
  \raggedleft PNSAC\\
\end{footnotesize}

% End of text.

%%% Local Variables: 
%%% mode: latex
%%% TeX-master: main_document.tex
%%% End: 

