% Template PNSAC newsletter - Article
% Language: Latex
%

% Head

\title{Conservator's Corner}
%\subtitle{PNS 2016 Status Update}
\author{R\'{e}jean Demers, AME---Conservator \/ Special Project Manager
Canada Aviation \& Space Museum}
%\author{Restoration Project Manager}

\maketitle

%\section{Zen and the Art of Aircraft Restoration.}

When touring the shops in the Museum's Conservation wing, the engine shop is
reserved for what is referred to as, the "Crown Jewel". Guests are lead through
the corridor and the lights are flicked on to reveal a shining piece of
aeronautical engineering and craftsmanship. The Merlin 622 engine, one of four
such power plants, sits resplendent. Gasps and expressions of delight from
guests are guaranteed to insight a prideful glance upon this piece of work by
volunteers and staff alike. Appearing as if the engine had just been delivered
from Bristol Aero Ltd. is an illusion. What is this strange magic? How can such
beauty manifested though steel and aluminum even exist? As any good sorcerer
will tell, illusion is lost upon the dissolution of mystery. Yet this piece of
work, a restoration of great effort, time and detail, cannot shed the cloak of
timelessness so easily. The result of thousands of hours culminate through
unrelenting intrigue, garnered by all who cast eyes upon the Merlin. These
words simply scatter as embellishments to the enduring design and purpose,
which cannot be hidden in all evidence of its greatness. 

Referencing an Ottawa Citizen article dated Saturday Feb. 1st, 2003. We read of
an effort to restore the Canadair North Star to a version of her "former self"
as PNSAC founder Robert Holmgren put it. A sad, sorry state at the time.
Destitute to serve as a bird house to the starlings. No mention of any spells
enacted to summon the aviation gods or mojo hand placed upon the yoke to stir
the spirits which have tilled the air.  Yet today, she's a Hangar Queen,
patiently awaiting final installation of engine \#4. Is this sorcery? No,
simply the fourth iteration of the Project's labour of love; Rolls-Royce Merlin
622, Serial \#309573. The task of accomplishing the finale to a four part, long
haul repeat restoration of a power plant began in 2014. Looking back at power
plant restoration cycles, we see the evolution of the Project North Star
volunteer organization since the early 2000's. The forming of Engine Shop crews
and the establishing of Crew Chiefs. The coordination of daily grunt work of
condition recording, treatment and reassembly of components. Today, members of
the project continue to develop ideas, adapting to new technology, finding
different approaches and maintaining a solid work ethic. This spirit remains
unchanged throughout difficult times and challenges we have encountered. 

The Tech. log states: Engine installed on North Star 17515 on August 23rd, 1965
at airframe Total Time of 19845.45 Hrs. with 1121 hours remaining in \#4
position (4th to the right). The engine had previously flown on 17514 until
April 30th, 1965 when a broken intake valve \@ \#1 cylinder, "B" bank
necessitated a repair. Flipping through pages, various signatures, authorities,
stamps and markings show evidence of a storied service life going back to 1948.
Engines are shared among power plant "skeletons" which in turn, are shared
across the fleet. This was one of the major design features of the "Power Egg"
as the promotional material of the day had put it, interchangeability and ease
of maintenance. A glance at the log tells me this engine serial number served
on North Stars 506, 510, 508 and 502 to name a few. The last remaining example
of its type, North Star 17515 holds pieces of the past, belonging to aircraft
that have long been gone. Like crew members who have also shared their time
amongst a fleet that traveled all over the globe, with change-overs,
turn-arounds and plenty of distance in-between.  

One of the big pay-offs for members of Project North Star is the opportunity to
display the work carried out by volunteers. Many occasions come to mind, such
as Canada Day, Doors Open Ottawa and family visits to the Museum. One
opportunity, of which I will always be grateful to have participated in, was
such a moment. Myself and a few volunteers met a retired employee of the
aforementioned Bristol Aero Industries Ltd. who worked at the overhaul shops
when this engine was undergoing acceptance runs in the early to mid-sixties.
Quickly, job experience and familiarity with the object was brought forward. A
quick look under the cowling to reveal preserved markings and decals, evidence
of original repair shop work was pointed out. "I probably worked on that job"
he said. I understood at this point, how someone's work, preserved in such a
way, could be of intangible value. The history coming around full-circle before
my eyes, as the gentleman pointed out the ROLLS-ROYCE cast upon the rocker
covers, saying "it was one of my first jobs, painting on those red letters". 

These records, object qualities, historic log entries and our treatment reports
all add up. We string together these bits of information to form longer chains
that communicate the history of an artifact. As focal points shift towards
other areas of interest, or priorities, work carried out in the past helps
maintain a continued relevance for others to pick-up where we left off. With
such a long term project, setting the pace is just as important as taking a
break now and then. Knowing that we had an impact and remembering the
experiences we've shared can carry us through new challenges. It's a pleasure
for me to observe a new volunteer work towards acquiring the skills that may be
passed on to others, like an old wizard speaking to his young apprentice. 

I encourage Project North Star members to communicate their interest in the
project to the public. Generate their own records that may contribute to the
history of the machine and the lives it affected. This newsletter, social media
and other forms of reaching out to the public are excellent ways to remember
one of Canada's great contributions to air transport. With the tools at our
disposition, I would not hesitate to believe we can create many new and
engaging experiences for the public to learn about the role this aircraft
undertook while in service. Now, in continuation with a sense of national
pride, the North Star remains an important platform for interpretation of the
work carried out by dedicated volunteers supported by a museum that understands
the value of such an artifact. 

\begin{footnotesize}
  \raggedleft PNSAC\\
\end{footnotesize}

% End of text.

%%% Local Variables: 
%%% mode: latex
%%% TeX-master: main_document.tex
%%% End: 

