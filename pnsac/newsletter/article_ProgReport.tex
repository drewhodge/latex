% Template PNSAC newsletter - Article
% Language: Latex
%

% Head

\title{Conservator's Corner}
%\subtitle{PNS 2016 Status Update}
\author{R\'{e}jean Demers}
\author{Restoration Project Manager}

\maketitle

\section{Zen and the Art of Aircraft Restoration.}

Through my daily experience over the past two years, working with volunteers, I
have asked myself the following question: What makes a volunteer dedicated to a
project? The answer is not a straightforward one. This article explores the
subject, much like Pirsig did in his book about metaphysical motorcycles; Zen
and the Art of Motorcycle Maintenance.

We strive for quality. As restoration work continues on the North Star, we get
absorbed into the weight and breadth of the subject matter. This is the kind of
project that demands the level of commitment found in monasteries, dojos and
Paratrooper units. Commitment comes from dedication and reward. Quality is
obtained through application of commitment, producing results which may be
enjoyed. The following effort in writing has been made to define such elements,
those necessary to sustain the project.

A voracious appetite for detail. Precise facts and data build foundations.
Foundations of information that erode over time. Lost to floods, fires, or
worst; thieves, vandals and neglect. Information is our firmament that keeps us
grounded in our work. Daily worksheets are filled with steps taken to satisfy a
condition. Drawings are made to translate reality to paper. These words,
drawings and tables form hard to translate time lines. Coloured bars overtaking
each other, running in thousand hour blocks of volunteer contributions. These
life hours, cannot be overlooked, as a restoration project advances through the
years.

Inorganic compounds risk distraction. Focus is important when avoiding the
ethereal draw of solvents. A volunteer scrubber, washing away years of residue,
layered upon a finish that once gleamed with pride. Slowing the return to dust,
so feared and fought by the living. A process to follow and adhere to. The
methodical dismantling and reconstruction of an object. Return to form, that
which was almost indiscernible. The result of all his work is not merely a
shiny piece of aluminium panelling; He has travelled through time.

More than just an object, we look to a hidden past. Beyond these material
trappings, lay the history of Canadair North Star 17515 and her sisters.
Examples of service are easily found in books and photos. What proves a
challenge is explaining how a bumblebee becomes a preserved specimen in an
engine, how graffiti made during heavy maintenance is the only evidence of a
man's past. Where the dirt under floorboards, came from countries that no
longer bear the name she visited.

The static, prevents complete appreciation. She's more than 45,000 lbs. of
metal. Superficial understanding can only consider her 39 years an outdoor
relic. Likewise, so would a skin deep treatment of a last remaining artefact of
this type. What impact she had before the stillness of the seasons is not lost
to those who look further.

A biological experience in personal evolution. The volunteer has developed new
muscles and sinew adapted to producing polished surfaces. An eye for
microtexture and grain direction has developed. Sympathetic scars are
accumulated through careful little sacrifices. Navigating the disruptions of
capricious human needs such as rest, nourishment and leisure. Through the
application of labour, a piece of history is preserved.

Disappearing horizons. Meditative expanses of aluminum, exhibiting varieties of
corrosion rarely encountered in a lifetime's career. Cavernous interior spaces,
where cargo of all sorts was bound for ports beyond. Flight control surfaces
that have pressed through all shades of dusk, emerging in the dawn of a new
day. Now removed from the ship, these weathered, beaten, bruised remnants of a
glorious past await restoration. A new horizon forms, some many years away,
known to those who are on-board for the long run.

Healthy obsessions. Volunteer restoration technicians take home an experience.
They take to heart the results of their labour, in forming a product that is
also intangible. The mind has strengthened, forming resolves that assist,
understand and process the elements of restoration work. As historians,
fanatics and unwitting library wanderers soon discover; The story of air
transport, in its beginning, is far from just a few obscure paragraphs in
Canada's ongoing history. It lives through those people who continue writing
it, preserving the physical, extending beyond.


\begin{footnotesize}
  \raggedleft PNSAC\\
\end{footnotesize}

% End of text.

%%% Local Variables: 
%%% mode: latex
%%% TeX-master: main_document.tex
%%% End: 

