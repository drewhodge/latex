% Template PNSAC newsletter - Article
% Language: Latex
%

% Head

\title{Conservator's Corner}
%\subtitle{PNS 2016 Status Update}
\author{R\'{e}jean Demers, AME---Conservator \/ Special Project Manager
Canada Aviation \& Space Museum}
%\author{Restoration Project Manager}

\maketitle

%\section{Zen and the Art of Aircraft Restoration.}

National Volunteer Week was held recently. It was a timely call to action. Let's focus our collective strengths and resources to prepare for Project North Star RESTART 2022. 

It has been roughly 20 years now, since the beginning of the North Star restoration project. Over the past two years, we've had to endure the effects of the CoVID-19 global pandemic. This spring, we have an opportunity to make an attempt to (gradually) return to normal operations. Ingenium is currently initiating actions that will permit the return of Project North Star volunteers. This will require a concerted effort on behalf of all stakeholders. 

My intent is to have crews receive training, gradually integrating in numbers, until we could finally support full daily compliments based on operational capacities. Now, with the demands imposed by Ingenium's recent acquisition approval of an RCAF CC-115 Buffalo, we must act upon this news in conjunction with a collection re-org. This signifies an earlier than expected requirement for Project North Star volunteer support, due to the increased scope of work. 

CASM Conservation is preparing a work schedule reliant upon this volunteer support, in order to mobilise the North Star from the Reserve Hangar.  Moving forward, through consultations with all parties, obtaining guidelines and limitations, as well as setting goals and expectations so that we may provide an integrated work plan for 2022-2023. 

High on our list of immediate concern is rendering 17515 mobile so that we may conduct a massive hangar shuffle. Collection objectives are driving CASM Conservation Services (and Project North Star) into a 24 month action plan which will eventually see us dispose of some artifacts and acquire one big yellow beast (115452). Thankfully, I have been able to secure accommodation within the storage hangar for the North Star between brief stints outdoors during fair weather movements, in amongst the storage hangar and the main museum. 

Project North Star has an important role in reinvigorating the museum ecosystem, providing a necessary boost of youthful vigour and effervescence. You may be asking yourself if during this prolonged absence, a time machine or temporal portal was made available for us to return to a simpler time, a golden age of aviation perhaps? Disappointingly, such a device has not yet surfaced beyond the top secret labs of the NRC. We must rely upon more tangible resources, such as students, cadets and young urban professionals. Diversification and integration of alternative human resources are being tapped, like springtime sugar maples.

It's been a long winter, or two. Attempts to grasp fleeting moments of human connection leave us questioning what sort of catharsis may be obtained from a grocery store checkout aisle. I myself, have taken steps towards awkward avenues of discussion by making cold-calls to random organizations which may prove to be valuable partners in the advancement of the project. Recognizing the North Star as an important platform for discussion and broadening the scope of Canadian heritage interpretation. 

With this recent streak of renewal, and a strong and healthy relationship, Project North Star and in turn, Ingenium may remain viable organizations as we continue our recovery. I look forward to hearing back from everyone, so that we may set a date for Project North Star RESTART 2022.


\begin{footnotesize}
  \raggedleft PNSAC\\
\end{footnotesize}

% End of text.

%%% Local Variables: 
%%% mode: latex
%%% TeX-master: main_document.tex
%%% End: 

