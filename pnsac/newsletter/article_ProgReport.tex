% Template PNSAC newsletter - Article
% Language: Latex
%

% Head

\title{Project Manager's Progress Report}
\subtitle{December 2013}
\author{Bruce Gemmill}

\maketitle

\section{Nr 3 Engine}
\label{sec:engines_3}

Engine 3 is nearing completion.  The supercharger and intercooler
preheat assemblies have been installed, along with the propeller
reduction gearbox.  While this work was being carried out by the
engine crew, the remainder of the volunteer work force busied
themselves completing the numerous cowl panels, pipes and hoses and
other pieces needed to complete the engine.  Two items that seemed to
take forever were the large steel exhaust shrouds.  A lot of hammering
was needed to pound out years of dents, then numerous rivets and all
the Dzus fasteners and springs needed to be replaced.  The complex
shape of the exhaust shroud made this challenging work.  Further
delaying completion was the search for suitable clear coat to protect
the steel from corrosion.  The old clearcoat we used successfully on
engines 1 and 2 is no longer available.  Mike Irvin finally located a
suitable replacement, and the shrouds were clear coated and serial
numbers were stencilled on.

\begin{figure}[htbp]
   \vspace{2em}
   \centering
   %name of the graphic, without the path AND in EPS format:
   \includegraphics[scale=0.5]{Eng3Intercooler.eps}
   %caption of the figure 
   \caption*{\small \em Garry installing the intercooler on engine 3.}
   %label of the figure, which has to correspond to \ref{}:
   \label{fig:Eng3Intercooler}
\end{figure}

The auxiliary gearbox was disassembled, and significant corrosion was
found on the bearings inside.  Bearings had to be ordered from
England.  Meanwhile, the gearbox was cleaned and painted.  At the same
time, the large DC generator was dismantled, repainted and
reassembled, as well as the air pump and the tachometer generator and
propeller synchronizer.  Once the new bearings were installed, the
gearbox was completed and the generator and air pump were attached.
Finally, the gearbox was installed on the firewall, which is now
finished, awaiting the installation of engine \#3.


\section{Crew Lounge, Galley,  and Forward Washroom}
\label{crewlounge}

Work has slowed on the forward section of the aircraft.  The repair
needed under the floor of the crew lounge is on hold until a suitable
piece of extruded aluminum can be found.  Until this is complete, the
new cushion flooring cannot be installed. This has also delayed the
installation of all of the crew lounge accessories, such as the table,
seats and secure storage bin, all of which are complete but waiting in
storage.

\section{Fuselage and Empennage}
\label{empenage}

The forward and rear belly compartments are now complete.  All the
frame pieces and hardware needed for the troop seats have been made,
and work is progressing on the seats and backs.  One section has
undergone a trial fit, and all pieces seem to fit perfectly.

\section{Planned Restoration Work 2013/14}
\label{sec:plannedwork}

The museum has advised all North Star volunteers that our work must be
put on hold for several months while a major exhibit for the 100th
anniversary of the beginning of the First World War is prepared.  Our
work space in the restoration hangar is needed to assemble and prepare
several aircraft that will be added to the display space in March.
Until then our work on the aircraft is suspended, but we don't expect
this delay to have a serious impact on the overall project.  We look
forward to resuming work in the Spring.


\begin{footnotesize}
  \raggedleft PNSAC\\
\end{footnotesize}

% End of text.

%%% Local Variables: 
%%% mode: latex
%%% TeX-master: main_document.tex
%%% End: 

