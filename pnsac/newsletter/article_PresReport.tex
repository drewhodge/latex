
% Template PNSAC newsletter - Article
% Language: Latex
%

% Head

\title{Notes from the President}
\author{Richard Lodge}

\maketitle

The last NStar Chronicle was published in March 2016 and since that time we have
continued our work on the aircraft, but at a greatly reduced pace. Last summer
the Museum staged a Star Trek event in the Reserve Hanger which affected our
ability to work on the North Star. We were not asked to suspend our work until
the end of September, unlike in 2014 when preparations for the Star Wars event
were being made. 

At the end of September 2016, Mike Irvin, the Museum's Conservation Special Projects
Manager, retired and we were asked to suspend work on the aircraft until a new
manager was appointed. In January, R\'{e}jean (Rej) Demers was appointed as the new
Special Projects Manager. Our work on the aircraft resumed on January 30th.

When Project North Star was originally formed in 2003, the Association was
treated with considerable scepticism by the Museum. It was felt that restoring
an aircraft of the size and complexity of a North Star, to the exacting
standards of a national museum, was likely to be beyond the capabilities of a
group of volunteers and that the project would probably fail.

We have now completed approximately 70,000 hours of volunteer work over a period
of 14 years, restoring everything we do to what is acknowledged to be a very
high standard. Scepticism is a thing of the past and we are now treated as a
respected and ongoing part of the Museum operations.

Respect must be earned and comes through hard work and in the case of aircraft
restoration work, attention to detail. Our team of volunteers needs guidance
both from the Museum staff and from our own senior volunteers. One volunteer has
been particularly important in helping us the get to the position we are now in.
Last year he reached a milestone only achieved on two other occasions by a
volunteer and in each case the volunteer worked at the Science and Technology
Museum. Our volunteer has worked 10,000 volunteer hours on the North Star and
for most of this time he also acted as our Project Manager for everything except
the engines. Bruce Gemmill probably was not aware that he had reached 10,000
hours when during June, his remarkable contribution was recognized by the Museum
and in a private event he was presented with a certificate and a gift by Cedric
St-Amour, the Museum Corporation Volunteer Co-ordinator.

I have always seen our Association as having multiple roles. First and foremost,
we are restoring a major artifact with our volunteers. Secondly, we show off our
restoration work to the public which in turn brings visitors and revenue into
the Museum and thirdly through our experienced volunteers we help transfer
skills to younger and less experienced volunteers.

Planning our work in conjunction with the Museum management and the Conservation
staff is something we must continuously work at. It is often a fine balancing
act to decide when to do work on the aircraft in relation to when the Museum is
having a major event such as Open Doors Ottawa and Canada Day. I am hoping that
in future we can regularly update our work schedule in conjunction with our
project plan so that we can more readily have the aircraft available for public
viewing at the times when the Museum would like us to do so, without having to
delay our restoration work unnecessarily.

During the last year, we were able to hold two of our successful Lecture Series
events and are planning to hold another one in May of this year. More details of
the next event will be available soon.

The Annual General Meeting of the Association was held on March 11, 2017 at the 
Museum. It was gratifying to see a good turnout of members, given the difficult
times our organization has gone through in the past few years. Both the Museum's 
Volunteer Coordinator, Cedric St. Amour, and the new Special Projects Manager 
R\'{e}jean Demers gave up their Saturday mornings to come and address our members. 
Their presentations were very well received and are further evidence of a new 
era in our relationship with the Museum.

I will conclude with an invitation to any of our members not volunteering at
present to come and join us. Although we have a limit on the number of
volunteers who can get their hands dirty working on restoration, our organization 
needs  volunteers to assist with other Association activities including merchandise
sales, special events, fund raising, communications and administrative work. Whether
you are working on the plane or doing other Association work you will meet likeminded
people and have the opportunity to speak to the general public. You do not need to 
be an expert on aviation or on the North Star to contribute. You can contact me by 
email at the address at the end of this newsletter if you would like to find out 
more about volunteering.

\begin{footnotesize}
    \raggedleft PNSAC\\
\end{footnotesize}

% End of text.

%%% Local Variables: 
%%% mode: latex
%%% TeX-master: main_document.tex
%%% End: 

