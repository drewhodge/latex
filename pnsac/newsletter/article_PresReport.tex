
% Template PNSAC newsletter - Article
% Language: Latex
%

% Head

\title{Notes from the President}
\author{Richard Lodge}

\maketitle

These are exciting times to be a member of the Project North Star Association.
We  have recently taken part in the 2018 Doors Open Ottawa. This is a major
event for the Museum and PNSAC volunteers were able to make a significant
contribution to the success of the two open days. Doors Open has become second
only to Canada Day in its importance to the Museum. (more on Open Doors 2018
later in this issue).

A few days before Canada Day the aircraft was brought outside. Unfortunately,
the day itself was extremely hot and the number of visitors was well down from
previous years. Nonetheless our volunteers did a yeoman service in talking to
the public about our restoration work.

Although showing the aircraft to the public is very important to us, our main
purpose is to restore the North Star to museum condition. Much of our work is
detailed, painstaking and unspectacular such as restoring a small part of the
aircraft which can take many days or weeks. At other times making progress is
very visible. We are in such a situation now.

At the end of May we had one of our regular meetings with Chris Kitzan (Director
General of CASM) together with Neil Raynor (VP of PNSAC), Cedric St-Amour (CASM
Volunteer Coordinator) and Reg Demers (Our CASM supervisor There was a full
exchange of ideas, suggestions for bringing  the public into the Museum and ways
in which PNSAC could be involved in these initiatives and events. The meeting
ended with me having a full sheet of notes covering future planning, major
restoration work and how we can make sure we recruit younger people both as
members of PNSAC and where possible active Restoration Volunteers. 

A meeting like the one above only happens when an organization is thriving, and
everybody is excited about the future. During the last 15 years PNSAC has
demonstrated that it is capable delivering very high-quality restoration work.
We have already accumulated over 80,000 hours of work and during this time we
have become a trusted partner of the Museum. Our website is now operational
again and in the next few weeks we expect to be able to update it and make it
more relevant to where we are with the project now. One of the major initiatives
taking place is to develop a diagrammatic history of the work done on the North
Star restoration and a timeline for future work on the aircraft. 

In closing, I once again urge Association members to become active volunteers.
There are many volunteering opportunities both for those with major aviation
experience to those who can only volunteer with us on occasions like Canada Day
or can provide help for our many behind-the-scenes activities.

Our next Members' Meeting is scheduled for
Saturday, September 29, 2018. Do reserve this date in your calendar if you can
be in the Ottawa area on that day. The meeting will be held at the Canada
Aviation and Space Museum. More details will follow later.

\begin{footnotesize}
    \raggedleft PNSAC\\
\end{footnotesize}

% End of text.

%%% Local Variables: 
%%% mode: latex
%%% TeX-master: main_document.tex
%%% End: 

