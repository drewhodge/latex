
% Template PNSAC newsletter - Article
% Language: Latex
%

% Head

\title{Notes from the President}
\author{Richard Lodge}

\maketitle

As I write these words I am watching the falling snow and preparing for a nasty weather mix this evening and at the same time thinking of the spring and summer and our outdoor activities with the North Star aircraft. 2016 has every chance of being an interesting year for the North Star volunteers. Although full details are not available yet, we are expecting to be able to show off the plane to the general public and to other groups several times this year. 

We are continuing to work closely with the Museum conservation staff to develop plans for extended hours working which basically means running a restoration shift on one or more Saturdays each month. This should enable some members to become Restoration Volunteers even though they are not available for work Mondays to Fridays. We had hoped to start the shift before Easter but it looks now as if the start date will be in the fall. There are still some details of supervision, security and safety to be worked out. Running a Saturday shift is a new endeavour for the Museum and it is important for everybody that we take the necessary time to get the operation properly setup so that we have a successful start and a good model for any further out of hours working arrangements for PNSAC or any other volunteer organization working on a museum corporation artifact. 

We were very pleased to welcome Chris Kitzan, the new Director General of the Canada Aviation and Space Museum to our first Members' Meeting of 2016 which was held in the Bush Theatre on February 27th. Chris acknowledged the importance of the partnership between the museum and Project North Star. Following the meeting, I saw Chris having a lively and upbeat conversation with our members, over coffee and snacks. 

Prior to the Members' Meeting, several of us ran into Mike Charters, who is a volunteer maintenance mechanic at the Warplane Heritage Museum in Hamilton. Mike was one of the team who flew to England in the summer of 2014 for the tour of the UK by the Canadian Lancaster together with the British Lancaster. Mike generously agreed to speak to the members for a few minutes and had some very interesting stories about the trip.
 
One of the best activities for our members is to become an active volunteer. Our volunteers fall into two categories, Restoration Volunteers and Association Volunteers. The former are those who work on the North Star restoration and have to be limited in number because it is not possible to have too many people working at any one time; this is mainly for supervision and safety reasons. Association Volunteers are those who volunteer for other activities in our Association. There are many different things which need to be done in a volunteer organization and we encourage our members to contact us to volunteer for such things as meeting the public on aircraft open days, helping to organize events and assisting with sales of our merchandise. Volunteering has many rewards, from comradeship with other volunteers, enjoying involvement with an interesting aviation artifact and personal satisfaction derived from doing something worthwhile. If you are interested in exploring volunteering opportunities, do contact any one of our directors whose contact information is shown at the end of this newsletter. 


\begin{footnotesize}
    \raggedleft PNSAC\\
\end{footnotesize}

% End of text.

%%% Local Variables: 
%%% mode: latex
%%% TeX-master: main_document.tex
%%% End: 

