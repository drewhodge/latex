
% Template PNSAC newsletter - Article
% Language: Latex
%

% Head

\title{Notes from the President}
\author{Richard Lodge}

\maketitle

Since my last Notes from the President in December 2013, restoration
work on the North Star has mostly been suspended. The Museum has
needed extra working space so that exhibits could be made ready for
the commemoration this year of the centenary of the start of the First
World War and the 70th anniversary on June 6 of D-Day in the Second
World War. These preparations have resulted in our normal working
areas being unavailable for North Star restoration work.

\begin{figure}[htbp]
  \vspace{2em}
  \centering
  %name of the graphic, without the path AND in EPS format:
  \includegraphics[scale=0.5]{richard_mtg.eps}
  %caption of the figure 
  \caption*{\small \em Richard at the Quarterly Meeting in April.}
  %label of the figure, which has to correspond to \ref{}:
  \label{fig:richard_mtg}
\end{figure}

During the shut down period some limited work has been continuing. Two
volunteers have been working in the cockpit of the plane and two other
volunteers have been doing some work on \#3 engine reassembly.

While the restoration work has been on hold, we have arranged three
monthly Volunteer Lunches at the Museum to enable the people who
normally work on the plane to meet and socialize. Although we would
all like to be working on the plane, we understand why the Museum has
had to ask us to temporarily suspend our activities. The Volunteer
Lunches are a way of keeping the team together and up-to-date on what
is happening. 

At the last meeting of the PNSAC Board of Directors we agreed to
reschedule the Annual General Meeting from June each year to
September. It has always been difficult to find a good weekend in June
to hold the AGM and it is felt that a middle of September date would
be much better. Our corporate secretary, Roger Button, is aware of
this proposed change and has confirmed that the existing Board of
Directors can extend its term until the new AGM date.

Although there has been little restoration work done since December,
we are working with the Museum to get the project back on track again
in early June. We are also working to enable us to recruit new
volunteers to replace the several people who for various reasons have
had to give up active work.

One of the recent discussion areas has been in connection with
long-term funding of the restoration project. Everybody is aware that
funding from the Federal Government has been substantially reduced. We
have accepted that major  expenditures will have to be met from funds
raised by donations from friends of the Museum and members of our
Association. This is a significant challenge since everybody receives
many appeals throughout the year for donations from many different
deserving organizations. 

During the next few months we hope to develop a revised restoration
work plan which will enable us to estimate the funds needed to
complete the project. We will then be able to target appeals for funds
to carry out specific parts of the restoration in the same way as we
did for our successful appeal for funds to make troop seats for the
interior of the plane.

\begin{footnotesize}
    \raggedleft PNSAC\\
\end{footnotesize}

% End of text.

%%% Local Variables: 
%%% mode: latex
%%% TeX-master: main_document.tex
%%% End: 

