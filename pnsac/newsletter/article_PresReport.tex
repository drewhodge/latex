
% Template PNSAC newsletter - Article
% Language: Latex
%

% Head

\title{Notes from the President}
\author{Richard Lodge}

\maketitle

As we come to the end of 2013, I think of it as an interesting year
with some disappointments and important successes. I believe we end
the year in a very positive way and I look forward to 2014 with
enthusiasm and renewed optimism.

First let me get the disappointments out of the way. 2013 was our 10th
anniversary year. We had hoped to make the event an opportunity to
show off our achievements during the first 10 years and create
additional interest around the plane. Unfortunately, due to events
beyond our control, it was not possible to stage a significant event
marking the anniversary. We therefore did our usual opening of the
plane to the public on Canada Day and marked the anniversary with our
PNSAC photographer taking photographs of members as they visited the
restored cockpit.

The second disappointment is more serious. During 2013 we did not take
on any new volunteers to work on the plane. We had several good
applicants but were unable to fit any of them into the working
schedule of volunteers at the Canada Aviation \& Space Museum
(CASM). This has two downsides; the first being that we are now
becoming short of active volunteers and the second being that we are
not training any new and sometimes younger volunteers to work on
pre-jet age aircraft. It is important to pass on the skills of the
older volunteers to younger members of our Association.

Funding the restoration of the aircraft is always a challenge and is
becoming more so as the Federal government cuts back on the funds it
appropriates to the CASM and other similar institutions..

You may well ask why I feel optimistic about 2014 after referring to
disappointments and problems. In the last two months the Directors of
PNSAC met with the Director General of CASM, Stephen Quick, and
subsequently a small group of directors met again with him together
with CASM conservation staff. The outcome of these meetings has been
an excellent exchange of views, a very firm commitment by the CASM to
continue to actively support the North Star restoration and a better
understanding by both sides of the challenges we face and possible
ways forward.

As many of you already know, volunteer work on the North Star will be
suspended for approximately the first three months of 2014. This is to
enable CASM to undertake a major reorganization of the Museum exhibits
in preparation for the centenary of the start of the First World
War. During this time the CASM professional staff will be occupied
with the reorganization and will not be able to provide the necessary
supervision of restoration work. During this time, though, we plan to
update the work plan and overall budget for completing the North Star
restoration, work on fundraising initiatives and finally update our
Memorandum of Understanding with CASM.

In addition to our hands-on restoration work, the Association has a
strong social side. Bill Tate, our Vice President, and his Special
Events team have organized some memorable bus trips, the last of which
was to Montreal in October. Bill has now turned his considerable
organizational skills to further developing our quarterly Members'
Meetings. In early December we had a great morning at CASM when
Capt. Bob Pearson of the Gimli Glider fame made a presentation to 150
of our members and members of other aviation organizations we invited
to attend. Although our Vice President has retired as an airline
captain, he certainly has not retired from anything else. Watch your
emails, Facebook and Twitter for news of events Bill and his team are
planning for 2014.

\begin{footnotesize}
    \raggedleft PNSAC\\
\end{footnotesize}

% End of text.

%%% Local Variables: 
%%% mode: latex
%%% TeX-master: main_document.tex
%%% End: 

