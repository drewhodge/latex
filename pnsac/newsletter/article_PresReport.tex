% Template PNSAC newsletter - Article
% Language: Latex
%

% Head

\title{Notes from the President}
\author{Richard Lodge}

\maketitle

During the past three months we have moved steadily forward with the
restoration of the North Star.  Our progress has been somewhat held up
by lack of available volunteers.

Our Association has two distinct and very important sides.  First and
foremost we are a group whose interest is in the North Star and the
restoration and preservation of the plane at the Museum.  Secondly,
over the years, we have developed a strong social side to the
Association where volunteers and other members can join in interesting
events centered around aviation. Elsewhere in this issue of the
Chronicle you will find a description by Bill Tate our Vice President,
of the very successful trip on October 4 to the Navcan Air Control
Centre and to the Bombardier factory in Montr\'{e}al.

After 10 years of restoration work on the North Star we recognize that
we must review our progress and develop a plan for the rest of the
work on the plane.  There will be three aspects to this review.  Our
first task will be to identify the major sections of the project and
develop ways of actually carrying out the restoration.  The
undercarriage of the plane is a particularly complicated challenge.
We then need to prepare a plan with projected timelines for completing
each part of the work.  Our third task will be to estimate costs and
to develop ideas for financing it.  During this process we must also
ensure that we will have sufficient qualified volunteers to get it
done.

Before the end of November we are hoping to have meetings with the
senior management of the Museum to update planning of all the tasks
above.  The end result of this should be a documented project plan and
a new Memorandum of Understanding between the Canada Aviation and
Space Museum and the Project North Star Association of Canada.

At times our progress on the restoration of the North Star may appear
to Association members to be somewhat slow.  Unlike other
organizations in Canada such as Vintage Wings in Gatineau or the
Canada Warplane Heritage Museum in Hamilton, PNSAC cannot make
decisions about the future of the North Star on its own.  As an
Association working with a Federal Government institution we are often
affected by decisions not related to the work we do.  This can at
times provide interesting challenges.

In conclusion, I would say that the Directors are working hard with
the Museum management to move our project forward even though at times
it may appear to be very difficult.  I hope that I will be able to
report significant progress in the next issue of the Chronicle.  In
the meantime we have some interesting events in the planning stages on
the social side of the Association.  Stay tuned.


\begin{footnotesize}
    \raggedleft PNSAC\\
\end{footnotesize}

% End of text.

%%% Local Variables: 
%%% mode: latex
%%% TeX-master: main_document.tex
%%% End: 

