
% Template PNSAC newsletter - Article
% Language: Latex
%

% Head

\title{Notes from the President}
\author{Richard Lodge}

\maketitle

Once again, we are experiencing summer with Covid restrictions in place and
suspended North Star volunteer operations.  As I write this piece in the middle
of July, we are seeing hopeful signs.  The Museum has reopened with reduced
numbers of visitors and we are expecting to arrange an in person coffee morning
in the middle of August for North Star volunteers outside or under the very
large door of the Restoration hangar if the weather is bad.

The Museum is reopening cautiously, and it will be at least September before the
North Star restoration can recommence.  Assuming that the Ontario Covid
infections remain low and the Ontario government rules allow it, we would also
like to have our AGM indoors in the fall.

This issue of the Chronicle is mainly about the completion of restoration of
engines 3 and 4. Some of our volunteers, however, have been busy during the
spring and early summer locating and acquiring a used North Star radar.  This
has now been delivered to the Museum.  In the next issue of the Chronicle, we
hope to have the story of how John Makadi and Chris McGuffin negotiated the
purchase of the radar and arranged for its transportation to Ottawa from
Springfield, Missouri.  Quite a story.

The crate containing the radar has not yet been opened.  We are expecting to do
this at the same time as we gather for our in person coffee morning in August.
We will be donating the radar to the Museum since it will be installed in the
restored North Star.

The used radar and its transportation costs have been financed in a very
pleasant but unexpected way.  Earlier in the year, our Membership Secretary,
Bruce Gemmill, at the request of the Board of Directors sent a message to all
our members saying that for the year 2020/2021 the annual membership dues of
\$25 would be cancelled and anyone who was a paid-up member in 2020 would
continue their membership for another year.  The Board decided to do this
because our members have received very little during the pandemic in return for
their support and membership.  

Members will have been surprised to receive the letter from Bruce, which as
always included a form for making donations.  Bruce decided that he would use a
large number of unused small denomination postage stamps he had collected over
many years.  As a result, all the letters had a rainbow of stamps affixed to
them.  Bruce will have had a very dry mouth when he finished attaching all the
stamps. A really nice idea by Bruce.

Although Bruce's letter did not specifically ask for donations, many of our
members decided to make donations to the project.  As a result we received over
\$2000 in donations we were not expecting.  The receipt of these donations
coincided with the decision to purchase the used radar from Missouri.  Although
the final cost of the purchase and transportation of the radar has not been
determined at the time of writing these notes, we are sure that the donations
received as a result of Bruce's letter will probably cover the full cost of the
acquisition.

Throughout this pandemic, Cedric St Amour, our Volunteer Coordinator, has every
Thursday morning hosted a Zoom virtual coffee morning.  This is normally
attended by six or seven people and remains open to all volunteers and members.
Anybody who has not been onto this meeting and would like to do so can contact
me at \email{president@projectnorthstar.ca}. I will then send the necessary
login information to you.

As we gradually get ready for recommencing volunteer work on the Northstar, we
will be talking with R\'{e}j Demers, CASM Special Project Manager, to organise a
workplan for the next few months, particularly in view of the fact that
restoration work on all four engines has mainly been completed.

\begin{footnotesize}
    \raggedleft PNSAC\\
\end{footnotesize}

\end{multicols}

\begin{multicols}{2}

% End of text.

%%% Local Variables: 
%%% mode: latex
%%% TeX-master: main_document.tex
%%% End: 

