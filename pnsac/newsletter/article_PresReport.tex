
% Template PNSAC newsletter - Article
% Language: Latex
%

% Head

\title{Notes from the President}
\author{Richard Lodge}

\maketitle

Since the last issue of the NStar Chronicle we have taken part in the
70th Anniversary celebrations of the first non-stop transcontinental
flight across Canada which was flown by a North Star. We partnered with
the Museum for this event and were very pleased to welcome
representatives from the RCAF who joined us for the day. After the
formal part of the celebration, when impressive plaques of the aircraft
made by our member Chris McGuffin, had been presented to the RCAF and
the Museum Director General, Chris Kitzan, everybody moved to the
Reserve Hangar and the aircraft was opened to our association members,
guests and in the afternoon to 90 members of the public who had
pre-booked a tour. R\'{e}jean(R\'{e}j)Demers, the Museum Special Project
Manager, guided the tours assisted by several of our members. Towards
the end of the tour session someone in the main Museum activated the
fire alarm. The Museum was promptly evacuated, and the public was
recommended to go to the Reserve  Hangar until the fire issue had been
dealt with (it turned out to be a false alarm). Quite unexpectedly R\'{e}j
and the guides found themselves surrounded by many Museum visitors who
had not signed up for the North Star tour. Unperturbed by this event,
Réj and the guides welcomed the unscheduled visitors and gave them an
impromptu tour of the aircraft. Both our treasurer and our membership
secretary were pleased with the days' result!

As I write this the days are getting longer and R\'{e}j is planning work to
be done on the aircraft when it is moved outside onto the ramp in the
spring. Each year we do work on the aircraft which cannot be done
inside the Reserve Hangar during the winter. We also prepare the
aircraft for display when, once again, we open the aircraft to enable
the public to view progress made in the last year.

This year we are planning two major display events. The first will be
on the Doors Open Ottawa weekend (June 6 and 7). This is a great time
for our members to meet the public, many of whom have an interest in
aviation. Our second main event, on Canada Day, is enjoyable for all
the members who sign up to be part of the team meeting the public.
There are always large crowds, many of them family groups, If any of
you reading this article would be interested in joining our members for
either of these events please contact me (my co-ordinates are on the
back page of this issue of the Chronicle)

Shortly before Christmas a major milestone, in the  restoration of the
North Star, was reached. Work on Merlin engine \#4 was completed and it
was dragged on its cradle by the Museum forklift truck to the
Restoration hangar for later mounting on the aircraft to join the other
three restored engines. Chris McGuffin put some video clips onto our
Facebook page of the engine being dragged through the Museum parking
lot and into the hangar. Watch for more of our Facebook messages since
Chris frequently posts new items.

In the late fall we welcomed the appointment of Erin Secord as the
Museum's new Manager Conservation Services. She is responsible also for
conservation of the collections at the Science and Tech Museum. Erin
has now been able to join us at two of our Members' meetings.

Finally, we are now working to update the Memorandum of Understanding
between PNSAC and the Museum.  Much has changed since the original
agreement was signed in May 2006.

Our next Members' Meeting is scheduled for Saturday, June 20, 2020
which will also be our AGM. Do reserve this date in your calendar if
you can be in the Ottawa area on that day. The meeting will be held at
the Canada Aviation and Space Museum. More details will follow later.

\begin{footnotesize}
    \raggedleft PNSAC\\
\end{footnotesize}

% End of text.

%%% Local Variables: 
%%% mode: latex
%%% TeX-master: main_document.tex
%%% End: 

