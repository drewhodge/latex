
% Template PNSAC newsletter - Article
% Language: Latex
%

% Head

\title{Notes from the President}
\author{Richard Lodge}

\maketitle

As always, before I start to write the Notes from the President, I reread my
previous Notes to remind myself of what I had previously said in order to set
the record straight if necessary. As I write this at the beginning of
September, I am now aware of the impact of the COVID-19 pandemic on the
restoration schedule for the North Star and the operations of your
Association.

Since March, restoration work on the aircraft has been suspended and the
activities of the Association have been limited. We have, however, been keeping
in touch with each other through the weekly Zoom meeting organized by the
Museum's volunteer co-coordinator C\'edric St.-Amour. This virtual meeting
takes
place every Thursday at 10 AM EST; it is open to all our volunteers and any
other PNSAC members who would like to join in.

As previously reported, the North Star project reached a major milestone with
the completion of the restoration of Engine \#4 which was moved into the
reserve hangar on a cold December 2019 morning.

Since we are unable to physically meet and there has been no restoration work
carried out, we felt that a special issue of the NSTAR Chronicle devoted
entirely to the 15 year restoration of all four of the Merlin engines would be
of interest to members.

Understandably, with a project which has taken many years. the volunteers
working on the engines have changed over time. Some have retired or stopped
volunteering and sadly, some have died. Earlier this year we asked Ted Devey to
contribute a piece but before writing about of his work, he passed away. Ted
was involved with restoring all the engines and for several of the early years
was the volunteer leading the restoration team. Ted was a great raconteur and
our oldest volunteer at 93 years old. His recollections would have added
considerably to the interest of this special issue of the NSTAR
Chronicle.

After much consideration, the Museum has decided that volunteering cannot
restart before January 2021. Until then we will continue with our regular Zoom
meetings and will also organize a virtual Annual General Meeting for the
Association. We will also continue to provide updates on the situation and much
will depend upon how the different levels of government handle the ongoing
recovery from the pandemic.

There is an old saying "That it is an ill wind that blows nobody any good."
This is very true even of this pandemic. One of the very pleasant results is
having conversations with PNSAC members who don' t reside in the National
Capital area and have sent membership renewals by mail. Where a renewal has
been received and confirmation has been delayed as a result of the pandemic, I
try to call a member on the telephone to explain what is happened and that the
renewal has not been lost. As a result I have had many interesting phone calls
with people whom I have never spoken to before. A definite bonus for me.

\begin{footnotesize}
    \raggedleft PNSAC\\
\end{footnotesize}

\end{multicols}


\begin{figure}[htbp]
   \vspace{2em}
   \centering
   %name of the graphic, without the path AND in EPS format:
   \includegraphics[scale=0.78]{north_star_welcome_1.png}
   %caption of the figure 
%   \caption*{\small \em  Mike Hope and Bruce Gemmill assembled Prop 1ready to
%install on Engine 1 October 26, 2008.}
   %label of the figure, which has to correspond to \ref{}:
   \label{fig:tim}
\end{figure}

\begin{multicols}{2}

% End of text.

%%% Local Variables: 
%%% mode: latex
%%% TeX-master: main_document.tex
%%% End: 

