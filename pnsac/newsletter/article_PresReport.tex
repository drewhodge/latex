
% Template PNSAC newsletter - Article
% Language: Latex
%

% Head

\title{Notes from the President}
\author{Richard Lodge}

\maketitle

The last North Star Chronicle was published in March 2016 and since that time we
have successfully continued our work on the aircraft while the Museum was making
preparations for the Star Trek event in the Reserve Hangar. We were not asked to
suspend our work, unlike in 2014 when preparations for the Star Wars event were
being made. This has helped us to retain our Restoration volunteers and recruit
two new ones. However, we are still short of Restoration volunteers and continue
to actively work with the Museum to address this problem.

When Project North Star was originally formed in 2004, the Association was
treated with considerable skepticism by the Museum. It was felt that restoring
an aircraft of the size and complexity of a North Star, to the exacting
standards of a national museum, was likely to be beyond the capabilities of a
group of volunteers and that the project would probably fail.

We will soon have completed 70,000 hours of volunteer work over a period of 14
years, restoring everything we do, to what is acknowledged to be a very high
standard. Skepticism is a thing of the past and we are now treated as a
respected and ongoing part of the Museum operations.

Respect has to be earned and comes through hard work and in the case of aircraft
restoration work, attention to detail. Our team of volunteers needs guidance
both from the Museum staff and from our own senior volunteers. One volunteer has
been particularly important in helping us the get to the position we are now in.
He has recently reached a milestone only reached on two other occasions by a
volunteer and in each case the volunteer worked in the the Science and
Technology Museum. Our volunteer has reached 10,000 volunteer hours of working
on the North Star and throughout this time he also acted as our Project Manager
for everything except the engines. Bruce Gemmill probably was not aware that he
had reached 10,000 hours when he unfortunately had to announce that he would
have to suspend his activities as a volunteer and director of PNSAC due to
family circumstances. We very much hope that he will soon be able to rejoin us
and continue to help us complete this large project. In the meantime Bruce will
continue to be our Membership Secretary. During June, Bruce's remarkable
contribution was recognized by the Museum and in a private event he was
presented with a certificate and a gift by Cedric St Amour, the Museum
Corporation Volunteer Coordinator. One of our volunteers, Robert Desjardins,
also joined Cedric on the occasion to congratulate Bruce on behalf of PNSAC.
Well done Bruce and we are all very proud of your achievement.

I have always seen our Association as having multiple roles. First and foremost
we are restoring a major artifact with our volunteers; secondly we show off our
restoration work to the public which in turn brings visitors and revenue into
the Museum and thirdly through our experienced volunteers we help to transfer
skills to younger and less experienced volunteers.

Planning our work in conjunction with the Museum management and the conservation
staff is something we must always try to jointly do better. It is often a fine
balancing act to decide when to do work on the aircraft in relation to when the
Museum is having a major event such as Canada Day. I am hoping that in future we
can regularly update our work schedule in conjunction with our project plan so
that we can more readily have the aircraft available for public viewing at the
times when the Museum would like us to do so, without having to delay our
restoration work unnecessarily.

I will conclude with an invitation to any of our members not volunteering at
present to come and join us. Although we have a definite limit on the number of
volunteers who can get their hands dirty working on restoration work, we always
welcome others who would like to participate in the running of the organization.
By becoming involved either as a restoration volunteer or an association
volunteer you will meet other like minded people and have an interesting time
talking to members of the public. You do not need to be an expert on aviation or
on the North Star to make a contribution. I can be reached by email or phone if
you would like to find out more about volunteering.



\begin{footnotesize}
    \raggedleft PNSAC\\
\end{footnotesize}

% End of text.

%%% Local Variables: 
%%% mode: latex
%%% TeX-master: main_document.tex
%%% End: 

