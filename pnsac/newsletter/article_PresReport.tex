
% Template PNSAC newsletter - Article
% Language: Latex
%

% Head

\title{Notes from the President}
\author{Chris McGuffin}

\maketitle

I showed up for the freebie, got an education and made some friends. That was my start in aviation. In 1985 I signed up for Air Cadets for something to do. One could say my mother made me. Certainly, she encouraged. The Cadets played sports and hosted camps; for free. I had no idea about the flight scholarships and other opportunities. Certainly, I was enamored with aviation. I built model airplanes and made a wind tunnel for a science fair. That wasn't much different from my sister's engagement with visual art or my brother's fascination with insects. My real passion was fixing things, improving them or making new things. My mum probably just wanted me to get out of the basement and socialize.

Aviation was exciting but I never considered earning a living from it. I studied electrical engineering at the Royal Military College. I thought a little military service would be a precursor to advancement in the civilian tech industry. Instead of Nortel I ended up serving 28 years with the Canadian Forces. Many of my friends wore the blue uniform. I wore army green or tan depending on the season. When I flew it was usually to participate in a meeting, attend training or visit a forward operating base. Still, there can be a persistence to childhood interest.

My interest in wood and metal working matured as a hobby. I have a particularly thoughtful wife who accommodated a larger work space with every move. Nonetheless, a residential workshop can be a solitary and confined space. When I retired from military service in 2017 it was time to look for different ways to pursue my interests. It would be fun to learn from new people and with tools I didn't have access to at home. I could not imagine a better opportunity than an aircraft restoration. 

My timing lined up with Project North Star, a comprehensive restoration effort: metal fabrication, engine repair, refinishing, reupholstery, overhauling mechanical systems, woodworking and more. Some volunteers contribute decades of skill and expertise while others offer enthusiasm and commitment to learn. Everyone benefits from cross training and camaraderie. Progress is important but not more than the historical value and preservation of the artefact. Each step is supervised by our project manager, an aircraft maintenance engineer employed by the Museum. 

Initially, I was assigned to reproduce and restore the wood paneling that lined the interior of our North Star. Occasionally I helped with other assignments. I watched as replacement parts were fabricated, engine \#4 was reassembled and bird nests were pulled out of the tail section. In 2018 I began publishing progress and event updates on the Project Facebook and Instagram pages. Many people enjoy seeing airplanes up close. Airshows are popular for good reasons. I certainly look forward to the time I spend at the aviation museum, particularly the areas normally hidden from public view. Taking a few photos to share with our social media audience was a natural fit for me. Those updates have been few during the pandemic but they will resume imminently.

2022 is going to be a year of renewal in many ways. In January I assumed the presidency of the Association. Our previous president, Richard Lodge, held the position for nearly 12 years. Richard has been involved with the Association for 20 years and I am thankful that he has agreed to stay on the Board of Directors. The Canada Aviation and Space Museum is actively renovating the conservation workshop. Plans for a return of volunteers to the Museum are being developed and restoration activity should resume in the months ahead. I am grateful for the encouragement of our supporters and the tenacity of our volunteers. We'll be back to the restoration effort soon. 


\begin{footnotesize}
    \raggedleft PNSAC\\
\end{footnotesize}

\end{multicols}

\begin{multicols}{2}

% End of text.

%%% Local Variables: 
%%% mode: latex
%%% TeX-master: main_document.tex
%%% End: 

