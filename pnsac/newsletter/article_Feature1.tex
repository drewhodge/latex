% Template PNSAC newsletter - Article
% Language: Latex
%

% Head

\title{My Days at the Home of the Merlin Engine}
\subtitle{Part 4}
\author{Richard Lodge}

\maketitle

One of the most mysterious departments at Rolls-Royce was the
Technical Publications Department. Word processing was unheard of in
the 1960s and every time an engine modification was made, it resulted
in much work documenting even a small change and distributing the
updated documentation to all engine users around the world. I never
spent any time working with the people in this department even though
there was a complete building devoted to the work. Compared with the
resources allocated to engine development, the technical publication
side of the business was very small and it always remained a mystery
to me.

Since retiring from my career as an accountant in Canada, I have
become active on the actual restoration work of the North
Star. Understandably, I have a special interest in working on the
Merlin engines. Nobody now working on the restoration has ever worked
on a Merlin when it was in service. It is therefore necessary at all
times to refer to the various publications and manuals we have managed
to accumulate for the North Star.

It is only now, actually working on the engine and referring
frequently to the exploded diagrams that I realize how complicated the
work of the Technical Publications Department must have been. The men
and women who worked there were largely unsung heroes in that it was
very tedious and unglamorous work which was of great importance for
the safe and trouble-free running of engines in service. I can
remember that I would hear from time to time of the difficulties of
ensuring that publication updates were actually delivered to many
remote places in the world and the considerable logistical
complications of ensuring that the packages arrived safely and
furthermore that the updates were actually inserted in the relevant
manuals, particularly as many of the engine users did not speak
English. I cannot remember now if any of the documentation was
translated into other languages although I have a recollection that we
employed people who could translate into French and Spanish.

After a couple of years I was given a new job in the Long Range
Planning Department. I was one of three accountants working in this
office under a very senior accountant. Here we looked at the costings
of the development work, the projected sales  and production costs on
new engine designs. This was the time when the Spey was going into
service and the the company had signed a fixed price contract for the
RB211 engines for the new Lockheed 1011. This was a major project,
championed by David Huddie the Chief Engineer. Computer spreadsheets
did not exist at that time - pencil and columnar paper were the order
of the day, together with efficient erasers. Long range forecasts of
25 years required many yearly columns and each time we produced
figures that the engineers did not like, we were told to rework the
numbers until we produced an acceptable answer. We were not limited in
the number of erasers we could use. The usual way of cutting costs was
to reduce the forecast development and testing time. Of course this
did not work in real life.

We were often dealing with the directors of R-R who lived on the top
floor of the Nightingale Road offices. Many of them were engineers and
they had little time for young bean counters. It was still a time of
optimism for the company. The huge contract for the Lockheed engines
had been signed and the company was still enjoying its reputation as
the company whose Merlin engines had saved Britain in the air
war. This was in 1965/6 and it was several years before our
projections of doom resulted in the bankruptcy of the
company. Fortunately I left the company in 1966 and was not involved
in what must have been a horrible process as a symbol of British
engineering excellence slowly went to the wall. 

After the chaos caused by the demise of Rolls-Royce Limited, the
British government formed a new nationalized company to take over the
remains of the old one. This has been very much a success story and
the newly privatized Rolls-Royce is now a highly profitable and
innovative company. I look back on my days at R-R with nostalgia but I
now realize that much of the time we were all living in a fool's
paradise.


\begin{footnotesize}
    \raggedleft PNSAC\\
\end{footnotesize}

% End of text.

%%% Local Variables: 
%%% mode: latex
%%% TeX-master: main_document.tex
%%% End: 

