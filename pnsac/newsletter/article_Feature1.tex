% Template PNSAC newsletter - Article
% Language: Latex
%

% Head

\title{Volunteers Working on North Star Restoration}
%\subtitle{Part 4}
\author{Tom MacGregor, Legion Magazine}

\maketitle

% \textit{Photos courtesy of Library Archives Canada and 426 Thunderbird
%  Sqn, RCAF.}\\

\begin{quotation}
	You have to be careful opening up the panels covering the 
	engines, you never know if a bird or some animal might have made
	its home there.
\end{quotation}

That’s one lesson a group of dedicated volunteers has learned in its
10 years of painstaking work restoring a Royal Canadian Air Force
North Star aircraft at the Canada Aviation and Space Museum in Ottawa.

%\begin{figure*}[htbp]
%	\vspace{2em}
%	\centering
%	%name of the graphic, without the path AND in EPS format:
%	\includegraphics[scale=0.5]{op-hawk-one.jpg}
%	%caption of the figure 
%	%\caption*{\small \em Volunteers put in countless hours restoring a
%	North Star aircraft.}
%	%label of the figure, which has to correspond to \ref{}:
%	\label{fig:op-hawk-one.eps}
%\end{figure*}

The aircraft had sat neglected outside the museum for
years. “Essentially, the RCAF flew this one into the Rockcliffe Airport
when it was decommissioned. It sat in a old until Robert Holmgren said
someone should do something about it,” said project manager Bruce
Gemmill.

The airplane had been stripped of its non-essential gear and turned
over to what was then the National Aeronautical Collection, housed in
hangars built during the Second World War at the former RCAF station
at Rockcliffe in Ottawa. Time and weather took its toll. Birds and
animals made nests where they could get inside the panelling.

“We approached the museum about restoring the aircraft. At first, they
didn’t know how to take us,” said Gemmill. “The museum had its own
professional conservation staff but it had never worked with a team of
volunteers.”

Eventually the museum agreed to work with the volunteers who formed
the Project North Star Association of Canada with the late Robert
Holmgren as president. “We became the model for other volunteer groups
now working with the museum,” said the association’s current
president, Richard Lodge.

Now the aircraft gets towed into a hangar where it is available to the
volunteers between September and May before it is towed outside for
the summer.

The North Star was built by Canadair in the late 1940s and 1950s. It
was the RCAF version of the DC-4 civilian aircraft except that it used
four Rolls-Royce Merlin engines. Canadair built 71 of the aircraft.

The RCAF assigned the rst North Stars to 412 Squadron where they
transported VIPs and were used in various transfigurations for
reliable, long-range transport services.

During the Korean War the North Star was used by No. 426 (Thunderbird)
Sqdn. to ferry sup- plies across the Pacific Ocean to Japan (Operation
Hawk: The Korean Airlift, July/August). They would y 599 round trips
over the Pacific and deliver seven million pounds of cargo and 13,000
personnel on return trips. All this was achieved without a fatal
crash.

Most of the air force’s North Stars were declared surplus in the
1970s.

In the 10 years that Project North Star has been in operation, the
group has restored two of the four Merlin engines with a third one
nearly completed.

Each engine had to be taken o the aircraft, brought into the museum’s
shops where they were taken apart. Each item was tagged, catalogued 
and usually photographed. “We take a lot of photographs. That’s
how we know how they go back together,” said Lodge.

Engine project leader Garry Dupont said, “The first engine took us about
four and a half years to restore. We’re almost finished the third engine
which as taken us about two years. So we are getting faster.” The
propellers were restored separately by Hope Aero Propeller and
Components Inc. in Mississauga, Ont. All restoration work has to be up
to the professional standards set by the museum.

The group plans to restore the aircraft to its original look, not to
have it y again.

Lodge said that they have some RCAF manuals but a lot of time they do
not help when it comes to putting 50-year-old parts together today.

The aircraft itself, 17515, has seen countless hours of work put into
re- storing the cockpit with all its switches and navigation
equipment and a small galley for the crew of seven.

Still to be done is work in the cargo section and painting the
exterior of the aircraft.

“We know this aircraft ew during the Korean War. In fact, we have 
pictures showing it configured with litters to bring the wounded back to
Canada,” said Gimmell. “We don’t want to put in the litters but we
have people working on creating the rope seats that were used.”

The plan is to paint the aircraft to its post-1965 markings, since it
was painted with the Maple Leaf ag on it.

Still the search goes on to nd new members for the association. “The
important thing is to nd new people to whom we can pass on this 
knowledge and skills we are developing,” said Lodge.












\begin{footnotesize}
    \raggedleft PNSAC\\
\end{footnotesize}

% End of text.

%%% Local Variables: 
%%% mode: latex
%%% TeX-master: main_document.tex
%%% End: 

