% Template PNSAC newsletter - Article
% Language: Latex
%

% Head

\title{RAF Lightning with Over-Wing Tanks}
%\subtitle{Part 4}
\author{Jim Riddoch}

\maketitle

During the early 60's Britain's Defence Department decided that the
new fleet of Lightning aircraft should have the capability to extend
its current sortie of 22 minutes to permit ferry to the Middle
East. In-flight refuelling had not worked successfully due to the
offset probe positioned under one wing. Accordingly BAC (British
Aircraft Corporation,Warton Division) was awarded a contract to
develop extended wing tanks that could be jettisoned if necessary.

\begin{figure}[htbp]
   \vspace{2em}
   \centering
   %name of the graphic, without the path AND in EPS format:
   \includegraphics[scale=0.5]{jim_riddoch.eps}
   %caption of the figure 
   \caption*{\small \em Jim Riddoch.}
   %label of the figure, which has to correspond to \ref{}:
   \label{fig:jim_riddoch.eps}
\end{figure}

%% \begin{footnotesize}
%% [Image: {\normalfont\color{blue}\href{http://www.airteamimages.com/english-electric-lightning_XP693_united-kingdom---royal-air-force_43865.html}{English
%%     Electric Lightning with over-wing tanks}}]
%% \end{footnotesize}

%\begin{figure}[htbp]
%   \vspace{2em}
%   \centering
%   %name of the graphic, without the path AND in EPS format:
%   \includegraphics[scale=0.5]{Eng3Intercooler.eps}
%   %caption of the figure 
%   \caption*{\small \em Garry installing the intercooler on engine 3.}
%   %label of the figure, which has to correspond to \ref{}:
%   \label{fig:Eng3Intercooler}
%\end{figure}

Unfortunately the Lightning with its high speed wings had limited
capacity for wing tip tanks and already with armament pylons under
wing could not accommodate under-wing tanks, so the decision was made
to develop over-wing tanks secured with pylons through the wing
structure. These tanks would each carry 250 imperial gallons to
provide initial fuel transfer after take off and be jettisoned once
empty.

I was designated to be the Development Engineer to carry out the
development and testing of this design about 1963/64. The design
concept of these tanks had to include a balanced transfer system to
maintain aircraft C of G during take off and climb out, this meant 5
small interior sealed compartments pressurized with air to permit
sequential transfer. Also an emergency fuel jettison system prior to
tank wing ejection. As you can readily appreciate this was a very
challenging design and even more demanding for testing and
certification.

Once the initial test tanks were built they were dispatched to
Mechanical Test Department at Warton Lancashire, where I worked.
Needless to say the RAF were keen to witness the development and
testing of these over-wing tanks. A Wing Commander headed a team of
three Senior Techs to keep track of our progress. Fortunately the
Techs were very helpful and hands-on in working with us.  You can
imagine the challenges we faced checking the transfer sequences at
varying attitudes. Many a Saturday morning was spent testing and
verifying these change in attitudes and minimizing fuel transfer loss,
the end result being approximately 225 gallons of fuel transferred
from each wing tank depending on aircraft attitude and roll.

The next major design requirement was the emergency fuel jettison;
this was to be accomplished by air pressure from centre compartment to
rear of the tank where a jettisonable fairing was fastened by
explosive bolts and attached by a cord to a bung on the end of the
fuel ejection pipe. Talk about Heath Robinson device, this takes the
biscuit! Suffice to report that on the ground test of this
ill-conceived system with Chief Designer and my Chief Engineer and
many observers with cameras, the rear fairing departed into oblivion
and the cord broke, due to a miserable sub standard cord substituted
instead of a parachute cord as specified. The bung remained firmly
attached to the jettison pipe. It was about this time I first thought
of emigrating to Canada, in fact anywhere outside Britain!

Needless to say the cord was replaced, the ejection system proved
satisfactory and the tanks were certified for flight test. This was to
be done with one tank attached to the top of one wing and painted with
black and white checkered squares for easy recognition when ejected. I
always remember the Test Pilot, Peter Knight inquiring about the
likely occurrence of the tank separating from the wing and hitting the
tail of the aircraft. I assured him that ground testing of a model in
the wind tunnel had worked, in fact the small tank had disappeared
through the end of the tunnel and took some time to recover in the
neighbouring field.

To replicate the fuel system jettison some bright spark proposed
filling the tank with coloured water for better visibility. The
aircraft took off, circled Warton Aerodrome and made a low pass to
jettison the coloured water over the attentive crowd including chiefs
and dignitaries. Fortunately, the jettison of the tail fairing and
bung release worked perfectly and the water emerged in a colourful
shower.  

The weather gods not being with us, the wind blew all the water over
the spectators covering their expensive clothing with a colourful dye!
Finally, the tank was ejected safely over Morecombe Bay and eventually
recovered and returned to BAC Warton amidst embarrassed and colourful
faces. As one might expect, this completely successful test left the
Wing Commander disgusted with the entire experiment. He cancelled the
project.

Any thoughts I had of joining the RAF completely disintegrated. Once
again, immigration to Canada looked much more inviting.

\begin{footnotesize}
    \raggedleft PNSAC\\
\end{footnotesize}

% End of text.

%%% Local Variables: 
%%% mode: latex
%%% TeX-master: main_document.tex
%%% End: 

