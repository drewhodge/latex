% Template PNSAC newsletter - Article
% Language: Latex
%

% Head

\title{Operation Hawk}
%\subtitle{Part 4}
\author{Hugh A. Halladay, Legion Magazine}

\textit{Photos courtesy of Library Archives Canada and 426 Thunderbird
  Sqn, RCAF.}

\maketitle

\begin{figure}[htbp]
   \vspace{2em}
   \centering
   %name of the graphic, without the path AND in EPS format:
   \includegraphics[scale=0.5]{jim_riddoch.eps}
   %caption of the figure 
   \caption*{\small \em Jim Riddoch.}
   %label of the figure, which has to correspond to \ref{}:
   \label{fig:jim_riddoch.eps}
\end{figure}

In 1947, while Prime Minister Mackenzie King was in London, his
Minister of Foreign Affairs (and chosen suc-cessor), Louis St-Laurent,
consented to Canadian participation on the United Nations Temporary
Commission of Korea (UNTOK). On his return to Canada, King furiously
confronted St-Laurent. He wanted no part---even at the fringes—of
Korean affairs, which he considered dangerous.

St-Laurent, in turn, threatened to resign. Ultimately, the old prime
minister backed down. The incident marked a phase in the changing of
the guard in Ottawa, and a movement towards increasing Canadian
involvement in international affairs.

Mackenzie King died on July 22, 1950. As he lay in state in the
Parliament Buildings, six North Star aircraft of No. 426 Squadron flew
over, saluting his passing as they were en route to Tacoma,
Washington, to commence a Royal Canadian Air Force transoceanic
airlift in sup-port of United Nations forces fighting in Korea.

It is not known if the former prime minister was spin-ning in his
coffin.

Although the RCAF had considerable air transport experience during
the Second World War, only the work of No. 168 (Heavy Transport)
Sqdn., delivering service mail to Europe, constituted a sustained
transatlantic operation. With the retirement of its Fortress and
Liberator aircraft, which were converted from bombers to freighters,
the RCAF briefly lost its long-range transport capability. This gap
was soon filled by the Canadair North Star, built in Montreal for the
RCAF, TCA (Trans-Canada Airlines and later Air Canada) and other civil
carriers.

The North Star was basically a Douglas DC-4, but with Rolls-Royce
Merlin in-line engines instead of Pratt-andWhitney radials. It was a
dependable aircraft, but generations of passengers and crew would
remember the tremendous noise and vibration in flight. The full story
of this is related by Larry Milberry in his 1982 book The Canadair
North Star.

Deliveries of the new aircraft to the RCAF were delayed when
priority was given to TCA to help it establish a Canadian presence in
transatlantic commercial flying. Crews of No. 426 (Thunderbird)
Sqdn. finally began North Star training early in 1948, and actual
operations on the type commenced that summer.

While the RCAF did not participate in the July 1948 to May 1949 Berlin
Airlift, a few individual aircrew did while on exchange duties with
the Royal Air Force and United States Air Force. As it happened, the
airlift seriously degraded the USAF Military Air Transport Service
(MATS) after flour and coal dust penetrated the seams of every
participating aircraft. When it was next marshalled to meet the Korean
crisis, MATS needed all the help it could get—and that was forthcoming
when the Canadian government committed No. 426 Sqdn.—its only
long-range transport unit—to the Korean campaign.

In recounting the RCAF’s transpacific operational story, one must pay
tribute to Laurence Motiuk, author of two monumental volumes, namely
Thunderbirds at War: Diary of a Bomber Squadron, and Thunderbirds for
Peace: Diary of a Transport Squadron, which set out the history of
No. 426 Sqdn. The latter work was particularly difficult to write,
given that in the 1950s, RCAF units often submitted incomplete
historical narratives and in some instances neglected to prepare any
at all.

Motiuk’s account of the postwar squadron was drawn from numerous
unofficial records, including log books as well as the sketchy unit
reports.

Almost from the moment that North Korea launched its invasion of South
Korea on June 25, 1950, No. 426 Sqdn. personnel were preparing for
overseas operations. If Canada was going to be involved, that unit was
certain to be mobilized. Cabinet approval came through on July 19,
1950. The squadron cancelled all further transport work with the
exception of vital northern resupply.

Deployment to McChord Air Force Base, Tacoma, Washington, began on the
25th. Six North Stars, 12 crews and 185 ground personnel were
dispatched. Twothirds of the aircrew were veterans of the Second
World War; the remainder were men who had joined the RCAF after
1947. Three of the aircraft left Tacoma on the 27th, arriving at
Haneda in Tokyo two days later. Operation Hawk was under way.

It might be noted that No. 426 Sqdn. was not alone in supporting
operations in Korea. Historian Carl Mills reminds us that Canadian
Pacific Airlines provided charter services to MATS from August 1950 to
March 1955, conducting 703 transpacific flights. Unlike the RCAF
operation, these involved passengers only. They lost one aircraft
(CF-CPC), a DC-4 with seven crew members and 31 passengers on July 2,
1951. The aircraft, outbound from Vancouver, disappeared near
Yakutat, Alaska; no trace was ever found.

The North Stars followed a challenging route. From McChord to
Elmendorf Air Force Base in Alaska was 1,490 miles. This was followed
by a 1,537-mile run to Shemya Air Force Base in the Aleutian
Islands. Next came the 2,104mile flight to Haneda in Tokyo, with the
aircraft making landfall at Matsushima (Honshu, Japan).

The Shemya-Matsushima leg was particularly challenging; care had to
be taken not to drift into Russian airspace, a task complicated by
Russian efforts to jam radio and navigational aids. Depending on the
season, the aircraft might return by the same route, or take a more
southerly central Pacific track from Haneda to Wake Island to Hickam
Air Force Base in Honolulu. From there the aircraft would journey to
Travis Air Force Base in San Francisco, and then on to McChord.

Initially it was hoped the unit could dispatch one sortie per day, but
this proved beyond the resources of a six-plane outfit. Eventually,
No. 426 found it could send off five aircraft per week. Not
surprisingly, its peak activity coincided with the buildup of United
Nations forces in Korea and their maintenance during the period of
mobile warfare.

Between July 28 and Dec. 31, 1950, the squadron dispatched 123
aircraft across the Pacific. In the whole of 1951 it was 193 missions,
but in 1952 it dropped to 133. Not only were the Thunderbirds
returning to more traditional duties in Canada, they were also
assuming the job of providing transport, including personnel
rotation, to Canada’s growing land and air presence in Europe. In
1953—the last full year of Hawk—it was 85 missions.

The final Hawk mission departed McChord on May 31, 1954, and ended at
the squadron’s home base at Dorval on June 9, 1954. As it wound up, an
RCAF dispatch reported that No. 426 Sqdn. had flown 599 round
trips—four of them directly to Korea itself—logged 34,000 flying
hours, carried 13,000 personnel, and airlifted 3,500 tons of
freight. The report erred on two points. Six flights (not four) had
gone directly to Korea. The other discrepancy was that there were
only 584 missions; for some unknown reason the numbered listing of
flights had jumped from “439” on Nov. 26, 1952, to “455” on Nov. 28,
1952; the inaccurate 1954 press release continues to be quoted, ad
nauseam, to this day.

Operation Hawk was accomplished with no fatalities but some close
calls and more than a few incidents. On Sept. 15, 1950, Wing Commander
C.H. Mussells, Commanding Officer of No. 426 Sqdn., departed McChord
for Elmendorf with American soldiers and anti-tank ammunition. Both
were urgently needed in Korea, so the aircraft had been permitted an
overload clearance.

Three hours after takeoff, one of the Merlin engines began overheating
through a coolant leak. Mussells shut down the engine, feathered the
propeller, and prepared to return to McChord. Then a second engine on
the same side overheated. It was also shut down. There was now no
question about getting to McChord; Mussells prepared to make an
emergency landing at Sandspit (Moresby Island).

To lighten the aircraft, Mussells dumped excess fuel. Some passengers
mistook the misty gasoline trail for smoke, panicked and crowded to
the back of the fuselage, as if that would have offered safety. This
complicated the process of trimming the aircraft for landing. A
crewman ordered everyone back to their seats, but the soldiers could
not have been reassured as the North Star flew over a crashed USAF
aircraft, its tail still sticking out of the water, during the final
approach into Sandspit. Happily, Mussells landed safely.

On April 19, 1951, a North Star piloted by Flight Lieutenant J.A. Watt
was on a route-training flight to an unfamiliar field—Ashiya—located
between Osaka and Kobe, Japan. He was cleared by Ashiya tower to
descend from 4,000 to 3,000 feet. At 3,400 feet the aircraft hit trees
atop a hill that was not even indicated on the airfield map. No one
was injured but the aircraft sustained considerable damage to its
nose, oil cooler and exterior radio aerials. The pitot head was also
wiped out, and with it the airspeed indicator. The aircraft also lost
the use of an engine.

Fortunately, the passengers included three of the most experienced
North Star captains in the RCAF: Wing Cmdr. Mussells, Wing
Cmdr. J.K. MacDonald, and Flying Officer Robert Edwards. After
assessing the damage, Mussells ordered MacDonald and Edwards to com-
plete the landing, which they did.

On the night of Dec. 27, 1953, Squadron Leader E.L. Hare was homeward
bound from Japan in North Star 17516, bound for Shemya. The weather
was bad and the ceiling minimal. Visibility was a half mile with a
savage crosswind blowing at a 90-degree angle. The runway surface was
covered with wet snow; braking conditions were poor. On the third
landing attempt, everything came together in the worst possible
way. The North Star was blown off the runway into a gully. Everyone
survived, but 17516 was a total “write-off”–the only RCAF aircraft
lost in the course of Operation Hawk.

Canadian military policy with respect to honours and awards was very
restrictive from 1946 onwards, but rules were relaxed during the
Korean War. Personnel involved in Hawk were awarded one Officer of the
Order of the British Empire (OBE), one Member of the Order of the
British Empire (MBE), four Air Force Crosses, two Air Force Medals and
13 Queen’s Commendations. Most of these were to aircrew personnel, but
Sqdn. Ldr. W.H. Lord was granted an MBE for his work in establishing
and supervising the maintenance detachments that serviced the North
Stars wherever they alighted along the 11,000 mile route.

\begin{footnotesize}
    \raggedleft PNSAC\\
\end{footnotesize}

% End of text.

%%% Local Variables: 
%%% mode: latex
%%% TeX-master: main_document.tex
%%% End: 

