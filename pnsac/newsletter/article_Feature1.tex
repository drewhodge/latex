% Template PNSAC newsletter - Article
% Language: Latex
%

% Head

\title{My Days at the Home of the Merlin Engine}
\subtitle{Part 3}
\author{Richard Lodge}

\maketitle

``Lodge, the expenses on the charity wagon go to Miss Prod''. What was a
charity wagon and who was Miss Prod? I was working in the Overheads
Department of the Rolls-Royce Aero Engine Division in Derby. As a
young, ambitious accountant, I did not want to admit to the people in
the department that I did not know what they were talking about. I was
conscious that I had spent several weeks of initiation under the
guidance of the old timer, Les Hart, and should now know something
about the company. I was sitting in front of a huge leather bound
ledger where accounting information was recorded. Eventually, by
listening to other people talking around me, I established that a
charity wagon was a much prized company car and that Miss Prod was not
a person but referred to ``miscellaneous production'' work at R-R. Put
in plain English, the sentence meant that the cost of repairing this
individual's car was to be charged to a special account in the
miscellaneous production section of the overheads. There was so much
to learn!  I was completely bewildered. Each day was a challenge and
the thought that aero engines were produced in the surrounding
factories seemed very far away.

On being placed in the Overheads Department, I was given an internal
mail address. The company was so large that a worldwide internal mail
system had been established and I rapidly found out that I was at the
bottom of the heap. My address was JD/JS/LOD which meant that JS was
my immediate supervisor and JD was his boss. If you were important in
the company, your mail would come to you with just your initials such
as JD. Everybody with any ambition wanted to get a single initials
mailing address. If you were really close to God, such as a director
of the company, your mailing address would be Psn for the CEO, Sir
Denning Pearson, or DPH for David Huddie, the Director of Engineering.

There were continuous battles between the engine development engineers
and the accountants. When the engineers spent money, they tried
wherever possible to avoid having it charged against their
budget. Their favourite game was to charge the cost into the overheads
where it would be lost in a mass of other expenditure such as emery
paper and toilet rolls. Our task was to catch them at their game and
reallocate the expense back to them. The development engineers were
very skilled at this part of their work and became so good at it that
they were able to hide major cost overruns, sometimes
indefinitely. Little did I realize that I was helping to perpetuate
the culture that was one of the causes of R-R going bust a few years
later.

In the 1960s R-R was a paternalistic organization, steeped in
tradition, still operating with many cost-plus government contracts,
vast amounts of paper and virtually no computers. The Ministry of
Aviation would regularly send auditors into the company to check how
public money was being spent. These gentlemen had a permanent office
allocated to them and would regularly attack the expenses recorded by
the Overheads Department. When this happened, we and the development
engineers became great buddies for a short time. Our unspoken job was
to confuse the government auditors as much as possible so that they
would not report back to their masters that things were not as they
should be. We became quite good at our confusion duties. It was not a
difficult job because we were frequently confused by all the paper and
numbers as well.

R-R continued to deliver new aero engines to customers and I gradually
started to understand what I was doing. After a few months, my
supervisor was promoted and I was appointed supervisor of the
Overheads Department, much to the disgust of many of the people in the
department. Although my pleasure was to be short-lived, I was
delighted to get this promotion. By the time of my promotion, I had
been around many of the facilities of R-R both in the Derby area and
in Scotland. I felt that I knew my way around the company.

Almost everyone working in the Overheads Department, including the
three section leaders under me, were older and more experienced than
I. They did not like this young upstart accountant becoming their
boss. Underground warfare started between us. Several of the men had
fought through WW2 and were well practiced at the gentle art of ``dumb
insolence''. The three section leaders came to me one day and told me
they really disliked me being their boss. This was not good for my
ego. I suggested to them that the four of us should go out after work
and have a beer and that on this occasion I would just be one of them
and not their boss. I bought the first round of beer and they really
tore into me. After several more rounds of beer, we agreed to a truce
and decided on a new strategy. I would stop trying to micromanage them
and they would stop their dumb insolence. Of course, they never
admitted that they were practicing dumb insolence. Life became much
better after our beers. Miss Prods, charity wagons, drop-offs and all
the other R-R jargon became normal for me with the active co-operation
of my three section leaders.

Apart from the pay rise upon my promotion I had now moved up in the
company mailing system and my address had become JD/LOD. This probably
helped my head to swell until the three section leaders brought me
rapidly back to earth.

I must have learnt a few of life's lessons about people management
because when I was again promoted, my three section leaders tried to
encourage me to refuse the promotion. No chance. I was once again
moving up the mailing system and would in future have my mail
addressed to LOD, without a more senior person's initials appearing
before mine. I was going in the right direction.



\begin{footnotesize}
    \raggedleft PNSAC\\
\end{footnotesize}

% End of text.

%%% Local Variables: 
%%% mode: latex
%%% TeX-master: main_document.tex
%%% End: 

