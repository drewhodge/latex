% Template PNSAC newsletter - Article
% Language: Latex
%

% Head

\title{Reunion of Giants}
%\subtitle{Part 4}
\author{Bruce Grant}

\maketitle

% \textit{Photos courtesy of Library Archives Canada and 426 Thunderbird
%  Sqn, RCAF.}\\

The "Reunion of Giants" video came to the Canadian Aviation and Space Museum's
theatre on May 14th, courtesy of the Canadian Warplane Heritage Museum in
Hamilton. CWHM sent along the pilot Captain Leon Evans (retired Air Canada), and
Dave Rohrer, CEO, as well as Morgan Elliott the videographer. Also present were
two World War Two decorated Lancaster pilots Don McKechnie and Bill Button. With
almost a full house of 200 it was a very successful Project North Star public
event.


The outstanding feature of the story is the audacity of CWHM to fly Canada's
only airworthy Lancaster to England for a reunion with the only other flying
Lancaster in the world. The Lanc is the flagship of the CWHM, its biggest
attraction and an irreplaceable historic artifact. In a typical year it flies
only about fifty hours, attending aviation events in Canada and every move is
calculated some years in advance.


In England, the arrival over the North Atlantic route of our Lanc was embraced
as a tribute to the quality of the machine and the courage of its crew. It
reinforced the link with Canada which provided so much support to Great Britain
during the war.


While on a final approach in England, the supercharger on engine \#3 went to
pieces and the plane landed with a trail of smoke. A replacement engine would
have to be found. The replacement that turned up was a Rolls Royce Merlin that
would have to be substituted for the Packard Merlin in our Canadian-built
Lancaster. This presented a challenge in sorting out the elements of QEC with
incompatible Whitworth and SAE fasteners. The new engine would be a "bitsa"
(bitsa this and bitsa that). They did it in about three days and got the plane
back in the air.


On the return flight, stronger than anticipated headwinds raised the risk of
running out of gas over the Davis Strait. They diverted to Narsarsuaq in
Greenland.


Following the showing of the film there was a spirited question and answer
session with CWHM's representatives moderately by Mike Pearson (retired Air
Canada pilot of the Gimli glider fame). There is no doubt that all those in
attendance enjoyed themselves thoroughly and gained some understanding of what
it meant for those involved in this project.


At Project North Star, we have some appreciation idea of the scale of CWHM's
accomplishment. As our members know we are in the twelfth year of our project to
restore the last surviving North Star. But, we are restoring it only as a museum
artifact, not an airworthy machine. The video is available for sale at
www.warplane.com and the trailer can be viewed ath www.suddenlyseemore.com .

%\begin{quotation}
%	\textit{You have to be careful opening up the panels covering the 
%	engines, you never know if a bird or some animal might have made
%	its home there.}
%\end{quotation}

%\begin{figure*}[htbp]
%	\vspace{2em}
%	\centering
%	%name of the graphic, without the path AND in EPS format:
%	\includegraphics[scale=0.5]{op-hawk-one.jpg}
%	%caption of the figure 
%	%\caption*{\small \em Volunteers put in countless hours restoring a
%	North Star aircraft.}
%	%label of the figure, which has to correspond to \ref{}:
%	\label{fig:op-hawk-one.eps}
%\end{figure*}


%\textit{Reprinted from the March/April 2014 edition of "Legion Magazine," with the kind permission of Jennifer Morse, General Manager of Canvet Publications, and Tom MacGregor.}


\begin{footnotesize}
    \raggedleft PNSAC\\
\end{footnotesize}

% End of text.

%%% Local Variables: 
%%% mode: latex
%%% TeX-master: main_document.tex
%%% End: 

