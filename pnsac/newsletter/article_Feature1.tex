% Template PNSAC newsletter - Article
% Language: Latex
%

% Head

\title{The Four Merlins: The Second Two}
%\subtitle{Part 4}
\author{Based on Bruce Gemmill's notes; reviewed by Garry Dupont, and with editing assistance from Richard Lodge}

\maketitle

% \paragraph*{Engine 3}

After its removal in the Spring of 2012, engine \#3 was completely stripped of
all accessories, pipes and hoses, and disassembled.  The block, cylinders and
pistons were cleaned, and work was begun to reassemble this engine.  It was at
this point that a lack of volunteers started to slow down the restoration
process.

The engine rebuild was nearing completion by April 2013, with magnetos and a few
other accessories left to be installed.  The basic engine frame was back
together.  The three radiator sections were cleaned and painted, then attached
to the front of the frame.  The rear cowl ring had been disassembled, and new
mounting hardware was fitted before the parts were painted and reassembled.  The
front cowl assembly (radiator cover) was also cleaned, repaired and polished,
ready to be installed.  Other cowl panels were then cleaned, repaired and
polished, ready to install once the complete engine was in place.  Work was also
underway on a host of pipes, cables, accessories, and fittings needed to
complete the assembly of the engine and the Quick Engine Change (QEC) prior to
installation on the aircraft.

By September 2013 engine \#3 was almost completely reassembled, with only the
supercharger and inter cooler preheat assemblies left to install on the engine,
which had been moved to the QEC engine frame in early summer.  The work slowed,
due to summer down time for staff and volunteers. The auxiliary gearbox was
removed from the engine firewall and was then being disassembled.  This was the
last major assembly that had to be overhauled before the engine could be placed
back on the aircraft.

Engine \#3 was nearing completion by December 2013. The supercharger and inter
cooler preheat assemblies had been installed, along with the propeller reduction
gearbox.  While this work was being carried out by the engine crew, the
remainder of the volunteer workforce busied themselves completing the numerous
cowl panels, pipes, hose and other pieces needed to complete the engine.  Two
items that seemed to take forever were the large steel exhaust shrouds.  A lot
of hammering was needed to pound out years of dents, then numerous rivets and
all the Dzus fasteners and springs needed to be replaced.  The complex shape of
the exhaust shroud made this challenging work.  Further delaying completion was
the search for suitable clear coat to protect the steel from corrosion.  The old
clear coat used successfully on engines \#1 and \#2 was no longer available.  A
suitable replacement was located, the shrouds were clear coated and serial
numbers were stenciled on.

The auxiliary gearbox was disassembled, and significant corrosion was found on
the bearings inside.  Bearings had to be ordered from England.  Meanwhile, the
gearbox was cleaned and painted.  At the same time the large DC generator was
dismantled, repainted and reassembled, as well as the air pump, and the
tachometer generator and propeller synchronizer.  Once the new bearings were
installed, the gearbox was completed, and the generator, and air pump were
attached.  Finally, the gearbox was installed on the firewall awaiting the
installation of the engine.

In April 2013 \#3 engine frame was completed after the installation of the engine
and supercharger.  Fitting the many cowl panels proved difficult, due to the
extensive rework that many of these items went through, causing some alignment
difficulties when fitting these panels onto the engine frame.  This work was
done in July 2014, and the completed engine was successfully installed on the
aircraft.

\begin{figure}[H]
   \vspace{2em}
   \centering
   %name of the graphic, without the path AND in EPS format:
   \includegraphics[scale=0.5]{img-1-merlins-aug2021.png}
   %caption of the figure 
   \caption*{\small \em Volunteers Bill Tate and Garry Dupont discussing work on Engine 3.}
   %label of the figure, which has to correspond to \ref{}:
   \label{fig:img1}
\end{figure}

% \paragraph*{Engine 4}

The last engine to undergo restoration was removed iand work
immediately began in the Engine Shop to disassemble the engine.  All cowl
panels, pipes, hoses and accessories were removed and stored until space was
available to restore these items. The engine was moved to the yellow rotating
stand at this point. The priority was to disassemble the engine.]. The crank
case was cleaned and clear coated before installing back into the yellow stand.
The refurbished crankshaft was installed in the engine block, and the connecting
rods and pistons cleaned and polished. On this engine we were able to remove the
propeller shaft from the reduction gear box, unlike engine \#3 the shaft could
not be removed.  Work was then begun on the more complex cylinder heads and
valve train assemblies.

Work then proceeded on much of the ancillary equipment for the engine. Once the
frame was completed and placed on the blue stand, the engine block was removed
from the yellow stand and installed in the frame. The supercharger \& inter
cooler were completed and installed, as well as the reduction gearbox. The
radiators and header tank were restored and painted, and two of the three
radiator sections were installed. Restoration work was then begun on the air
intake system, and many of the fuel and oil lines.  The starter motor was found
to be very corroded and disassembly and repair not viable. A replacement unit
was located and installed. The fire suppression lines were complete and would be
installed along with the cowls when ready. Some of the cowl panels were
restored, although it was found that this engine had more corrosion than the
other three engines. Possibly it had been in service longer.  This meant that
more work was required to refurbish each of the many panels.

After restarting work on the aircraft in late January 2018, the conservation
effort focused on the completion of engine number \#4 accessories and cowl
panels.   Many of the cowl panels required extensive repairs, including making
many new panel stiffeners, since the old ones were often badly corroded.
Besides painting and polishing, the refurbished panels also required new part
number and serial number stencils, and this led to investigating missing
stencils from the same panels on earlier engines.  

Before final assembly could take place, the radiator cowl panel had to be fitted
which was done by December. The panels were fitted and installed with some
fasteners being replaced. The front support was installed for transportation.

\begin{figure}[H]
    \vspace{2em}
    \centering
    %name of the graphic, without the path AND in EPS format:
    \includegraphics[scale=0.5]{img-2-merlins-aug2021.png}
    %caption of the figure 
    \caption*{\small \em Engine 4 completed and ready to be moved to the Storage Hangar.}
    %label of the figure, which has to correspond to \ref{}:
    \label{fig:img2}
 \end{figure}

 \begin{figure}[H]
    \vspace{2em}
    \centering
    %name of the graphic, without the path AND in EPS format:
    \includegraphics[scale=0.5]{img-3-merlins-aug2021.png}
    %caption of the figure 
    \caption*{\small \em Some of the crew who worked on the restoration of Engine 4 (Richard Lodge, R\'{e}j Demers, Garry DuPont, Ted Devey, Charles Baril).}
    %label of the figure, which has to correspond to \ref{}:
    \label{fig:img3}
 \end{figure}

On December 19, 2019, the engine, still on its blue stand, was towed by the
Museum forklift, from the Restoration Shop to the Storage Hangar where it now
sits next to the North Star prior to re-installation on the aircraft.

\begin{figure}[H]
    \vspace{2em}
    \centering
    %name of the graphic, without the path AND in EPS format:
    \includegraphics[scale=0.5]{img-4-merlins-aug2021.png}
    %caption of the figure 
    \caption*{\small \em R\'{e}j Demers (Museum Special Projects Manager) very carefully bringing the Museum forklift to lift one end of Engine 4 ready for moving from the Engine Shop to the Storage Hangar on 13 December 2019.}
    %label of the figure, which has to correspond to \ref{}:
    \label{fig:img4}
 \end{figure}

 \begin{figure}[H]
    \vspace{2em}
    \centering
    %name of the graphic, without the path AND in EPS format:
    \includegraphics[scale=0.5]{img-5-merlins-aug2021.png}
    %caption of the figure 
    \caption*{\small \em Engine 4 starts it long, slow, cold 20 minute journey from the Engine Shop to the Storage Hanger to join the North Star aircraft.}
    %label of the figure, which has to correspond to \ref{}:
    \label{fig:img5}
 \end{figure}
 
 \begin{figure}[H]
    \vspace{2em}
    \centering
    %name of the graphic, without the path AND in EPS format:
    \includegraphics[scale=0.5]{img-6-merlins-aug2021.png}
    %caption of the figure 
    \caption*{\small \em The Museum’s forklift, driven with great skill by  R\'{e}j Demers, hauls Engine 4 towards the Storage Hangar.  The blue engine stand and its tiny wheels certainly was not made for racing!}
    %label of the figure, which has to correspond to \ref{}:
    \label{fig:img6}
 \end{figure}

2020 Finally early in 2020, before the lock down,  propeller \#4 was assembled
and stored on the refurbished prop stand in the Storage Hangar ready to mount on
engine \#4. 




% \begin{figure}[H]
%    \vspace{2em}
%    \centering
%    %name of the graphic, without the path AND in EPS format:
%    \includegraphics[scale=0.5]{four-merlins-1.png}
%    %caption of the figure 
%    \caption*{\small \em North Star being moved inside hangar 11 August 2005 .}
%    %label of the figure, which has to correspond to \ref{}:
%    \label{fig:tim}
% \end{figure}



 

% \begin{figure}[htbp]
%    \vspace{2em}
%    \centering
%    %name of the graphic, without the path AND in EPS format:
%    \includegraphics[scale=0.5]{four-merlins-2.png}
%    %caption of the figure 
%    \caption*{\small \em  Mike Hope and Bruce Gemmill assembled Prop 1ready to
% install on Engine 1 October 26, 2008.}
%    %label of the figure, which has to correspond to \ref{}:
%    \label{fig:tim}
% \end{figure}



% \begin{figure}[htbp]
%    \vspace{2em}
%    \centering
%    %name of the graphic, without the path AND in EPS format:
%    \includegraphics[scale=0.55]{four-merlins-3.png}
%    %caption of the figure 
%    \caption*{\small \em Engine 1 assembled in the Conservation shop Feb 2,
% 2010.}
%    %label of the figure, which has to correspond to \ref{}:
%    \label{fig:tim}
% \end{figure}

% \begin{figure}[htbp]
%    \vspace{2em}
%    \centering
%    %name of the graphic, without the path AND in EPS format:
%    \includegraphics[scale=0.55]{four-merlins-4.png}
%    %caption of the figure 
%    \caption*{\small \em Firewall ready, auxiliary gearbox installed Jan 7,
% 2010.}
%    %label of the figure, which has to correspond to \ref{}:
%    \label{fig:tim}
% \end{figure}



% \begin{figure}[htbp]
%    \vspace{2em}
%    \centering
%    %name of the graphic, without the path AND in EPS format:
%    \includegraphics[scale=0.55]{four-merlins-5.png}
%    %caption of the figure 
%    \caption*{\small \em Engine 1 complete at the Restoration Shop Feb 9, 2010
% .}
%    %label of the figure, which has to correspond to \ref{}:
%    \label{fig:tim}
% \end{figure}

% \begin{figure}[htbp]
%    \vspace{2em}
%    \centering
%    %name of the graphic, without the path AND in EPS format:
%    \includegraphics[scale=0.45]{four-merlins-6.png}
%    %caption of the figure 
%    \caption*{\small \em Engine 1 installed Feb 24, 2010. Shown top left to
% bottom right, engine
% supervisor Ted Devey,  Bill Tate, Ron Lemieux, Giuseppe Zanetti, John Corby,
% Garnet Chapman, Volunteer Project Manager Jim Riddoch, John Tasseron, Bruce
% Gemmill, Museum Project Coordinator Mike Irvin, Charles Baril, and then
% Association President Tim Timmins.}
%    %label of the figure, which has to correspond to \ref{}:
%    \label{fig:tim}
% \end{figure}



% \paragraph*{Engine 2}\



% \begin{figure}[htbp]
%    \vspace{2em}
%    \centering
%    %name of the graphic, without the path AND in EPS format:
%    \includegraphics[scale=0.5]{four-merlins-8.png}
%    %caption of the figure 
%    \caption*{\small \em Engine 2 with cowl panels removed, April 6, 2010.}
%    %label of the figure, which has to correspond to \ref{}:
%    \label{fig:tim}
% \end{figure}



% \begin{figure}[htbp]
%    \vspace{2em}
%    \centering
%    %name of the graphic, without the path AND in EPS format:
%    \includegraphics[scale=0.5]{four-merlins-9.png}
%    %caption of the figure 
%    \caption*{\small \em Engine 2 removed April 14, 2010.}
%    %label of the figure, which has to correspond to \ref{}:
%    \label{fig:tim}
% \end{figure}



% \begin{figure}[htbp]
%    \vspace{2em}
%    \centering
%    %name of the graphic, without the path AND in EPS format:
%    \includegraphics[scale=0.5]{four-merlins-10.png}
%    %caption of the figure 
%    \caption*{\small \em Stan Rideout working to reassemble Engine 2 in engine
% shop, Oct 7, 2010.}
%    %label of the figure, which has to correspond to \ref{}:
%    \label{fig:tim}
% \end{figure}



% \begin{figure}[htbp]
%    \vspace{2em}
%    \centering
%    %name of the graphic, without the path AND in EPS format:
%    \includegraphics[scale=0.5]{four-merlins-11.png}
%    %caption of the figure 
%    \caption*{\small \em Austin (Tim) Timmins.}
%    %label of the figure, which has to correspond to \ref{}:
%    \label{fig:tim}
% \end{figure}



% \begin{figure}[htbp]
%    \vspace{2em}
%    \centering
%    %name of the graphic, without the path AND in EPS format:
%    \includegraphics[scale=0.5]{four-merlins-12.png}
%    %caption of the figure 
%    \caption*{\small \em Jim Riddoch and Bill Tate fill the radiators with
% inhibiting oil.}
%    %label of the figure, which has to correspond to \ref{}:
%    \label{fig:tim}
% \end{figure}


% \begin{figure}[htbp]
%    \vspace{2em}
%    \centering
%    %name of the graphic, without the path AND in EPS format:
%    \includegraphics[scale=0.7]{four-merlins-13.png}
%    %caption of the figure 
%    \caption*{\small \em Engine 2 on rotating assembly stand.}
%    %label of the figure, which has to correspond to \ref{}:
%    \label{fig:tim}
% \end{figure}

 

% \begin{figure}[htbp]
%    \vspace{2em}
%    \centering
%    %name of the graphic, without the path AND in EPS format:
%    \includegraphics[scale=0.45]{four-merlins-14.png}
%    %caption of the figure 
%    \caption*{\small \em  Engine Nr 2 being re-assembled in the engine shop.}
%    %label of the figure, which has to correspond to \ref{}:
%    \label{fig:tim}
% \end{figure}



% \begin{figure}[htbp]
%    \vspace{2em}
%    \centering
%    %name of the graphic, without the path AND in EPS format:
%    \includegraphics[scale=0.5]{four-merlins-15.png}
%    %caption of the figure 
%   % \caption*{\small \em .}
%    %label of the figure, which has to correspond to \ref{}:
%    \label{fig:tim}
% \end{figure}



\begin{footnotesize}
    \raggedleft PNSAC\\
\end{footnotesize}


% End of text.

%%% Local Variables: 
%%% mode: latex
%%% TeX-master: main_document.tex
%%% End: 

