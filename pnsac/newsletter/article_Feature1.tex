% Template PNSAC newsletter - Article
% Language: Latex
%

% Head

\title{APS 42 Radar -- Another Piece of the North Star Puzzle}

%\author{Based on Bruce Gemmill's notes; reviewed by Garry Dupont, and with editing assistance from Richard Lodge}

\maketitle

% \paragraph*{Engine 3}

\textit{The article below was written by our members John Makadi and Chris McGuffin. Chris and John played key roles in the acquisition of the APS 42 Radar for the North Star aircraft. In 2019 John was approached by R\'{e}jean Demers, CASM's project manager for the North Star restoration to see if he could locate the radar unit which was missing from the aircraft as explained below. John was successful and as a result in 2001  R\'{e}j asked Chris if  our association  could arrange for the acquisition of the unit. John and Chris coordinated its acquisition and what follows is a very fascinating story about what may be the only unit of its kind still in existence. The authors have been very modest about their contributions in dealing with a very challenging process made more difficult by the pandemic Covid-19.}

The North Star was the RCAF’s first strategic lift platform. It was dispatched across the country, and bridged the Atlantic and Pacific in service to Canada. In the late 1950s, the RCAF began upgrading the fleet with the addition of the APS-42 military navigation/weather radar. This radar was still sensitive equipment in 1966 when North Star 17515 was donated to the Canada Aviation and Space Museum. The radar was stripped out of the airframe along with the radios and other sensitive equipment.

Project North Star is frequently described as an effort to return 17515 to the condition she was in on her last day of service with the RCAF. The hunt was on. In 2019 Project North Star Association of Canada began searching for an APS-42 radar. Initial results were discouraging. US aviation museums with USAF aircraft had no components to offer. Further research of US military logistics documents revealed that US Department of Defense disposal instructions for APS-42 radar was to “Destroy” them due to their sensitive nature at the time.

\begin{figure}[H]
   \vspace{2em}
   \centering
   %name of the graphic, without the path AND in EPS format:
   \includegraphics[scale=0.45]{radar_the_unit.png}
   %caption of the figure 
   \caption*{\small \em Radar -- The Unit.}
   %label of the figure, which has to correspond to \ref{}:
   \label{fig:img1}
\end{figure}

We were disheartened by that discovery but certainly not defeated. Over the years, PNSAC volunteers have been resourceful about rescuing parts and assemblies. The search continued. In the spring of 2021, while continuing on-line research John stumbled across an eBay listing for an APS-42 transmitter out of Cross Timbers, Missouri (Ozarks). The seller, a military communications enthusiast, acquired the radar in a lot purchase of US Government surplus equipment. After negotiating an acceptable price for the radar, we just needed to get it to CASM.

\begin{figure}[H]
    \vspace{2em}
    \centering
    %name of the graphic, without the path AND in EPS format:
    \includegraphics[scale=0.5]{radar_the_aps_42.png}
    %caption of the figure 
    \caption*{\small \em Radar -- The APS-42.}
    %label of the figure, which has to correspond to \ref{}:
    \label{fig:img2}
 \end{figure}
 
Shipping freight is usually pretty straight forward. In the summer of 2021, there were “unusual” circumstances. Transportation was a bottle neck to many industries and costs rose across the carriers. We were unable to find a freight company willing to collect the radar in the seller’s remote location. Our seller was willing to deliver to Springfield, Missouri (for a fee) but would not provide a crate. Furthermore, the freight companies that offered custom crating services were prohibitively high. Then we made contact with the “Air and Military Museum of the Ozarks” (AMMO) – a small volunteer-based museum in Springfield. The volunteers at AMMO were sympathetic to our cause and enthusiastic to help. They agreed to receive our radar from the seller, build a custom crate and transfer it to the freight carrier. We were grateful to pay them a small honorarium which wasn’t much more than the cost of lumber for the crate. This was a wonderful example of museums cooperating across borders to preserve aviation heritage. AMMO ended up hosting the crated radar for several weeks. The hurdles of NAFTA attestations and the export of US military technology took additional bureaucratic kung-fu but the radar arrived safely in Ottawa on 10 June, 2021.

 \begin{figure}[H]
    \vspace{2em}
    \centering
    %name of the graphic, without the path AND in EPS format:
    \includegraphics[scale=0.5]{radar_opening_the_unit.png}
    %caption of the figure 
    \caption*{\small \em Radar -- Opening the Unit.}
    %label of the figure, which has to correspond to \ref{}:
    \label{fig:img3}
 \end{figure}

After 17 months of social distancing, Project North Star volunteers and CASM staff gathered in the parking lot to celebrate the arrival of a missing radar. Customs seals were removed and a wood crate was opened to reveal our beautifully preserved APS-42. We look forward to seeing this new artifact in the nose wheel well of 17515.

% \begin{figure}[H]
%    \vspace{2em}
%    \centering
%    %name of the graphic, without the path AND in EPS format:
%    \includegraphics[scale=0.5]{four-merlins-1.png}
%    %caption of the figure 
%    \caption*{\small \em North Star being moved inside hangar 11 August 2005 .}
%    %label of the figure, which has to correspond to \ref{}:
%    \label{fig:tim}
% \end{figure}



 

% \begin{figure}[htbp]
%    \vspace{2em}
%    \centering
%    %name of the graphic, without the path AND in EPS format:
%    \includegraphics[scale=0.5]{four-merlins-2.png}
%    %caption of the figure 
%    \caption*{\small \em  Mike Hope and Bruce Gemmill assembled Prop 1ready to
% install on Engine 1 October 26, 2008.}
%    %label of the figure, which has to correspond to \ref{}:
%    \label{fig:tim}
% \end{figure}



% \begin{figure}[htbp]
%    \vspace{2em}
%    \centering
%    %name of the graphic, without the path AND in EPS format:
%    \includegraphics[scale=0.55]{four-merlins-3.png}
%    %caption of the figure 
%    \caption*{\small \em Engine 1 assembled in the Conservation shop Feb 2,
% 2010.}
%    %label of the figure, which has to correspond to \ref{}:
%    \label{fig:tim}
% \end{figure}

% \begin{figure}[htbp]
%    \vspace{2em}
%    \centering
%    %name of the graphic, without the path AND in EPS format:
%    \includegraphics[scale=0.55]{four-merlins-4.png}
%    %caption of the figure 
%    \caption*{\small \em Firewall ready, auxiliary gearbox installed Jan 7,
% 2010.}
%    %label of the figure, which has to correspond to \ref{}:
%    \label{fig:tim}
% \end{figure}



% \begin{figure}[htbp]
%    \vspace{2em}
%    \centering
%    %name of the graphic, without the path AND in EPS format:
%    \includegraphics[scale=0.55]{four-merlins-5.png}
%    %caption of the figure 
%    \caption*{\small \em Engine 1 complete at the Restoration Shop Feb 9, 2010
% .}
%    %label of the figure, which has to correspond to \ref{}:
%    \label{fig:tim}
% \end{figure}

% \begin{figure}[htbp]
%    \vspace{2em}
%    \centering
%    %name of the graphic, without the path AND in EPS format:
%    \includegraphics[scale=0.45]{four-merlins-6.png}
%    %caption of the figure 
%    \caption*{\small \em Engine 1 installed Feb 24, 2010. Shown top left to
% bottom right, engine
% supervisor Ted Devey,  Bill Tate, Ron Lemieux, Giuseppe Zanetti, John Corby,
% Garnet Chapman, Volunteer Project Manager Jim Riddoch, John Tasseron, Bruce
% Gemmill, Museum Project Coordinator Mike Irvin, Charles Baril, and then
% Association President Tim Timmins.}
%    %label of the figure, which has to correspond to \ref{}:
%    \label{fig:tim}
% \end{figure}



% \paragraph*{Engine 2}\



% \begin{figure}[htbp]
%    \vspace{2em}
%    \centering
%    %name of the graphic, without the path AND in EPS format:
%    \includegraphics[scale=0.5]{four-merlins-8.png}
%    %caption of the figure 
%    \caption*{\small \em Engine 2 with cowl panels removed, April 6, 2010.}
%    %label of the figure, which has to correspond to \ref{}:
%    \label{fig:tim}
% \end{figure}



% \begin{figure}[htbp]
%    \vspace{2em}
%    \centering
%    %name of the graphic, without the path AND in EPS format:
%    \includegraphics[scale=0.5]{four-merlins-9.png}
%    %caption of the figure 
%    \caption*{\small \em Engine 2 removed April 14, 2010.}
%    %label of the figure, which has to correspond to \ref{}:
%    \label{fig:tim}
% \end{figure}



% \begin{figure}[htbp]
%    \vspace{2em}
%    \centering
%    %name of the graphic, without the path AND in EPS format:
%    \includegraphics[scale=0.5]{four-merlins-10.png}
%    %caption of the figure 
%    \caption*{\small \em Stan Rideout working to reassemble Engine 2 in engine
% shop, Oct 7, 2010.}
%    %label of the figure, which has to correspond to \ref{}:
%    \label{fig:tim}
% \end{figure}



% \begin{figure}[htbp]
%    \vspace{2em}
%    \centering
%    %name of the graphic, without the path AND in EPS format:
%    \includegraphics[scale=0.5]{four-merlins-11.png}
%    %caption of the figure 
%    \caption*{\small \em Austin (Tim) Timmins.}
%    %label of the figure, which has to correspond to \ref{}:
%    \label{fig:tim}
% \end{figure}



% \begin{figure}[htbp]
%    \vspace{2em}
%    \centering
%    %name of the graphic, without the path AND in EPS format:
%    \includegraphics[scale=0.5]{four-merlins-12.png}
%    %caption of the figure 
%    \caption*{\small \em Jim Riddoch and Bill Tate fill the radiators with
% inhibiting oil.}
%    %label of the figure, which has to correspond to \ref{}:
%    \label{fig:tim}
% \end{figure}


% \begin{figure}[htbp]
%    \vspace{2em}
%    \centering
%    %name of the graphic, without the path AND in EPS format:
%    \includegraphics[scale=0.7]{four-merlins-13.png}
%    %caption of the figure 
%    \caption*{\small \em Engine 2 on rotating assembly stand.}
%    %label of the figure, which has to correspond to \ref{}:
%    \label{fig:tim}
% \end{figure}

 

% \begin{figure}[htbp]
%    \vspace{2em}
%    \centering
%    %name of the graphic, without the path AND in EPS format:
%    \includegraphics[scale=0.45]{four-merlins-14.png}
%    %caption of the figure 
%    \caption*{\small \em  Engine Nr 2 being re-assembled in the engine shop.}
%    %label of the figure, which has to correspond to \ref{}:
%    \label{fig:tim}
% \end{figure}



% \begin{figure}[htbp]
%    \vspace{2em}
%    \centering
%    %name of the graphic, without the path AND in EPS format:
%    \includegraphics[scale=0.5]{four-merlins-15.png}
%    %caption of the figure 
%   % \caption*{\small \em .}
%    %label of the figure, which has to correspond to \ref{}:
%    \label{fig:tim}
% \end{figure}



\begin{footnotesize}
    \raggedleft PNSAC\\
\end{footnotesize}


% End of text.

%%% Local Variables: 
%%% mode: latex
%%% TeX-master: main_document.tex
%%% End: 

