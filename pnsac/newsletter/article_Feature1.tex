% Template PNSAC newsletter - Article
% Language: Latex
%

% Head

\title{The Four Merlins: The First Two}
%\subtitle{Part 4}
\author{Bruce Gemmill}

\maketitle

When North Star 17515 was retired from service with the RCAF, it was delivered
to the National Aviation Collection at Rockcliffe  in 1966, where it sat
exposed to the elements for many years. When a volunteer group was formed in
2001 to restore the aircraft, the priority was to move the aircraft inside the
newly constructed Reserve Hangar to prevent further deterioration.  This was
accomplished on August 11, 2005 when the North Star was moved inside for the
first time.

\begin{figure}[H]
   \vspace{2em}
   \centering
   %name of the graphic, without the path AND in EPS format:
   \includegraphics[scale=0.5]{four-merlins-1.png}
   %caption of the figure 
   \caption*{\small \em North Star being moved inside hangar 11 August 2005 .}
   %label of the figure, which has to correspond to \ref{}:
   \label{fig:tim}
\end{figure}

The restoration project attracted many aviation enthusiasts and the early group
of volunteers were eager to work on restoring the four Rolls Royce engines. The
Canada Aviation Museum (as it was then called) agreed and set this as the top
priority for the volunteers.  Work started in earnest in 2006 under the
supervision of the Museum's Project Coordinator, Mike Irvin.

This is the detailed  history of the restoration work on the first two Merlin
engines.

\paragraph*{Engine 1}\

The removal, restoration and assembly of Engine Nr 1 began in June 2006 and was
completed in Feb 2010 with the installation of the engine back on the North
Star.  This work included a complete teardown and re-assembly of all major
components including the Merlin V12 engine, propeller, supercharger,
intercooler and auxiliary gearbox, as well as the engine frame, cowlings,
radiators, pipes and accessories. 

Before the engine could be removed, the propeller and spinner cover needed to
be removed.  The spinners were popular nesting sites for birds, and the bird
droppings had caused extensive corrosion to the propeller hub.  To remove the
hub, a large wrench was fabricated to fit over the large, threaded ring that
held the hub firmly in place.  This was struck repeatedly with a 5 lb sledge
hammer before the ring began to move.  This process was repeated each time a
propeller was removed.  The propeller was then disassembled and sent to Hope
Aero in Mississauga, Ontario for restoration.

Engine 1 was removed and mounted in a mobile assembly stand and moved to the
conservation shop for inspection disassembly. Since this was the first Merlin
engine any of the volunteers had worked on, much time was taken photographing
the engine from all angles, and studying the Repair and Overhaul manuals that
the RCAF and Rolls Royce had provided.  The digital photographs were also an
important conservation record for the museum.

Each engine is contained within a Quick Engine Change (QEC) assembly or power
pack. This consisted of a rigid frame to hold the cowl panels which surround
the engine, and also hold such vital parts as the two radiator flaps, the three
piece radiator itself, and support the fire suppression system and various
pipes and hoses.  This frame also holds the four attachment points to secure
the engine assembly to the aircraft.

The cowl panels were removed, along with the support frame, to expose the
engine.  Large amounts of nesting material were removed, even though museum
staff had attempted to "bird-proof" the aircraft while outside.  Old engine oil
was drained from the crankcase, and the volunteers began familiarizing
themselves with the numerous parts of the wonderful V-12 Merlin engine they
would work on for the next four years. 

Along with learning the intricacies of the Merlin, a substantial number of
special tools and jigs needed to be found or made.  This began with a full set
of British sockets and wrenches for the Whitworth and BA nuts and bolts used
throughout the engine.

When it came time to remove the cylinder head, the volunteers found the pistons
were seized in place. To free up the pistons, large amounts of penetrating oil
was applied, then each piston struck with a large sledge hammer and block of
wood.

While this work was going on at the museum, Hope Aerospace in Mississauga was
restoring the first of the four Hamilton propellers sent to them under
contract. The first propeller was returned from Hope Aero after restoration in
2008, and Mike Hope volunteered his time to help assemble it.   

\begin{figure}[htbp]
   \vspace{2em}
   \centering
   %name of the graphic, without the path AND in EPS format:
   \includegraphics[scale=0.5]{four-merlins-2.png}
   %caption of the figure 
   \caption*{\small \em  Mike Hope and Bruce Gemmill assembled Prop 1ready to
install on Engine 1 October 26, 2008.}
   %label of the figure, which has to correspond to \ref{}:
   \label{fig:tim}
\end{figure}

By early 2009, most of the main engine components had been cleaned, and painted
or polished, ready to be assembled on the completed engine block and cylinder
banks one and two.

Reassembly took place on the rotary engine stand, which aided installation of
the crankshaft and other lower engine components. Below is a photo of the
nearly completed engine, showing the cylinder heads and spark plug cables in
place, before moving to the mobile engine stand for final assembly. Note the
highly polished brass expansion tank for the main radiator, nestled behind the
propeller cowl.

\begin{figure}[htbp]
   \vspace{2em}
   \centering
   %name of the graphic, without the path AND in EPS format:
   \includegraphics[scale=0.55]{four-merlins-3.png}
   %caption of the figure 
   \caption*{\small \em Engine 1 assembled in the Conservation shop Feb 2,
2010.}
   %label of the figure, which has to correspond to \ref{}:
   \label{fig:tim}
\end{figure}

Once the engine was moved to the assembly stand where the engine frame and
large radiator awaited, the wiring and various pipes and hoses could be fitted,
along with the already assembled wheelcase and supercharger. Lowering this very
large assembly into the tight space at the rear of the engine was a very
delicate manoeuvre, but the reward was a nearly completed Merlin!

After installing the massive air intake and intercooler, the various cowl
panels were fitted to the outside of the engine frame, to complete the QEC
assembly.

While this work was going on, final preparations were being made at the
aircraft to receive Engine 1.  This required cleaning and polishing the
firewall and various engine fittings, and installing the auxiliary gearbox that
drives the electrical generator and other accessories on the engine.

\begin{figure}[htbp]
   \vspace{2em}
   \centering
   %name of the graphic, without the path AND in EPS format:
   \includegraphics[scale=0.55]{four-merlins-4.png}
   %caption of the figure 
   \caption*{\small \em Firewall ready, auxiliary gearbox installed Jan 7,
2010.}
   %label of the figure, which has to correspond to \ref{}:
   \label{fig:tim}
\end{figure}



\begin{figure}[htbp]
   \vspace{2em}
   \centering
   %name of the graphic, without the path AND in EPS format:
   \includegraphics[scale=0.55]{four-merlins-5.png}
   %caption of the figure 
   \caption*{\small \em Engine 1 complete at the Restoration Shop Feb 9, 2010
.}
   %label of the figure, which has to correspond to \ref{}:
   \label{fig:tim}
\end{figure}

\begin{figure}[htbp]
   \vspace{2em}
   \centering
   %name of the graphic, without the path AND in EPS format:
   \includegraphics[scale=0.45]{four-merlins-6.png}
   %caption of the figure 
   \caption*{\small \em Engine 1 installed Feb 24, 2010. Shown top left to
bottom right, engine
supervisor Ted Devey,  Bill Tate, Ron Lemieux, Giuseppe Zanetti, John Corby,
Garnet Chapman, Volunteer Project Manager Jim Riddoch, John Tasseron, Bruce
Gemmill, Museum Project Coordinator Mike Irvin, Charles Baril, and then
Association President Tim Timmins.}
   %label of the figure, which has to correspond to \ref{}:
   \label{fig:tim}
\end{figure}

The engine was reinstalled on February 24, 2010 and the crew could take a short
break before beginning work on engine 2.

\paragraph*{Engine 2}\

Work started on the removal of Engine 2 in early April, beginning with removal
of all the cowl panels, draining any fluids left in the engine and radiators,
and disconnecting all pipes, hoses and electrical fittings at the firewall. 
Care was taken to take better photos of the removal of parts so reassembly
would not be as much a challenge second time around.

\begin{figure}[htbp]
   \vspace{2em}
   \centering
   %name of the graphic, without the path AND in EPS format:
   \includegraphics[scale=0.5]{four-merlins-8.png}
   %caption of the figure 
   \caption*{\small \em Engine 2 with cowl panels removed, April 6, 2010.}
   %label of the figure, which has to correspond to \ref{}:
   \label{fig:tim}
\end{figure}

To remove the engine, a sling was attached to the lifting points on the engine,
and a fork lift took the weight while four attachment bolts were removed from
the firewall.  Keeping the engine level while it was slowly backed away from
the aircraft was important, to avoid damage.

\begin{figure}[htbp]
   \vspace{2em}
   \centering
   %name of the graphic, without the path AND in EPS format:
   \includegraphics[scale=0.5]{four-merlins-9.png}
   %caption of the figure 
   \caption*{\small \em Engine 2 removed April 14, 2010.}
   %label of the figure, which has to correspond to \ref{}:
   \label{fig:tim}
\end{figure}

The engine team made a quick job of disassembling the number 2 engine.  All
major assemblies such as the intercooler and supercharger were removed, along
with all pipes and hoses.  The engine was removed from the frame and installed
on the rotatable engine stand.  The cylinder heads, cylinder blocks and pistons
were removed, as well as the crankshaft.  These major engine components were
cleaned so that the task of reassembling the engine could begin.  Disassembly
was not easy.  After years outside exposed to the elements, several of the
pistons were seized inside the cylinders.  Freeing these required a lot of
muscle and Liquid Wrench.  Still, the disassembly of Nr 2 went much faster than
the first engine.

A lot of corrosion was removed by a process called glass beading, but
volunteers needed to be careful not to damage aluminum surfaces or contaminate
seals.  Most aluminum components were clearcoated to protect from further
corrosion, while other parts were often painted with silver aluminum paint to
restore the original look.

As with Engine 1, this engine was assembled in the rotating stand.  Shown
below, volunteer Stan Ride is adjusting the torque on the main bearing caps.

\begin{figure}[htbp]
   \vspace{2em}
   \centering
   %name of the graphic, without the path AND in EPS format:
   \includegraphics[scale=0.5]{four-merlins-10.png}
   %caption of the figure 
   \caption*{\small \em Stan Rideout working to reassemble Engine 2 in engine
shop, Oct 7, 2010.}
   %label of the figure, which has to correspond to \ref{}:
   \label{fig:tim}
\end{figure}

By December 2010 many major components such as the engine block, crankshaft and
pistons had been refurbished and re-assembled.  The experience gained from
Number 1 engine paid off in spades.

The wiring harness was refurbished, but would not be installed until after the
engine was in place.  This was a key lesson learned from engine Nr 1.  Some
assemblies were installed too early, and had to be removed again so the engine
was accessible.  This time, the crew assembled the Quick Engine Change  module
from the inside out.  Pipes and hoses would only go on after major parts were
in place.

By March 2011 Nr 2 frame had been refurbished, and work was progressing on
cowlings, radiators and accessories.  In the meantime all four propellers had
been restored, by Hope Aero along with the four spinners. Three were installed
back on the aircraft, while the fourth was stored until all four engines were
completed.

The Nr 2 engine was  re-assembled by April 2011. The crankshaft and pistons had
been installed, along with the cylinder heads and valve train. Work was then
take to restore the ignition wiring and the auxiliary gearbox. 

\begin{figure}[htbp]
   \vspace{2em}
   \centering
   %name of the graphic, without the path AND in EPS format:
   \includegraphics[scale=0.5]{four-merlins-11.png}
   %caption of the figure 
   \caption*{\small \em Austin (Tim) Timmins.}
   %label of the figure, which has to correspond to \ref{}:
   \label{fig:tim}
\end{figure}

Although engine Nr 1 was installed on the North Star in February, 2010, it 
continued to serve as a model for what engine Nr 2 should look like when
completed.  Volunteers made numerous trips to the storage hangar to compare
engine Nr 1 to the items currently then being restored.  This allowed the crew
to solve several re-assembly problems quickly. The engine block was complete
and the crankshaft, cylinder heads and pistons had  been installed by June
2011. Many engine accessories and sub-assemblies had also been treated and were
ready to be reattached. 

The engine frame also received some attention.  The rear cowl ring that held
several of the external panels was completely stripped and repaired before
painting and re-assembly. The three radiators for the engine, intercooler and
oil cooler were dismantled, cleaned and painted.  These three units were
assembled and attached to the front of the engine frame.  The radiators were
then filled with inhibiting oil to prevent corrosion.  There were many
accessories that needed to be refurbished before being installed on the engine
frame.  When completed, most of these items were stored until after the engine
was installed in the frame.  This made assembly much easier.

\begin{figure}[htbp]
   \vspace{2em}
   \centering
   %name of the graphic, without the path AND in EPS format:
   \includegraphics[scale=0.5]{four-merlins-12.png}
   %caption of the figure 
   \caption*{\small \em Jim Riddoch and Bill Tate fill the radiators with
inhibiting oil.}
   %label of the figure, which has to correspond to \ref{}:
   \label{fig:tim}
\end{figure}

The auxiliary gearbox, which is a complex piece of machinery, was reassembled,
but the reduction gear that connects the propeller shaft to the engine proved
to be difficult, due to excessive corrosion.

\begin{figure}[htbp]
   \vspace{2em}
   \centering
   %name of the graphic, without the path AND in EPS format:
   \includegraphics[scale=0.7]{four-merlins-13.png}
   %caption of the figure 
   \caption*{\small \em Engine 2 on rotating assembly stand.}
   %label of the figure, which has to correspond to \ref{}:
   \label{fig:tim}
\end{figure}

The cowl panels were rebuilt, as severe corrosion had affected many of the
steel stiffeners riveted to the edges of the panels.  These had to be removed
and refinished.  Several were so badly gone that new ones were fabricated and
installed.  Once riveted in place, the stiffeners were painted and the outside
of the panels polished to a high shine.  Again, these were stored until the
engine assembly was complete, and then installed once all other assembly work
was finished.

By December 2011 the engine was nearly complete, with only a few sub-assemblies
to be added.  The engine was installed in the engine frame and this allowed
many pipes and hoses and electrical components to be added.  The supercharger
was installed, which was another major milestone for the engine crew.  This
went much quicker than installation on engine  Nr.1.  

\begin{figure}[htbp]
   \vspace{2em}
   \centering
   %name of the graphic, without the path AND in EPS format:
   \includegraphics[scale=0.45]{four-merlins-14.png}
   %caption of the figure 
   \caption*{\small \em  Engine Nr 2 being re-assembled in the engine shop.}
   %label of the figure, which has to correspond to \ref{}:
   \label{fig:tim}
\end{figure}

Engine Nr. 2 was completed by March 2012, with all assemblies and most cowl
panels in place.  A lot of detail work was needed to ensure all pipes, hoses,
clamps and other items were all installed exactly where they were originally. 
Quality control was important to ensure accuracy of the final assembly.  The
auxiliary gearbox was installed on the firewall, so final preparations could be
made  to install engine  Nr. 2 and the propeller, before the aircraft was moved
outside for Canada Day.

It is worth noting that while Engine 1 required more than four years to
complete the restoration, Engine 2 was completed in just 2 years.

Finally, after six years of work by a host of dedicated volunteers, both
restores engines were once again fitted to the North Star by May 2012 as shown
in the photo below.

\begin{figure}[htbp]
   \vspace{2em}
   \centering
   %name of the graphic, without the path AND in EPS format:
   \includegraphics[scale=0.5]{four-merlins-15.png}
   %caption of the figure 
  % \caption*{\small \em .}
   %label of the figure, which has to correspond to \ref{}:
   \label{fig:tim}
\end{figure}

In our next edition we plan to tell the story of the restoration of engines 3
and 4.

\begin{footnotesize}
    \raggedleft PNSAC\\
\end{footnotesize}


% End of text.

%%% Local Variables: 
%%% mode: latex
%%% TeX-master: main_document.tex
%%% End: 

