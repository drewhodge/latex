% Template PNSAC newsletter - Article
% Language: Latex
%

% Head

\title{Our Members}
\author{This is the edited transcript of an interview with Garry Dupont by Richard Lodge.}

\maketitle

\textit{Garry why did you join Project North Star?}

I joined it when I was working full time and it was a discussion between me and
my wife as I knew retirement was looming and it was what I was going to do. I
had seen the article that was in the Ottawa Citizen with Robert Holmgren and Tim
Timmins requesting people to attend a meeting starting the project, and my wife
suggested I go to it, which I did and it gave me something to look forward to.
Even when I was employed I was able to physically contribute by coming in the
odd day and helping out with special projects and overhaul a few components and
stuff like that.\\

\noindent\textit{So you worked only in the engine shop?}

No.  When initially I was working, I worked on various other things and only
after I fully retired, I showed up and said Mike Irvin "I am here. I'll be here
for four days a week".  That's when he asked me to go into the engine shop
because they needed help in there.\\

\noindent\textit{Could you have worked anywhere but you chose to work in the engine shop?}

Actually, Mike chose for me.  Mike was the one who said go to the engine shop.\\

\noindent\textit{When you started with Project North Star would you have felt
that you had most experience of aviation work on engines, or could you gone
anywhere on the plane?}

I could have gone anywhere on the plane. Of this particular engine I had no
knowledge. I had never worked on these liquid cooled engines before.  I never
worked on a piston engine of that size before and actually I hadn't worked on
piston engines for quite a number of years because with my employment most of my
airplanes were with turbines.  We had some small piston engined airplanes of
which I had limited experience.\\

\noindent\textit{So quite a bit of your work in the engine shop was a learning
experience apart from learning about how the Merlin 622 was assembled, you were
also learning a bit about actually working on a large piston engine aircraft?}

That's correct and the systems on this particular engine were completely
different from anything I had ever worked on and I came across  an RCAF training
manual, which somebody had donated and I took it home and read it from one end
to the other over a couple of weeks. I was able to learn a considerable amount
about the engine and understand the systems and it started making sense to me,
so when you start assembling and disassembling and you look at a line you know
what that the line is for. You know what the electrical connections are for. You
know what system they belong to so it makes that part of the process much
easier.\\

\noindent\textit{I know from when we first started working together, you said
you worked for Perimeter Airlines in Winnipeg.  When you were working for
Perimeter, you were working mostly on Metroliners. Correct?}

Yes.  We had 2 Metroliners when I was there, but we also had 4 piston operated
Beechcraft Queenairs, we had 6 other Beechcraft.\\

\noindent\textit{When you worked at Perimeter, were you working on any part of
the plane and you weren't specialising in any particular part, so moving into
solely engines was really a new area of work for you?}

That's correct.\\

\noindent\textit{When you first started working in the engine shop, who were you
working with?}

There was Ted Devey, John Tasseron, Michel Lacasse and two other
fellows, the names escape me.\\

\noindent\textit{I asked a few minutes ago.  If you were only interested in
working on the engines.  Where else would you like to work on the plane? So you worked only in the engine shop?}

Anything actually, I don't do well with sheet metal work, I have done some,
minor bits. I could develop those skills a bit better.  Hydraulic systems; I
have a fairly good interest in, landing gear, the flaps, the hydraulics and all
those systems.  I have learned where the pumps are and how all that works now.
I have also read up little bit on the landing gear to understand it a bit better
and how it works and that would be interesting.\\

\noindent\textit{So you could, if they had been doing the landing gear at the
time you started have been quite interested in working on that side of things?}

Yes.\\

\noindent\textit{What would you say in retrospect was the most difficult part of
the work you were having to do on engines 3 and 4?  Was it lack of paperwork,
was it corrosion or what was it?}

The most difficult part was dealing with corrosion because of the amount of
corrosion in the engine and also the decision-making between my career and what
I would do and what the museum wanted in restoration. The hardest learning part
was learning about restoration and how to do it properly. I have had some good
mentors.  and I have learned fairly well on what is required to do and what to
look forward to in the future when planning the project; how to tackle it from a
restoration perspective rather than a maintenance perspective.\\

\noindent\textit{Was Mike Irvin one of your mentors helping you to know the
various procedures needed for restoration?}

Yes, Mike was the main mentor.  He was the one that took me under his wing and
showed me the requirements and the basis of doing restoration.\\

\noindent\textit{When you started working there, you weren't really in charge of
the engine because it was a combination of you and Ted Devey who was still
working there and who had worked on engines 1 and 2.  So you were working with
him under Mike Irvin?}

Yes.  That's right.\\

\noindent\textit{You feel one of the most difficult parts was to make the shift
from keeping an aircraft flying to restoring and conserving everything?}

That's right, yes.\\

\noindent\textit{Having done that and got into that sort of mindset, what would
you feel was the most difficult part of doing the engine, of restoring the
engines that you had to be involved with, or you had to work with.  Which part,
I am thinking of?}

The biggest part with the engines was the cleaning.  Believe it or not, the
engines and the old oil had been sitting there for 35 years, it was very
congealed and very difficult to remove.  Cleaning with solvents is not a
pleasant thing to do and that was about the only method we could use was to
clean and put it in the parts washer.  Once you put a part in the parts washer
you also had to be very careful because if it was steel, when it was removed
from the part washer it would flash freeze almost immediately, so it had to be
removed and treated straightaway with some sort of preserving oil to prevent the
surface from oxidizing.\\

\noindent\textit{I very much remember John Tasseron working on the pistons and
all work of getting the pistons out of the block  and then John working on the
cleaning the piston rings.  Were you very much involved in that side of things?}

Yes.  Cleaning up the piston rings was a real tedious job where you could spend
a whole week on one piston ring trying to free it up by scraping away the
corrosion, by tapping it to free the rings.  Sadly, a lot of them broke.  There
was nothing we could do to free them up.  We tried various chemicals, soaking
procedures, vibration procedures, putting them in sonic cleaners so that the
vibration would free up the dirt, but nothing seemed to work really well, except
getting in there with a little scraper and trying to scrape them clean.\\

\noindent\textit{And when you did manage to free the parts and get them clean,
you couldn't use a glass bead machine to do any cleaning because the parts were
aluminum so we had to use Scotch Brite and other things to clean them?}

Correct.  The purpose of glass beading is to remove corrosion and not to clean.
If you glass bead a surface, it changes the outer texture of it considerably. In
the engine shop we had at Perimeter, the only time they used glass beading was
for removing the outside surface of the engine block and then they could prepare
it for painting.\\

\noindent\textit{Now that you have finished the main part of the work on engines
3 and 4, what are the main additional parts or work still to be done on engines
3 and 4 and maybe on engines 1 and 2?}

R\'{e}j [Demers] wants to remove the engines so we will reinstall engine 4 when
it is ready and then we will go back and will work on each engine successively
and then work on the firewall and behind the firewall with the engine removed.
At the time when we do this we will inspect the engines to see if corrosion has
come back in some areas and I notice on some of the engines during Canada Day
when we removed parts and cowlings to display the engine that there was
corrosion showing up in certain areas where it had not been treated properly.
So when we remove the engines, and R\'{e}j agrees with me, we will put them back
in the engine shop and go over them thoroughly.  We have also found some issues
with some of the cowl parts just not being adjusted properly.  So in the end, I
would like to have all four engines installed on the wings properly treated so
that we don't have to worry about them again and also the cowls adjusted and
done properly so that at any time we can remove the cowl parts and reinstall
them with ease.  So that in the future, if you want to display the engine again,
you could effectively remove all the cowls on all four engines with ease and
then reinstall them without making any big issue of it.\\

\noindent\textit{Effectively you have got to remove any distortions in the cowl
panels. So those are the things to be done?  When you are finished working on
the Merlin engines, what would you like to do?  This is purely a personal thing
and nothing to do with the Museum.  More engines on a different plane or working
on the airframe?}

It really doesn't matter to me, whatever the Museum sees fit as a project, I am
willing to tackle it. 


% \end{multicols}

%\begin{quotation}
%	\textit{Jim Riddoch is one of the earliest and longest serving members of the
%Association. He has made a significant contribution to the operations of the
%Association including working on the aircraft as well as being an officer and
%member of the board of directors. Although officially retired from formal
%positions in the organization, he can still be counted on to help out when
%needed.}
%\end{quotation}

% \begin{multicols}{2}





%1.\textit{What is your background in aviation?}
%
%
%2. \textit{How long have you been involved with Project North Star and how and
%why did you get involved?}
%
%3. \textit{What has been the history of your involvement to date?}
%
%4. \textit{What has been the highlight of your involvement?}
%
%5. \textit{What has been the most challenging part of your involvement?}

\begin{footnotesize}
    \raggedleft PNSAC\\
\end{footnotesize}



% End of text.

%%% Local Variables: 
%%% mode: latex
%%% TeX-master: main_document.tex
%%% End: 

