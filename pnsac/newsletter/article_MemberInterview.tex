% Template PNSAC newsletter - Article
% Language: Latex
%

% Head

\title{Our Members}
\author{Interview with Garry Dupont}

\maketitle

\end{multicols}

%\begin{quotation}
%	\textit{Jim Riddoch is one of the earliest and longest serving members of the
%Association. He has made a significant contribution to the operations of the
%Association including working on the aircraft as well as being an officer and
%member of the board of directors. Although officially retired from formal
%positions in the organization, he can still be counted on to help out when
%needed.}
%\end{quotation}

\begin{multicols}{2}

1.\textit{What is your background in aviation?}

I graduated from the Confederation College Aircraft Maintenance course
in the Spring of 1975. After returning to my home town in the Red Lake
District I gained employment with Red Lake Seaplane Service. After
completing the course I had to apprentice for 18 months before writing
the final exam for my Aircraft Maintenance Engineers' License. I
successfully wrote my final exam in December 1976 and obtained my AME
certification. In February 1977 I was hired by Perimeter Aviation in
Winnipeg Manitoba. There I obtained my Turbine Engine and
Pressurization Aircraft endorsements.  I started my 30 year career with
the RCMP Air Services Branch in February 1981. My career took me to
Edmonton Alberta, Goose Bay Labrador, Winnipeg Manitoba then finally to
Ottawa. I finished my career as Quality Assurance Manager.

2. \textit{How long have you been involved with Project North Star and how and
why did you get involved?}

I attended the very first meeting held for Project North Star. This was
at my wife's suggestion that it would be a good retirement pastime.

%\begin{figure}[htbp]
%   \vspace{2em}
%   \centering
%   %name of the graphic, without the path AND in EPS format:
%   \includegraphics[scale=0.5]{fwd_cockpit_complete.eps}
%   %caption of the figure 
%   \caption*{\small \em Forward cockpit -- complete.}
%   %label of the figure, which has to correspond to \ref{}:
%   \label{fig:fwd_cockpit_complete}
%\end{figure}

3. \textit{What has been the history of your involvement to date?}

While I was employed I worked an evening shift every 4 weeks so I would
come in to CASM in the morning for a few hours. Upon my retirement Mike
Irving had me work in the engine shop, which I have been doing since my
retirement. I also have been on the board of directors for a number of
years.


%\begin{figure}[htbp]
%	\vspace{2em}
%	\centering
%	%name of the graphic, without the path AND in EPS format:
%	\includegraphics[scale=0.9]{resources/PNS2008-10-26_23-31-42.png}
%	%caption of the figure 
%	\caption*{\small \em Propeller assembly.}
%	%label of the figure, which has to correspond to \ref{}:
%	\label{fig:propeller}
%\end{figure}

%\begin{figure}[htbp]
%	\vspace{2em}
%	\centering
%	%name of the graphic, without the path AND in EPS format:
%	\includegraphics[scale=1.3]{resources/NorthStarCockpitDec2011.png}
%	%caption of the figure 
%	\caption*{\small \em Forward cockpit -- complete.}
%	%label of the figure, which has to correspond to \ref{}:
%	\label{fig:fwd_cockpit_complete}
%\end{figure}

4. \textit{What has been the highlight of your involvement?}

The biggest highlight for me was installing number 2 engine. It was
about half completed when I started in the engine shop. This is the
largest engine I have ever installed in my entire career.

%\begin{figure}[htbp]
%   \vspace{2em}
%   \centering
%   %name of the graphic, without the path AND in EPS format:
%   \includegraphics[scale=0.75]{resources/NavigatorRack.png}
%   %caption of the figure 
%   \caption*{\small \em Navigator rack.}
%   %label of the figure, which has to correspond to \ref{}:
%   \label{fig:nav-radio-rack}
%\end{figure}

5. \textit{What has been the most challenging part of your involvement?}

One of the most challenging parts was learning how the engine worked
with all its systems. I found a RCAF training manual in the office and
took it home. Over several months of reading I started to understand
what made this monster of old technology tick. Another challenge was
learning the British Hardware system and tools. A further challenge was
reading  the old manuals which were written before they were
standardized by the ATA chapter system which all aircraft manuals
follow today. Learning the art of conservation was also a  challenge
but was made easier by the guidance of Mike Irving.

\begin{footnotesize}
    \raggedleft PNSAC\\
\end{footnotesize}



% End of text.

%%% Local Variables: 
%%% mode: latex
%%% TeX-master: main_document.tex
%%% End: 

