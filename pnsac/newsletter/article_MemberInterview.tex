% Template PNSAC newsletter - Article
% Language: Latex
%

% Head

\title{Our Members}
\author{Interview with Richard Lodge}

\maketitle

\end{multicols}

%\begin{quotation}
%	\textit{Jim Riddoch is one of the earliest and longest serving members of the
%Association. He has made a significant contribution to the operations of the
%Association including working on the aircraft as well as being an officer and
%member of the board of directors. Although officially retired from formal
%positions in the organization, he can still be counted on to help out when
%needed.}
%\end{quotation}

\begin{multicols}{2}

1.\textit{What is your background in aviation?}\\

I trained in England as a Chartered Accountant, now often known as a bean
counter. After I obtained my designation, I wanted to work in industry rather
than in the accountancy profession. Following several interviews, I was offered
a job with the Aero Engine Division of Rolls-Royce Ltd. in Derby, England.

Until I joined Rolls-Royce I had very little interest aviation. I had always
been interested in transportation and had a major passion for railways which was
the mode of transport I normally used in Britain in 1950s and 1960s.

I became interested in aviation at Rolls-Royce and was fascinated by the
complexity of designing and building aero engines.

After leaving Rolls-Royce in 1966, I was not involved with aviation until I came
to Canada in 1978 and returned to the accountancy profession and started doing
tax returns for pilots and other aviation personnel located around Dorval,
Qu\`{e}bec.\\

2. \textit{How long have you been involved with Project North Star and how and why did you get involved?}\\

One of the driving forces behind the foundation of Project North Star was
Robert Holmgren who was among a group who persuaded the National Aviation and
Museum, as it was known then, that a group of a volunteers should be allowed to
work on the North Star and bring it back to Museum display standards. The
Project North Star Association of Canada (PNSAC) was incorporated in 2003 with
Robert as its first president. Robert had worked for Air Canada on the
engineering side of the airline in Dorval and I got to know him when doing the
accounting for his wife's antiques business. At the time of the introduction of
the Québec language laws (Bills 22 and 101), many Anglophones left the Montr\'{e}al
area and moved to Ottawa. Both Robert and I moved in the early 1980's. After
transferring my business to Ottawa, I remained in contact with Robert
professionally and when PNSAC was formed he asked me to become the Association's
first Treasurer.\\

%\begin{figure}[htbp]
%   \vspace{2em}
%   \centering
%   %name of the graphic, without the path AND in EPS format:
%   \includegraphics[scale=0.5]{fwd_cockpit_complete.eps}
%   %caption of the figure 
%   \caption*{\small \em Forward cockpit -- complete.}
%   %label of the figure, which has to correspond to \ref{}:
%   \label{fig:fwd_cockpit_complete}
%\end{figure}

3. \textit{What has been the history of your involvement to date?}\\

I remained Treasurer until 2010 when I was elected President after our second
President, Tim Timmins, stood down. My election as President coincided with my
partial retirement from the accountancy profession. Apart from my interest in
railways, I had also always been interested in engineering and retirement gave
me the opportunity to become a Restoration Volunteer on the North Star. Working
on a Merlin engine and becoming part of the Engine Shop crew was an obvious
place for me to work. I have enjoyed developing some of the limited skills I
learned when working on pre-WW2 cars in England, which mostly used similar
technology to the Merlin engines.\\

%\begin{figure}[htbp]
%	\vspace{2em}
%	\centering
%	%name of the graphic, without the path AND in EPS format:
%	\includegraphics[scale=0.9]{resources/PNS2008-10-26_23-31-42.png}
%	%caption of the figure 
%	\caption*{\small \em Propeller assembly.}
%	%label of the figure, which has to correspond to \ref{}:
%	\label{fig:propeller}
%\end{figure}

%\begin{figure}[htbp]
%	\vspace{2em}
%	\centering
%	%name of the graphic, without the path AND in EPS format:
%	\includegraphics[scale=1.3]{resources/NorthStarCockpitDec2011.png}
%	%caption of the figure 
%	\caption*{\small \em Forward cockpit -- complete.}
%	%label of the figure, which has to correspond to \ref{}:
%	\label{fig:fwd_cockpit_complete}
%\end{figure}

4. \textit{What has been the highlight of your involvement?}\\

In the 1950s I was one of the early volunteers working on the first preserved
railway in the world, the Talyllyn Railway in Wales. During this I learned much
about volunteering and how such an organization could work. Helping to organize
and build PNSAC over the last 14 years has both been challenging and very
interesting. We are now seeing the fruits of our work both in the actual
achievements in the restoration of the North Star and the enthusiasm of our
Association members.\\

%\begin{figure}[htbp]
%   \vspace{2em}
%   \centering
%   %name of the graphic, without the path AND in EPS format:
%   \includegraphics[scale=0.75]{resources/NavigatorRack.png}
%   %caption of the figure 
%   \caption*{\small \em Navigator rack.}
%   %label of the figure, which has to correspond to \ref{}:
%   \label{fig:nav-radio-rack}
%\end{figure}

5. \textit{What has been the most challenging part of your involvement?}\\

Volunteers working on an aircraft owned by the Government of Canada in a
national Federal museum was in the early 2000s an entirely new concept. Until
that time all work on the aircraft collection had been carried out by Museum
permanent staff. It was not easy to persuade the Museum that volunteers could be
trusted to carry out restoration work to the high standards of the Museum and
that as a group they would have the tenacity to continue working for many years.
Now 14 years later and after over 80,000 hours of volunteer restoration work,
PNSAC is going strong and will definitely see the end of the North Star
restoration in the years to come.

Starting a new project is not easy. There is always skepticism as to whether the
new organization will succeed and in some cases resistance to the concept of the
new organization. Over the years we have had to overcome all these problems and
avoid being discouraged when we were unable to do things the way we initially
wanted to do them. Our acceptance by the Museum as a trusted partner is a
testament to the patience and hard work of all our volunteers and the support of
key members of the Museum's permanent staff.

\begin{footnotesize}
    \raggedleft PNSAC\\
\end{footnotesize}



% End of text.

%%% Local Variables: 
%%% mode: latex
%%% TeX-master: main_document.tex
%%% End: 

