% Template PNSAC newsletter - Article
% Language: Latex
%

% Head

\title{Our Members}
\author{Interview with Bruce Gemmill}

\maketitle

\end{multicols}

\begin{quotation}
	\textit{Bruce Gemmill has been involved with Project North Star since 2005 and	has exceeded 9,000 volunteer hours. Bruce is presently the Project Manager, as well as a member of the board of directors.}
\end{quotation}

\begin{multicols}{2}

1.\textit{What is your background in aviation?}

Before joining Project North Star, I served in the Canadian Forces for
38 years as a communications and electronics officer.  Although I
never flew as air crew, I worked on air force communications and radar
systems, air traffic control, and NORAD air defence.  I served on
several air force bases, including North Bay and Trenton, and my last
air force position was senior communications adviser to the Commander
of Canadian NORAD Region in Winnipeg.

2. \textit{How long have you been involved with Project North Star and how and why did you get involved?}

After I retired from the Canadian Forces in 2005, I wanted to do
something that was technically challenging and interesting.  I'd done
some woodworking and motorcycle restoration, and one day I was riding
past the Aviation Museum and decided on a whim I would ask if there
was an opportunity to volunteer on an aircraft restoration project.
Six months later I started on Project North Star.

%\begin{figure}[htbp]
%   \vspace{2em}
%   \centering
%   %name of the graphic, without the path AND in EPS format:
%   \includegraphics[scale=0.5]{fwd_cockpit_complete.eps}
%   %caption of the figure 
%   \caption*{\small \em Forward cockpit -- complete.}
%   %label of the figure, which has to correspond to \ref{}:
%   \label{fig:fwd_cockpit_complete}
%\end{figure}

3. \textit{What has been the history of your involvement to date?}

Since I had no specific aircraft maintenance training or experience, I
began working on electronic assemblies that had been removed from the
aircraft cockpit during the early days of the project.  With no test
equipment, manuals or spare parts, the work was mainly cleaning and
repainting the cases and front panels.  As I got more familiar with
the equipment, I learned how to spray paint, and did minor repairs to
wiring.  I also worked on electrical components from the Merlin
engine, such as the magnetos and starter motor. 

\begin{figure}[htbp]
	\vspace{2em}
	\centering
	%name of the graphic, without the path AND in EPS format:
	\includegraphics[scale=0.9]{resources/PNS2008-10-26_23-31-42.png}
	%caption of the figure 
	\caption*{\small \em Propeller assembly.}
	%label of the figure, which has to correspond to \ref{}:
	\label{fig:propeller}
\end{figure}

The most demanding job was removing several of the propellers so they could be shipped for restoration.  Years of corrosion had sealed the spinners onto the propeller hubs.  Removal required the use of dental picks and a large sledge hammer!


After helping with restoration work on Engine One, I began working on
the cockpit as a major project.  I spent approximately five years
working with other volunteers, beginning with removal of as much
equipment as possible from the cockpit and crew lounge, including the
control column, rudder pedals and center control console, seats and
headliners.  A lot of time was spent cleaning and repairing corrosion
damage before painting the cockpit and installing new insulation.  I
found it very rewarding when we began installing previously restored
equipment to the "new" cockpit, but also very challenging. 

\begin{figure}[htbp]
	\vspace{2em}
	\centering
	%name of the graphic, without the path AND in EPS format:
	\includegraphics[scale=1.3]{resources/NorthStarCockpitDec2011.png}
	%caption of the figure 
	\caption*{\small \em Forward cockpit -- complete.}
	%label of the figure, which has to correspond to \ref{}:
	\label{fig:fwd_cockpit_complete}
\end{figure}

 A lot of equipment and fittings had been removed prior to me joining the project, and not all removal work had been documented.  Detective work was required to search out photographs that showed where some equipment was installed.

4. \textit{What has been the highlight of your involvement?}

Completion of the cockpit and crew lounge has certainly been the most
significant achievement for me personally. The project team has
received high praise from numerous visitors who have commented on the
high quality of the restoration work.  Although we consider the work
done, I do still want to complete the restoration by finding a few
missing radios and instruments that were removed by the RCAF.

\begin{figure}[htbp]
   \vspace{2em}
   \centering
   %name of the graphic, without the path AND in EPS format:
   \includegraphics[scale=0.75]{resources/NavigatorRack.png}
   %caption of the figure 
   \caption*{\small \em Navigator rack.}
   %label of the figure, which has to correspond to \ref{}:
   \label{fig:nav-radio-rack}
\end{figure}

5. \textit{What has been the most challenging part of your involvement?}

Finding spare and missing parts for the aircraft has always been a
challenge.  Our ability to search out possible sources is limited, as
are funds.  We have been lucky to get most parts needed for the Rolls
Royce engines, but many items for the airframe are not easy to come
by.  These include everything from very large and expensive items,
such as the de-icing boots for the wings and tail, to door and window
seals.  Rubber and fabric items are particularly hard to find, since
these items deteriorate with age.  Finding the correct paint has
recently become a major challenge because of stricter environmental
laws.  Our goal of achieving an authentic restoration of the North
Star will only become more difficult with the passage of time.


\begin{footnotesize}
    \raggedleft PNSAC\\
\end{footnotesize}



% End of text.

%%% Local Variables: 
%%% mode: latex
%%% TeX-master: main_document.tex
%%% End: 

