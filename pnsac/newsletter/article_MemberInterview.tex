% Template PNSAC newsletter - Article
% Language: Latex
%

% Head

\title{Our Members}
\author{Interview with Jm Riddoch}

\maketitle

\end{multicols}

\begin{quotation}
	\textit{Jim Riddoch is one of the earliest and longest serving members of the
Association. He has made a significant contribution to the operations of the
Association including working on the aircraft as well as being an officer and
member of the board of directors. Although officially retired from formal
positions in the organization, he can still be counted on to help out when
needed.}
\end{quotation}

\begin{multicols}{2}

1.\textit{What is your background in aviation?}

I started with English Electric Aviation Company in 1956 as an
apprentice technician eventually being assigned to the Mechanical
Engineering Department as a fuel systems technician. I remained there
for 5 years until I was 28 years old and married with a daughter. After
immigrating to Canada in March 1966 I joined DeHavilland Spar Division
in Malton and then moved to Montreal with Jarry Hydraulics as a
development engineer. I remained there for 2 years working chiefly on
development proposals for the wing sweep actuators for F111 aircraft
and Boeing's proposed supersonic Concorde.

In 1968 I joined Air Canada as a junior engineer in the Mechanical
Systems Department and assigned as a landing gear technologist until I
achieved Professional Engineer status in 1974. I then moved around
several other departments in engineering and maintenance until I was
promoted to Superintendent of DC9 maintenance. In that capacity I was
involved in the accident investigation into the loss of one aircraft at
Cincinnati.

I eventually returned to engineering as Director of Interior Systems and
Equipment to supervise extensive fleet modifications and new aircraft
acquisitions. I remained in that position until I retired in 1990.
Following a short stint with First Air I was asked to join a newly formed
Canadian Aviation Maintenance Council to standardize aircraft
maintenance trades basic training. I was hired as the Accreditation
Manager to approve various training establishments including schools,
companies and military programs, I eventually also took on the job of
Registration Manager to accept qualified trade technicians.

2. \textit{How long have you been involved with Project North Star and how and why did you get involved?}

Following my retirement from the Canadian Aviation Maintenance
Council in 2003 I was approached by Robert Holmgren to participate in
a voluntary program to help restore an the Canadair North Star aircraft
parked outside at the Canada Aviation Museum. Along with Tim
Timmins and a few others we formed a steering committee to approach
the Museum about forming a voluntary work force to assist the museum
staff to restore the aircraft. At first we met with opposition from some
staff members who felt their jobs were being compromised and possible
staff reductions. Slowly we were approved by the Museum staff with
limited supervision and controlled access and work direction. I was
assigned by the Project North Star Association as Chief Engineer of the
project although all work was controlled and supervised by the
Museum's staff manager.

%\begin{figure}[htbp]
%   \vspace{2em}
%   \centering
%   %name of the graphic, without the path AND in EPS format:
%   \includegraphics[scale=0.5]{fwd_cockpit_complete.eps}
%   %caption of the figure 
%   \caption*{\small \em Forward cockpit -- complete.}
%   %label of the figure, which has to correspond to \ref{}:
%   \label{fig:fwd_cockpit_complete}
%\end{figure}

3. \textit{What has been the history of your involvement to date?}

I worked on the Project for 10 years starting in 2003 both as a “hands
on” volunteer as well as a member of the board of directors and officer
of the Association. Generally I put in at least two days per week and
sometimes more. 

%\begin{figure}[htbp]
%	\vspace{2em}
%	\centering
%	%name of the graphic, without the path AND in EPS format:
%	\includegraphics[scale=0.9]{resources/PNS2008-10-26_23-31-42.png}
%	%caption of the figure 
%	\caption*{\small \em Propeller assembly.}
%	%label of the figure, which has to correspond to \ref{}:
%	\label{fig:propeller}
%\end{figure}

%\begin{figure}[htbp]
%	\vspace{2em}
%	\centering
%	%name of the graphic, without the path AND in EPS format:
%	\includegraphics[scale=1.3]{resources/NorthStarCockpitDec2011.png}
%	%caption of the figure 
%	\caption*{\small \em Forward cockpit -- complete.}
%	%label of the figure, which has to correspond to \ref{}:
%	\label{fig:fwd_cockpit_complete}
%\end{figure}

4. \textit{What has been the highlight of your involvement?}

The chief highlights have been the high level of workmanship of the
volunteers and displaying the aircraft to the public and interested
persons.

%\begin{figure}[htbp]
%   \vspace{2em}
%   \centering
%   %name of the graphic, without the path AND in EPS format:
%   \includegraphics[scale=0.75]{resources/NavigatorRack.png}
%   %caption of the figure 
%   \caption*{\small \em Navigator rack.}
%   %label of the figure, which has to correspond to \ref{}:
%   \label{fig:nav-radio-rack}
%\end{figure}

5. \textit{What has been the most challenging part of your involvement?}

Keeping the Museum staff and management on side with our goals
and objectives and recruiting enough new volunteers to keep the
project going.

\begin{footnotesize}
    \raggedleft PNSAC\\
\end{footnotesize}



% End of text.

%%% Local Variables: 
%%% mode: latex
%%% TeX-master: main_document.tex
%%% End: 

