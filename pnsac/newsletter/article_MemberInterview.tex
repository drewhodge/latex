% Template PNSAC newsletter - Article
% Language: Latex
%

% Head

\title{Our Members}
\author{Interview with Ted Devey}

\maketitle

\end{multicols}

%\begin{quotation}
%	\textit{Jim Riddoch is one of the earliest and longest serving members of the
%Association. He has made a significant contribution to the operations of the
%Association including working on the aircraft as well as being an officer and
%member of the board of directors. Although officially retired from formal
%positions in the organization, he can still be counted on to help out when
%needed.}
%\end{quotation}

\begin{multicols}{2}

1.\textit{What is your background in aviation?}

My service background was naval. Professionally I was an electrical engineer
with the Department of Communications specializing in radio spectrum
engineering. After retirement in 1989 I sought activities that interested me
such as helping to install a pipe organ in a theatre in Renfrew, working on
restoring a sailing yacht, which unfortunately stopped due to lack of money. I
also took courses in welding, machine shop and woodworking at Algonquin College
to enhance my manual skills. 

2. \textit{How long have you been involved with Project North Star and how and
why did you get involved?}

I attended a public meeting in the Museum in 2003 in which it was indicated
that efforts were being made to restore the Museum's North Star to a showroom
condition. In 2004 the project finally was approved and work started.
Professional work had to be carried out to remove asbestos and many dead birds
and their mess. At first I used my welding skills in adapting steps/platforms
for working on engines on the aircraft. In winter, seats were removed from the
cockpit and brought into the shop where they were dismantled, treated and
cleaned, painted and finally spray painted. I found the work interesting and
satisfying and so I was hooked into the project.

%\begin{figure}[htbp]
%   \vspace{2em}
%   \centering
%   %name of the graphic, without the path AND in EPS format:
%   \includegraphics[scale=0.5]{fwd_cockpit_complete.eps}
%   %caption of the figure 
%   \caption*{\small \em Forward cockpit -- complete.}
%   %label of the figure, which has to correspond to \ref{}:
%   \label{fig:fwd_cockpit_complete}
%\end{figure}

3. \textit{What has been the history of your involvement to date?}

In 2006 I expressed interest in the work on the four Roll-Royce Merlin engines.
As it turned out I was asked if I would take on the job of Merlin Crew Chief
and agreed with the proviso that I would turn the task over if someone came
along who was properly qualified. When Garry Dupont retired from the RCMP as
head AME at the RCMP Hangar at Uplands, I offered the position to him which he
took over. From 2006 until the summer 2018 I did only engine work on the four
engines. Unfortunately my limited mobility prevented me from helping with the
more recent work that was being done behind the firewall of \#4 engine. So I
went into the Conservation shop to start doing different work.


%\begin{figure}[htbp]
%	\vspace{2em}
%	\centering
%	%name of the graphic, without the path AND in EPS format:
%	\includegraphics[scale=0.9]{resources/PNS2008-10-26_23-31-42.png}
%	%caption of the figure 
%	\caption*{\small \em Propeller assembly.}
%	%label of the figure, which has to correspond to \ref{}:
%	\label{fig:propeller}
%\end{figure}

%\begin{figure}[htbp]
%	\vspace{2em}
%	\centering
%	%name of the graphic, without the path AND in EPS format:
%	\includegraphics[scale=1.3]{resources/NorthStarCockpitDec2011.png}
%	%caption of the figure 
%	\caption*{\small \em Forward cockpit -- complete.}
%	%label of the figure, which has to correspond to \ref{}:
%	\label{fig:fwd_cockpit_complete}
%\end{figure}

4. \textit{What has been the highlight of your involvement?}

I think that the highlight of my involvement was back in 2006. To work on the
engines we needed two engine stands, one to transfer the entire engine assembly
from the aircraft for dismantling to the core engine and a second stand to
mount the core engine and further dismantle it to bare components. This had to
be a rotary affair so that the engine could be positioned for whatever
operation was being done. The first stand had to be designed and built from
scratch. The design was an adaptation of the engine stand located next to the
Lancaster bomber on the Museum floor with ample room for a person to work
underneath. A specification was drawn up for the various pieces of steel to be
cut precisely by the steel supplier and to be delivered to the RCMP Hangar at
Uplands, as instructed by Garry Dupont. On a Saturday a group of volunteers
assembled at the RCMP Hangar and assembled the horizontal and vertical sections
of the stand with Garry doing the welding. Following this the stand was
delivered to the Museum's welding shop for final assembly, then into the main
shop where it was finally assembled and painted. Everything fit as intended.

The rotary stand was located in the storage area of the Science and Tech
Museum. The base was extended to the required length and a new frame was built
to hold the core engine. That everything fit together so well from the start
was the highlight of the project for me.

%\begin{figure}[htbp]
%   \vspace{2em}
%   \centering
%   %name of the graphic, without the path AND in EPS format:
%   \includegraphics[scale=0.75]{resources/NavigatorRack.png}
%   %caption of the figure 
%   \caption*{\small \em Navigator rack.}
%   %label of the figure, which has to correspond to \ref{}:
%   \label{fig:nav-radio-rack}
%\end{figure}

5. \textit{What has been the most challenging part of your involvement?}

The most challenging part of the job was working on engine \#1, the port outer.
Since I was not a mechanic, I had to plan the disassembly of the engine and
carry it out with the help of crew members under the strict guidance of the
then Shop Supervisor Mike Irvin. He was a hard taskmaster. This was not an
engine that had just been removed from an operating aircraft for overhaul. The
North Star and its engines had been left out in the Great Canadian outdoors for
39 years without any attempt at protection from the four seasons of weather
during those years. Corrosion was rampant. In each of the twelve cylinders the
piston rings were rusted to the cylinder liners. Many hours of soaking the
pistons with Liquid Wrench, the use of compressed air and other techniques had
to be employed to remove the pistons from the cylinders. There were other
issues too that made the overall disassembly of the engine very difficult. The
foregoing applied to each of the other three engines, so it has taken about 12
years to process the four Merlins to a preserved non-running condition. The
first engine I found the most challenging as we had to learn as we went along.
The remaining engines were very difficult but the experience gained with \#1
engine was valuable in restoring the others.

\begin{footnotesize}
    \raggedleft PNSAC\\
\end{footnotesize}



% End of text.

%%% Local Variables: 
%%% mode: latex
%%% TeX-master: main_document.tex
%%% End: 

