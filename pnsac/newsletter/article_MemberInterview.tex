% Template PNSAC newsletter - Article
% Language: Latex
%

% Head

\title{Our Members}
\author{Richard Lodge.}

\maketitle

Last New Year's Eve, December 2021, I became the third past president of PNSAC, after 12 years in the position and following several years at treasurer. I was delighted when Chris McGuffin agreed take on the role of president following the unanimous support of the board of directors.

Before starting to write this article, I made a list of some of the more important and memorable events of my presidency. This article is my recollection of the things that have stuck in my mind over the years. During my term in office, PNSAC changed from being an experimental volunteer organization which still had to prove itself to the Canadian Aviation and Space Museum (CASM), to being part of the CASM family where PNSAC members are well respected and trusted by the museum staff and management.

Several events stand out when I think about my years as president. The first was on Canada Day 2017 shortly after R\'{e}j Demers had arrived at CASM to take over the position of Special Project Manager. The ex RCAF Hercules C130 had arrived in April 2016 and was parked on the apron outside the Storage Hangar.  A large crowd gathered around the aircraft when the day was at its it hottest.  Standing in the middle of the cargo ramp was R\'{e}j where for half an hour or so he entertained the crowd.  The event was unplanned, and his comments were unscripted. The crowd just stood there enjoying the informality of listening to what he was saying and answering questions shouted to him.

The second event was in December 2019 when Engine \#4 was moved from the Engine Shop to be located in front of the North Star in the Storage Hangar. It was a cold early winter day with snow flurries. On this occasion the engine could only be moved on its stand which had four wheels and no steering. It was towed by R\'{e}j driving CASM's forklift from the staff parking lot along the side of the building and into the Storage Hanger. It was a very slow process taking something like half an hour. By the time we got the engine safely to bed the three engine shop volunteers (me, Garry Dupont, and Charles Baril) as well as other volunteers who were walking behind the engine were really not enjoying the beginning of a Canadian winter. R\'{e}j who was sitting in the heated cab of the forklift, with a nice smile on his face, was actually quite warm! 

\begin{figure}[H]
   \vspace{2em}
   \centering
   %name of the graphic, without the path AND in EPS format:
   \includegraphics[scale=0.75]{engin_4.png}
   %caption of the figure 
   \caption*{\small \em From left to right: Garry Dupont, R\'{e}j Demers (in the cab), and Chris McGuffin.}
   %label of the figure, which has to correspond to \ref{}:
   \label{fig:img1}
\end{figure}


The next event was when I was MC at the showing of the film  made in 2014 when the  Lancaster bomber, part of the collection of the Canadian Warplane Heritage Museum, flew to England to join the only other airworthy Lancaster in the world for a summer tour of the airshows in Britain. The epic flight was filmed throughout its three-day journey to England, during its time in England and on its return.  Bill Tate, one of our very active members, arranged for the film to be shown in the auditorium of the CASM to about 250 members of the general public.  The film itself was very interesting, but what made it even more interesting was that the flight crew agreed to come to Ottawa and talk to the audience after the showing of the film.  Hearing the flight crew talk of the preparations and difficulties of the tour and knowing how successful it had been was particularly interesting to me.  I had managed to attend the first airshow in England when the two Lancasters, a Hurricane and two Spitfires had flown together off the south coast of England, near where they would regularly pass on bombing missions during World War 2. The sound of 11 Merlin engines flying together reminded me of my boyhood in England when I would hear the bombers leaving their bases in Yorkshire to bomb Germany – quite an emotional sound all these years later.

There are always difficulties and problems to deal with when one is the head of any organization, be it in a paid position or as a volunteer.  PNSAC has experienced a  number of challenges during it existence.  We have had two major closures and a lock down and closure caused by the Covid 19 pandemic.  The Covid 19 closure needs no explaining but the previous two were necessitated by CASM staging major events in 2013 and 2016 around the Storage Hangar.  While the closures were disruptive to the Project North Star operations, the presentation of the Star Wars and Star Trek exhibits allowed CASM to reach a broader audience and raise awareness of CASM. The exhibits made it impossible for the restoration team to work on the North Star aircraft for many weeks in both years.  The challenge was to try to keep the restoration volunteers together when some volunteers were finding other ways to occupy their time.  Many meetings and discussions took place, and another side of my role was to help the restoration volunteers understand that these closures were only temporary and that as soon as the spectacle events were finished, restoration work would start on the aircraft again.  Fortunately, we did not lose many restoration volunteers although our actual non-volunteer membership dropped considerably.

During the summer of 2018 it was decided that the vertical fin of the North Star should be removed.  This involved weeks of preparatory work inside the aircraft, loosening the many nuts and bolts securing the fin to the aircraft.  On the appointed day a large mobile crane arrived to remove the vertical fin. The rental cost was financed by PNSAC donations received from members.  It was a hot day, and the crane was duly stationed beside the aircraft.  However, several of the securing bolts had been left in place and needed to be removed once the crane had been attached to the fin.  Several volunteers assisted in the attaching of the crane hook to the fin, but one volunteer, Charles Baril, offered to crawl inside the plane and remove the last remaining bolts.  The inside of the plane was like an oven with no ventilation and no proper lighting. The only way Charles could communicate with the crew outside was by shouting.  After about an hour of very difficult work, Charles shouted that all the nuts and bolts had been undone and the actual lifting of the fin could proceed.  After the fin had been lifted by the crane, it was lowered into a specially prepared cradle. Charles emerged from the aircraft to the applause of all people outside, looking as if he was about to melt and thoroughly tired out by his exertions.

\begin{figure}[H]
   \vspace{2em}
   \centering
   %name of the graphic, without the path AND in EPS format:
   \includegraphics[scale=0.5]{nstar_tail_fin.png}
   %caption of the figure 
   \caption*{\small \em The North Star with its tail fin removed.}
   %label of the figure, which has to correspond to \ref{}:
   \label{fig:img1}
\end{figure}

 As president, I was involved in several other memorable events organized by Bill Tate.  Bill has an amazing ability to organize events and had a substantial list of contacts.  I was primarily involved as the MC of the trips while in the buses, also providing a backup for Bill should he encounter any problems.  Twice we visited RCAF Trenton and once we went to Montréal and Mirabel to see the Canadair/Bombardier production line in action. The most memorable was the weekend bus trip to Hamilton to attend the airshow organized by the Canadian Warplane Heritage Museum on the Saturday and on the Sunday, a visit to the Air Canada control centre in Toronto. On the Friday evening, Bill organized a visit to the vineyards in Niagara-on-the-Lake. Sunday morning saw us all back in the bus heading for downtown Toronto, where we breakfasted in the brewery district before our Air Canada visit.   Many members will have great memories of these times, which all took place prior to the closures I referred to earlier in this article. 
 
\end{multicols}

\begin{figure}[H]
   \vspace{2em}
   \centering
   %name of the graphic, without the path AND in EPS format:
   \includegraphics[scale=0.4]{july2012.png}
   %caption of the figure 
   \caption*{\small \em Hamilton trip.}
   %label of the figure, which has to correspond to \ref{}:
   \label{fig:img1}
\end{figure}

\begin{multicols}{2} 

One of the best aspects of assuming responsibility as a volunteer is that the more one puts into it, the more pleasure and interest one gets from the work.  As president, I regularly met all the senior management of CASM and many other people outside CASM.  Often, I was involved in discussions about the future direction of PNSAC, which of course involved discussing future and sometimes confidential CASM plans and operations.  

A very small incident, probably four or five years ago, really demonstrated to me how as a volunteer association we had "arrived".  My predecessor presidents, Bob Holmgren and Tim Timmins had to overcome the skepticism of CASM management as to whether it would be possible for us to sustain a continuing volunteer effort.  By the time I became president in 2010, CASM management had accepted that our members would work to a very high standard of restoration work and would continue to put in several thousand hours of painstaking and sometimes difficult work every year.  The incident I am remembering best is a particular day when I went upstairs to the office level  at CASM and saw a group of CASM staff talking in the corridor.  I stopped and said hello to them and then started to excuse myself by saying that they were having a discussion and I would get out of the way.  Chris Kitzan, the CASM Director General, looked at me and said "There was no need to go away Richard, you are part of the family".  This for me was a watershed moment and one I shall always appreciate.

Looking towards the future, as the pandemic begins to be contained, we must get the restoration operation going again and try to recruit more members and restoration volunteers.  I feel very confident that Chris McGuffin and the other directors of PNSAC will be able to do this and continue to help CASM in volunteer work in addition to North Star restoration.  I look forward to continuing volunteering, but definitely not climbing around the top of the aircraft in a full safety harness!  




% \end{multicols}

%\begin{quotation}
%	\textit{Jim Riddoch is one of the earliest and longest serving members of the
%Association. He has made a significant contribution to the operations of the
%Association including working on the aircraft as well as being an officer and
%member of the board of directors. Although officially retired from formal
%positions in the organization, he can still be counted on to help out when
%needed.}
%\end{quotation}

% \begin{multicols}{2}





%1.\textit{What is your background in aviation?}
%
%
%2. \textit{How long have you been involved with Project North Star and how and
%why did you get involved?}
%
%3. \textit{What has been the history of your involvement to date?}
%
%4. \textit{What has been the highlight of your involvement?}
%
%5. \textit{What has been the most challenging part of your involvement?}

\begin{footnotesize}
    \raggedleft PNSAC\\
\end{footnotesize}



% End of text.

%%% Local Variables: 
%%% mode: latex
%%% TeX-master: main_document.tex
%%% End: 

