% Template PNSAC newsletter - Article
% Language: Latex
%

% Head

\title{Our Members}
\author{Interview with Austin J Timmins}

\maketitle

\end{multicols}

%\begin{quotation}
%	\textit{Jim Riddoch is one of the earliest and longest serving members of the
%Association. He has made a significant contribution to the operations of the
%Association including working on the aircraft as well as being an officer and
%member of the board of directors. Although officially retired from formal
%positions in the organization, he can still be counted on to help out when
%needed.}
%\end{quotation}

\begin{multicols}{2}

1.\textit{What is your background in aviation?}

My aviation experience includes 36 years service in the RCAF and nine
years with the Canadian Aeronautics and Space Institute. In the RCAF I
completed radio navigator training, which included communications,
navigation and air gunner. I later qualified as an airborne
interception navigator . My operational experience includes air
transport, search and rescue and training. My assignments included
Navigation Leader, Aircrew Leader, Squadron Commander, Air Command
Navigator and Deputy Commander (COO) of the Air Transport Group. I
logged 9500 hours , mostly on the following aircraft: Beechcraft,
Dakota, North Star , Mitchell, Yukon, Hercules, and Boeing 707. My
staff assignments in Europe included movement control (traffic
management) for 1 Air Division and a tour at The NATO Defense College
in Rome. 

On retirement from the RCAF I participated in a Transport
Canada A Base review of several airports. I then secured full
employment with Canadian Aeronautics and Space Institute as Executive
Director, retiring in 1995.

2. \textit{How long have you been involved with Project North Star and how and why did you get involved?}

My involvement with efforts to conserve North Star 17515 predates
Project North Star. Ottawa area members of the 426 Thunderbird Squadron
Association, mostly retired North Star aircrew, met with the National
Aviation Museum staff to express their concern about the state of the
North Star and proposed volunteer involvement in a preservation effort.
These expressions of concern and interest led to a meeting between the
Association representatives, where General Adamson presented a proposal
for the restoration of the aircraft. It was rejected but assurance was
given that the North Star would be first priority when the new hangar
was completed Also a North Star Trust Fund would be established. Robert
Holmgren, an observer at the meeting, then proceeded to develop a
proposal for volunteers to restore the North Star.
%\begin{figure}[htbp]
%   \vspace{2em}
%   \centering
%   %name of the graphic, without the path AND in EPS format:
%   \includegraphics[scale=0.5]{fwd_cockpit_complete.eps}
%   %caption of the figure 
%   \caption*{\small \em Forward cockpit -- complete.}
%   %label of the figure, which has to correspond to \ref{}:
%   \label{fig:fwd_cockpit_complete}
%\end{figure}

3. \textit{What has been the history of your involvement to date?}

I have been involved from the beginning of efforts to establish PNS,
first as a member of a Steering Committee, led by Robert Holmgren,
later, Vice President and then President of PNSAC. My contacts in
aviation provided much needed advice and support, including access to
initial funding for PNSAC.


%\begin{figure}[htbp]
%	\vspace{2em}
%	\centering
%	%name of the graphic, without the path AND in EPS format:
%	\includegraphics[scale=0.9]{resources/PNS2008-10-26_23-31-42.png}
%	%caption of the figure 
%	\caption*{\small \em Propeller assembly.}
%	%label of the figure, which has to correspond to \ref{}:
%	\label{fig:propeller}
%\end{figure}

%\begin{figure}[htbp]
%	\vspace{2em}
%	\centering
%	%name of the graphic, without the path AND in EPS format:
%	\includegraphics[scale=1.3]{resources/NorthStarCockpitDec2011.png}
%	%caption of the figure 
%	\caption*{\small \em Forward cockpit -- complete.}
%	%label of the figure, which has to correspond to \ref{}:
%	\label{fig:fwd_cockpit_complete}
%\end{figure}

4. \textit{What has been the highlight of your involvement?}

The highlight of my involvement has to be the establishment of PNS and
PNSAC.

%\begin{figure}[htbp]
%   \vspace{2em}
%   \centering
%   %name of the graphic, without the path AND in EPS format:
%   \includegraphics[scale=0.75]{resources/NavigatorRack.png}
%   %caption of the figure 
%   \caption*{\small \em Navigator rack.}
%   %label of the figure, which has to correspond to \ref{}:
%   \label{fig:nav-radio-rack}
%\end{figure}

5. \textit{What has been the most challenging part of your involvement?}

PNS is now in its 16th year. It has been a work in progress with many
bumps along the way. I spent a lot of time and effort, along with other
members of the Executive, evening out these bumps. Establishing and
maintaining effective channels of communication with CASM and CSTM
required constant attention. Much effort was directed to fund raising,
publicity, and, finding volunteers. In the beginning, every date
brought a new challenge.

\begin{footnotesize}
    \raggedleft PNSAC\\
\end{footnotesize}



% End of text.

%%% Local Variables: 
%%% mode: latex
%%% TeX-master: main_document.tex
%%% End: 

