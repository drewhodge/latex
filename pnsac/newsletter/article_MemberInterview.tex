% Template PNSAC newsletter - Article
% Language: Latex
%

% Head

\title{Our Members}
% \author{Interview with Garry Dupont}

\maketitle

% \end{multicols}

%\begin{quotation}
%	\textit{Jim Riddoch is one of the earliest and longest serving members of the
%Association. He has made a significant contribution to the operations of the
%Association including working on the aircraft as well as being an officer and
%member of the board of directors. Although officially retired from formal
%positions in the organization, he can still be counted on to help out when
%needed.}
%\end{quotation}

% \begin{multicols}{2}

Readers of the Chronicle will know that this item usually refers to one of our
members. This edition, however, salutes all of our members who worked as
volunteers on the restoration of the North Star engines namely:

\begin{itemize}
  \item Charles Baril
  \item John Corby (deceased)
  \item Ted Devey (deceased)
  \item Ed Hogan
  \item Peter Houston
  \item Ron Lemieux
  \item Richard Lodge
  \item Wib Neal
  \item Jim Riddoch
  \item Stan Rideout
  \item Ted Slack
  \item John Tasseron
  \item Bill Tate
  \item Tim Timmins (deceased)
  \item Giuseppe Zanetti
\end{itemize}

The following members have provided comments on their personal experiences.

\textbf{Charles Baril}

Working on all 4 Merlin engines has been an extremely rewarding experience that
has taught me a lot about older technology, an appreciation for history
involved (and those in the engine crew with that knowledge) and learning new
skills and techniques. Although most of the parts were auxiliary systems,
seeing them as part of the full engine gives a sense of accomplishment. It is
very satisfying to see a part going from the dirty, oil covered, rusty part to
newly refurbished one that looked like it was overhauled in an RCAF maintenance
depot. It has been an honour to be part of this project and restoration of
these pieces of history.

In particular I remember working on the lower cowling flaps.  Trying to undo
the inner skin was difficult due to longer rivets through the mounts for the
Flaps. It took several tries and a few extra holes, but it was done. Fixing
those holes often meant using original methods which was revealing and a
commitment to historical methods.

I also recall the work on the filter ducts.  The filter duct is the last stage
of the air intake before the supercharger. During its cleaning, there was a
starling nest on top of one filter. This caused a lot of corrosion damage that
needed to be removed and treated. Yet, what got me was that starlings could get
that far into the engine.

\textbf{Richard Lodge}

As probably the only volunteer who ever worked in the Aero Engine Division of
Rolls-Royce in Derby, England, I was greatly interested in becoming involved in
Project North Star, even though, as an accountant, I had never worked on the
design or building of an aeroengine.  By the time I started work at Rolls-Royce
in 1963, new build of Merlin engines had ceased, and only spare parts were
being manufactured.

I have several strong memories of working on the Merlin engine.  In general,  I
particularly remember John Tasseron working endlessly on freeing seized piston
rings.  I learned much from John about patiently working at something which
appeared to be impossible! Later, when working with  R\'{e}j Demers and Garry
Dupont, I started to learn how to rivet and how to lock wire. The most
important lesson for me has been, photograph, photograph, photograph, at each
stage of dismantling, cleaning and reinstalling an engine part. When sitting in
an overcrowded office at Rolls-Royce, it never occurred to me that at some time
in the future I would actually be working on a Merlin engine or working my way
through an original parts manual to work out how to reassemble an item.  Life
is certainly full of interesting surprises.

\textbf{John Tasseron}

The list below is intended to give a brief overview of the various highlights
that arose during my participation in the restoration of the second, third and
fourth Merlin engines:

\begin{enumerate}
  \item It's not every day one gets the opportunity to see up close what makes
the Rolls-Royce 620 Series Merlin engine tick!
  \item It's a real challenge to safely lower each "power egg" down off the
wing, place it in "the big blue stand" and move it in to the workshop.
  \item Then starts the process of carefully separating about 11,000 individual
parts before cleaning, inspecting and re-conditioning them as necessary. Do not
lose any and make sure everything is photographed for museum records!
  \item When things unexpectedly go wrong, the real expertise kicks in to
perhaps develop a repair scheme or even manufacture a brand-new replacement
part. The objective is to bring everything back together again and end up with
a "like new" engine, ready for preservation.
  \item It all takes team effort based on effective communication and
co-operation. There are no dumb questions and the answers are all aimed at the
challenge to create an "airworthy" product that will never fly.
  \item Finally, its important and fun to share the project with a curious
public, so be ready to answer any questions that might be asked about the
aircraft and its engines!
\end{enumerate}



%1.\textit{What is your background in aviation?}
%
%
%2. \textit{How long have you been involved with Project North Star and how and
%why did you get involved?}
%
%3. \textit{What has been the history of your involvement to date?}
%
%4. \textit{What has been the highlight of your involvement?}
%
%5. \textit{What has been the most challenging part of your involvement?}

\begin{footnotesize}
    \raggedleft PNSAC\\
\end{footnotesize}



% End of text.

%%% Local Variables: 
%%% mode: latex
%%% TeX-master: main_document.tex
%%% End: 

