% Template PNSAC newsletter - Article
% Language: Latex
%

% Head

\title{A Personal Journey in Support of Homeless Veterans in Canada}
\author{Neil Raynor}

\maketitle

Veterans' House: the Andy Carswell Building is Canada's first "Housing
First" community built for homeless Veterans. This pioneering project
specifically targets the needs of the rising number of Veterans who are
"living rough" in Canada and combines safe housing with essential
on-site rehabilitation and wrap around support services.


Who knew? How could we? Surely if we knew we'd have done something
about it?

Fact: every night in Ottawa between 50 and 85 Veterans are sleeping on
the streets, in the parks or under bridges in Ottawa. How could that be
happening in the 21st Century in a G7 capital?

Seven years ago a group of people came together within the Multifaith
Housing Initiative (MHI) to do something about that stat. MHI is a
federally registered charity based in Ottawa which provides affordable
housing: currently it has about 140 units with 400 residents scattered
across the city. What started as a desire to provide affordable housing
has morphed into an ethos to build communities---Veterans' House is the
latest iteration of that approach.

What MHI and its volunteer members did was: articulate the need to
address the issue of homelessness among Veterans; identify the need for
support for these Veterans that went beyond just housing; and together
with its partners develop a solution that wraps it all within a
military community initiative.

The physical manifestation of MHI's work is the structure being raised
on the former CFB Rockcliffe airbase called Veterans' House: the Andy
Carswell Building, a 40-unit community apartment building named for a
retired WWII RCAF Lancaster pilot who was shot down over Germany in
1943 and who remained a POW until liberated by allied troops in 1945.

Uniquely, this "Housing First" initiative will include provision for a
range of on-site support services focused on the individual needs of
the residents and provided by MHI's partner organizations who are
professionals in their fields of work. And being "Housing First" means
that residents can remain in Veterans' House for an indeterminate
period and focus on recovery from the underlying issues that
contributed to their homelessness, be they economic, health, mental
health or addiction related issues. Moreover, in line with best
practices in providing affordable housing to this at-risk group, all
the ground floor bachelor apartments in Veterans' House will be fully
wheelchair accessible. Again, with a community spirit in mind, the site
will also include a gym, gardens, service-dog park, and communal
kitchen and BBQ area for when residents want to share meals together.

Together with our partners---many of them with direct connections to
the military and Veterans---Veterans' House: the Andy Carswell Building
will provide a "military-family" structure to residents which they will
recognize and which research has shown many homeless Veterans miss and
which contributes to their slide into homelessness. A fully list of our
partners, progress to date and lots of other information is on the MHI
Veterans' House website at
{\normalfont\color{blue}\texttt{\url{https://www.multifaithhousing.ca/veterans-house}}}


The physical structure is rising from the ground this spring. We broke
ground on September 5th, 2019 and 96-year old Squadron Leader (retd)
Andy Carswell laid the cornerstone on September 16th. As at February
1st, 2020 we are up to the second floor with completion planned for
November 2020. There's lots to do but to date we are pretty much on
schedule despite building through the winter months.

The total cost of the initiative is about \$15.4 million, including \$3
million for the land which was transferred from the federal government
as part of its contribution. There is a balance of \$2.3 million still
left to raise and we are in full fund-raising mode. If you are willing
and able there are a number of ways you can help financially: you can
send a cash or stock-transfer donations to Veterans' House through MHI;
organize a fund-raising event within your community; or leave a bequest
in your Will. All donations are gratefully received and will be put to
good use where they will make a difference.

While Veterans' House: the Andy Carswell Building is the first of its
kind our plan is to create a robust model that can be replicated across
Canada to address the national issue of homeless Veterans.

For more general info about Multifaith Housing or to make a donation go
to: www.multifaithhousing.ca You'll find more information there
including a recently released report on a 2019 Survey of homeless
Veterans in the National Capital Region.

To volunteer you can apply online. Go to:
{\normalfont\color{blue}\texttt{\url{https://www.multifaithhousing.ca/volunteer}}}

%\begin{centering}
%Twinkle, twinkle, old North Star,\\
%With wrinkled skin and creaky
%spar,\\
%For years you've always been on
%sked,\\
%As many of our force you've sped\\
%to distant lands beyond the skies,\\
%you've never failed to see the rise\\
%of distant suns or moonrise glare.\\
%You've carried us with tender care\\
%-- back you came to take us home,\\
%from old Japan or France or Rome\\
%from England, Goose, and Kef.\\
%We cannot hear you.\\
%('cause we're deaf)\\
%\end{centering}

%\begin{figure}[htbp]
%   \vspace{2em}
%   \centering
%   %name of the graphic, without the path AND in EPS format:
%   \includegraphics[scale=0.5]{jim_jung_2.eps}
%   %caption of the figure 
%   \caption*{\small \em A Whispering Giant (North Star 17515) at
%     Resolute Bay NWT in about 1963, taken by Radio Operator Jim Jung.}
%   %label of the figure, which has to correspond to \ref{}:
%   \label{fig:jim_jung}
%\end{figure}

\begin{footnotesize}
    \raggedleft PNSAC\\
\end{footnotesize}

% End of text.

%%% Local Variables: 
%%% mode: latex
%%% TeX-master: main_document.tex
%%% End: 

