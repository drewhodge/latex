% Template PNSAC newsletter - Article
% Language: Latex
%

% Head

\title{North Star Propeller Restorations}
%\subtitle{Part 4}
\author{Harry Hope, with assistance from Michael Hope.}

\maketitle

\textit{Hope Aero has played a major role in the restoration of the North Star.
The following article provides some background about the company and its work
on the North Star's propellers. }\\

Hope Aero was formally government approved to certify aircraft propellers in
1992, at their facility in Mississauga, Ontario. Harry Hope however, already
had 40 years of propeller overhaul experience on many varieties that included
Hamilton Standard propellers, which were quite common, as well as Hartzell,
McCauley, and Sensenich. Harry's first restoration to flying condition, was the
propeller for the Lysander, which was restored by the Air Force in Winnipeg for
their Centennial project in 1967. This Lysander is now displayed at CASM. Since
that time, they have restored several propellers, some to flying condition,
three are the Lancaster for the Canadian Warplane Heritage, Super Constellation
(Super Star) for Lufthansa, and the Honningstad C5 Polar for Norway which is on
display at the Aviation museum in Boda, Norway. 

 Since that time, their capabilities have added– MT, Dowty, Hoffman, and
Hamilton Standard Dash 7 and 8, and later the Dowty propeller on the Q400.  As
their customer base expanded to include different types of aircraft, the
requirement for them to service more types became evident. They kept up with
the required technology.  

Along the way, they also expanded into aircraft Wheel and Brake overhaul,
sales, service and on wing dynamic balancing of fixed wing and Helicopters.
Harry retired in 2001, but the third generation is now capably looking after
the business.

In the spring of 2006, Hope Aero took on the task of restoring the four North
Star propellers, for the "North Star Restoration" project. The first challenge
was to remove the propellers from the engines as neither the restoration crew
nor Hope Aero had any recent experience removing this type of propeller.
Michael Hope made the trip to Ottawa to assist with this operation and to help
prevent any damage, which turned out to be exceedingly difficult due to the
corrosion, starting with the spinner and then the propeller. Once the propeller
was removed the blades were removed for shipping to Hope Aero. 

The first 43D50 Hamilton Standard Hydromatic propeller was received at Hope
Aero in 2006. It was completely disassembled, cleaned and inspected. The
external parts were very badly corroded, with most damage being to the aluminum
blades and dome shells, and rust to the steel hub and tail shaft, which was
likely caused by the outside storage environment. The inside parts were in
relatively good condition. It was established, that both technical and visual
effort would be required to restore the propeller to an acceptable display
condition.

The restoration work commenced with the blades, the alcohol de-icer feed shoes,
troughs and paint being removed. The corrosion of the blade surfaces was such
that removal by grinding was abandoned, as the inner half of the blades are
shot peened and grinding would remove this and the authenticity. It was decided
to media blast the complete airfoil surfaces with an aluminum oxide media, thus
removing the surface corrosion but leaving a good surface for paint to be
adhered to; an aviation grade grey paint was applied with yellow stripes at the
tips. New alcohol feed shoes were installed, however the troughs, which direct
the alcohol to the shoes, were not available. As the troughs are hidden inside
the spinner, it was determined that they could be omitted.

The external hub, barrel halves, spider tail shaft, slinger ring, de-icer
tubes, and dome retaining nut, were extremely rusty. They were glass bead
blasted to remove the rust; this did not remove the heavy pitting but did leave
a surface that could be cadmium plated.  After plating, a clear coat material
was sprayed on these parts to improve the resistance to rust. The internal
surfaces were in good condition, and they were well oiled. 

A replacement dome shell in relatively good condition was found, requiring only
to be sanded, anodized, and be reinstalled with new seals.

The propeller was then completely assembled to confirm that all the refurbished
parts fit together. The blades were then removed to facilitate being shipping
back to the museum.

All of the four propellers were found to be in similar condition, and all were
restored in a similar manner.

\begin{footnotesize}
    \raggedleft PNSAC\\
\end{footnotesize}

% End of text.

%%% Local Variables: 
%%% mode: latex
%%% TeX-master: main_document.tex
%%% End: 

